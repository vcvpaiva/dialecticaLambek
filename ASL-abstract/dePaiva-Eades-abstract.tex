%% FIRST RENAME THIS FILE <yoursurname>.tex. 
%% BEFORE COMPLETING THIS TEMPLATE, SEE THE "READ ME" SECTION 
%% BELOW FOR INSTRUCTIONS. 
%% TO PROCESS THIS FILE YOU WILL NEED TO DOWNLOAD asl.cls from 
%% http://aslonline.org/abstractresources.html. 


\documentclass[bsl,meeting]{asl}

\AbstractsOn

\pagestyle{plain}

\def\urladdr#1{\endgraf\noindent{\it URL Address}: {\tt #1}.}


\newcommand{\NP}{}
%\usepackage{verbatim}

\begin{document}
\thispagestyle{empty}

%% BEGIN INSERTING YOUR ABSTRACT DIRECTLY BELOW; 
%% SEE INSTRUCTIONS (1), (2), (3), and (4) FOR PROPER FORMATS

\NP  
\absauth{Valeria de Paiva,  and Harley Eades III}
\meettitle{Dialectica categories for the Lambek calculus}
\affil{Sunnyvale Lab, Nuance Communications, 1198 E. Arques Ave, Sunnyvale, CA 94085, USA}
\meetemail{valeria.depaiva@gmail.com}
%%% NOTE: email required for at least one author
%\urladdr{OPTIONAL}
%
% Second author's affiliation
\affil{Computer and Information Sciences, Augusta University, 2500 Walton Way, Augusta, Georgia 30904, USA}
\meetemail{heades@augusta.edu}
%\urladdr{OPTIONAL}


%% INSERT TEXT OF ABSTRACT DIRECTLY BELOW
  Dialectica categorical models of the Lambek Calculus were first presented in the Amsterdam Colloquium \cite{depaiva1991}. Following Lambek's lead we approached the Lambek Calculus from the perspective of Linear Logic and adapted the Dialectica categorical models for Linear Logic to Lambek's non-commutative calculus \cite{lambek1988}. This  work took for granted the syntax of the Lambek calculus
and only discussed the exciting possibilities of new models for modalities that the work on Linear Logic 
introduced. Many years later we find that the work on dialectica models of the Lambek calculus is still interesting and that it might inform some of the most recent work on the relationship between Categorial Grammars and notions of Distributional Semantics (see e.g. \cite{coecke2013}). 
So we revisited the old work, making sure that the syntax details that were sketchy on the first version got completed and verified, using automated tools such as Agda and Ott.



We recall the Lambek calculus with its Curry-Howard isomorphic term assignment system. We extend it with a $\kappa$ modality, inspired by Yetter's work, which makes the calculus commutative. Then we add the of-course modality $!$,  as Girard did, to  re-introduce weakening and contraction for all formulas and get back the full power of intuitionistic and classical logic. We also present algebraic semantics and categorical semantics, proved sound and complete. Finally we show the traditional properties of type systems, like subject reduction, the Church-Rosser theorem  and  normalization for the calculi of extended modalities, which we did not have before. 
Ultimately we are  interested on the applicability  of the original systems to their intended uses in the construction of semantics of Natural Language. But before we can discuss that, we need to make sure that the mathematical properties that make the Lambek calculus attractive are all properly modeled. 


\begin{thebibliography}{10}

%% INSERT YOUR BIBLIOGRAPHIC ENTRIES HERE; 
%% SEE (4) BELOW FOR PROPER FORMAT.
%% EACH ENTRY MUST BEGIN WITH \bibitem{citation key}
%%
%% IF THERE ARE NO ENTRIES  
%% DELETE THE LINE ABOVE (\begin{thebibliography}{20}) 
%% AND THE LINE BELOW (\end{thebibliography})


\bibitem{depaiva1991}
{\scshape Valeria de Paiva},
{\itshape A {D}ialectica model of the {L}ambek calculus},
{\bfseries\itshape Proceedings of 8th Amsterdam Logic Colloquium}
(Paul Dekker and Martin Stokhof, editors),
ILLC, Amsterdam,
1991,
pp.~445--462.

\bibitem{lambek1988}
{\scshape Joachim Lambek},
{\itshape Categorial and Categorical Grammars},
{\bfseries\itshape Categorial Grammars and Natural Language Structures}
(Richard Oehrle,  Emmon Bach and Deirdre Wheeler, editors),
Springer Netherlands,
Dordrecht,
1988, pp.~297--317.


\bibitem{coecke2013}
{\scshape Bob Coecke, Edward Grefenstette  and Mehrnoosh Sadrzadeh},
{\itshape Lambek vs. Lambek: Functorial vector space semantics and string diagrams for {L}ambek calculus},
{\bfseries\itshape Annals of Pure and Applied Logic},
vol.~164 (2013), no.~11, pp.~1079--1100.	
	


\end{thebibliography}


\vspace*{-0.5\baselineskip}
% this space adjustment is usually necessary after a bibliography

\end{document}
@inproceedings{Sewell:2010,
	Author = {P. Sewell and F. Nardelli and S. Owens and G. Peskine and T. Ridge and S. Sarkar and R. Strnisa},
	Booktitle = {Journal of Functional Programming},
	Number = {1},
	Pages = {71-122},
	Title = {Ott: Effective tool support for the working semanticist},
	Volume = {20},
	Year = {2010}}

%% READ ME
%% READ ME
%% READ ME

INSTRUCTIONS FOR SUPPLYING INFORMATION IN THE CORRECT FORMAT: 

1. Author names are listed as First Last, First Last, and First Last.

\absauth{Valeria de Paiva,  and Harley Eades III}


2. Titles of abstracts have ONLY the first letter capitalized,
except for Proper Nouns.

\meettitle{Dialectica categories for the Lambek calculus} 


3. Affiliations and email addresses for authors of abstracts are  listed separately.

% First author's affiliation
\affil{Sunnyvale Lab, Nuance Communicatons, 1198 E. Arques Ave, Sunnyvale, CA 94085, USA}
\meetemail{valeria.depaiva@gmail.com}
%%% NOTE: email required for at least one author
\urladdr{OPTIONAL}
%
% Second author's affiliation
\affil{Computer and Information Sciences, Augusta University, 2500 Walton Way, Augusta, Georgia 30904, USA}
\meetemail{heades@augusta.edu}
\urladdr{OPTIONAL}
%



4. Bibliographic Entries

%%%% IF references are submitted with abstract,
%%%% please use the following formats

%%% For a Journal article
\bibitem{cite1}
{\scshape Author's Name},
{\itshape Title of article},
{\bfseries\itshape Journal name spelled out, no abbreviations},
vol.~XX (XXXX), no.~X, pp.~XXX--XXX.

%%% For a Journal article by the same authors as above,
%%% i.e., authors in cite1 are the same for cite2
\bibitem{cite2}
\bysame
{\itshape Title of article},
{\bfseries\itshape Journal},
vol.~XX (XXXX), no.~X, pp.~XX--XXX.

%%% For a book
\bibitem{cite3}
{\scshape Author's Name},
{\bfseries\itshape Title of book},
Name of series,
Publisher,
Year.

%%% For an article in proceedings
\bibitem{cite4}
{\scshape Author's Name},
{\itshape Title of article},
{\bfseries\itshape Name of proceedings}
(Address of meeting),
(First Last and First2 Last2, editors),
vol.~X,
Publisher,
Year,
pp.~X--XX.

%%% For an article in a collection
\bibitem{cite5}
{\scshape Author's Name},
{\itshape Title of article},
{\bfseries\itshape Book title}
(First Last and First2 Last2, editors),
Publisher,
Publisher's address,
Year,
pp.~X--XX.

%%% An edited book
\bibitem{cite6}
Author's name, editor. % No special font used here
{\bfseries\itshape Title of book},
Publisher,
Publisher's address,
Year.

