\documentclass{article}
\usepackage[utf8]{inputenc}

\title{More Dialectica Models of the Lambek Calculus}
\author{Valeria de Paiva and Harley Eades III}
\date{April 2016}

\usepackage{natbib}
\usepackage{graphicx}

\begin{document}

\maketitle

\section*{Introduction}
This   note recalls a Dialectica model of the Lambek Calculus presented by the first author in the Amsterdam Colloquium 1991,
from the point-of-view of Linear Logic. Twenty five years later we find that the work is still interesting and that it might inform some of the most recent work on word vectors. So we have decided to retype some of the old results and see if our new implementations can shed new light on some of the issues that were not solved. 

\subsection*{Brief Historical Overview}
The Lambek calculus was first introduced  with the name of Syntactic Calculus, by Jim Lambek in 1958  as an explanation of the mathematics of sentence structure.  After a long period of ostracism, around 1980 the Syntactic Calculus, now called the Lambek Calculus was taken up by logicians interested in Computational Linguistics, especially the ones in the area of Categorial Grammar. 
%The calculus was considered ``like a logical system" but much too weak to be taken seriously as a logic. To quote from van Benthem
% \begin{quotation}One fundamental system of this kind is the so-called `Lambek Calculus' whose type-change rules show a  close analogy with the inference rules of constructive propositional logic.
% \end{quotation}

The work on Categorial Grammar was given a serious impulse by  the advent of Girard's Linear Logic at the end of the 1980s.  Girard showed that there is a full embedding, preserving proofs, of Intuitionistic Logic into Linear Logic with a modality ``!", which meant that one could consider several systems of resource logics. These refined resource logics were applied to several areas of Computer Science. The Lambek calculus has seen a significant number of works written about it, apart from a number of monographs that
deal with logical and linguistic aspects of the type-logical
approach.
%For the general logical background, there is Languagein Action.
Type Logical Grammar situates
the type-logical
approach
within
the framework of Montague's
Universal Grammar
and
presents detailed
linguistic
analyses
for a substantive fragment of syntactic
and semantic phenomena
in the grammar
of English.
Type Logical Semantics
 offers
a general
introduction
to natural language
semantics studied
from a type-logical
perspective. Moortgat's Categorial
Type Logics, in the Handbook of Logic ad Language


However, fashion turned against these systems with the rise of probabilistic and machine learning systems.

This meant that a series of systems, implemented or not, were devised  that used the Lambek Calculus or variants of Linear Logic. These systems can be   as expressive as Intuitionistic Logic and the claim is that they are more precise i.e. they make finer distinctions.
From the beginning it was clear that the Lambek Calculus is the multiplicative fragment of non-commutative Intuitionistic Linear Logic. 
Hence several interesting questions, considered for Linear Logic,  could also be asked of the Lambek Calculus. 
One of them, posed by Morrill et al  is whether we can extend the Lambek calculus with a modality that does for the structural rule of \textit{(exchange)} what the modality \textit{of course} `!' does for the rules of \textit{(weakening)} and \textit{(contraction)}.
A very preliminary proposal, which answers this question affirmatively, is set forward in this paper. But it must be said from the start that the `answer' is  provided in semantical terms. 
%The Proof Theory of the systems considered should be investigated in future work. Another warning is that the perspective of this note is basically from Category Theory as a branch of Mathematics, so words like categories and functors are always meant in their mathematical, rather than linguistical or philosophical sense.


We first recall Linear Logic and provide the transformations to show that the Lambek Calculus \textsf{L} really is the multiplicative fragment of non-commutative Intuitionistic Linear Logic. 
In the second section we describe the usual String Semantics for the Lambek Calculus \textsf{L} and generalise it, using a categorical perspective. 
In the third section we describe our Dialectica model for the Lambek Calculus. 
In the last section we discuss modalities and some untidiness of the Curry-Howard correspondence for the fragments of Linear Logic in question.

\iffalse
\noindent{\textbf{Acknowledgments}} I would like to thank Jan van Eijck for inviting me to give the talk that became
this note, thereby gently `forcing' me to think about the subject, as well as for his
generous hospitality. I also would like to thank Martin Hyland, Harold Schellinx,
Dirk Roorda, Mark Hepple, Glyn Morrill and Michael Moortgat for several useful
discussions. Many of the ideas in this paper have been shaped by these discussions,
but of course the mistakes are all mine. Finally I want to thank Jim Lambek for     `putting me right'
 in the most friendly possible way on how completeness has nothing to do with the existence of two disjunctions.
\fi 

\section{The Lambek Calculus}


\section{Algebraic Semantics}

\section{A Dialectica Construction}

\section{Modalities}
\section{Conclusion}


\bibliographystyle{plain}
\bibliography{references}
\end{document}
\begin{figure}[h!]
\centering
\includegraphics[scale=1.7]{universe.jpg}
\caption{The Universe}
\label{fig:univerise}
\end{figure}