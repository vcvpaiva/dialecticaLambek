\documentclass{article}
\usepackage[utf8]{inputenc}

\title{Agda and Dialectica Categories \\for the Lambek Calculus}
\author{Valeria de Paiva and Harley Eades III}
\date{August 2016}

\usepackage{amssymb,amsmath,amsthm}
\usepackage{natbib}
\usepackage{graphicx}
\usepackage{hyperref}
\usepackage{mathpartir}
\usepackage[barr]{xy}
\usepackage{mdframed}
\usepackage{bbm}

\let\mto\to                     % Used for arrows
\let\to\relax                   % Used for implication
\newcommand{\to}{\rightarrow}
\newcommand{\id}{\mathsf{id}}
\newcommand{\V}{\text{V}^+}
\newcommand{\N}{\mathcal{N}}
\newcommand{\pow}[1]{\mathcal{P}(#1)}
\newcommand{\rto}{\leftharpoonup}
\newcommand{\lto}{\rightharpoonup}
\newcommand{\LB}{\text{L}_!}
\newcommand{\LE}{\text{L}_\kappa}
\newcommand{\redto}{\rightsquigarrow}
\newcommand{\cat}[1]{\mathcal{#1}}
\newcommand{\Set}{\mathsf{Set}}
\newcommand{\Dial}[2]{\mathsf{Dial}_{#1}(#2)}
\newcommand{\Hom}[3]{\mathsf{Hom}_{#1}(#2,#3)}

\newtheorem{theorem}{Theorem}
\newtheorem{lemma}[theorem]{Lemma}
\newtheorem{corollary}[theorem]{Corollary}
\newtheorem{definition}[theorem]{Definition}
\newtheorem{proposition}[theorem]{Proposition}
\newtheorem{example}[theorem]{Example}


%% Ott includes:
% generated by Ott 0.24 from: L.ott
\newcommand{\Ldrule}[4][]{{\displaystyle\frac{\begin{array}{l}#2\end{array}}{#3}\quad\Ldrulename{#4}}}
\newcommand{\Lusedrule}[1]{\[#1\]}
\newcommand{\Lpremise}[1]{ #1 \\}
\newenvironment{Ldefnblock}[3][]{ \framebox{\mbox{#2}} \quad #3 \\[0pt]}{}
\newenvironment{Lfundefnblock}[3][]{ \framebox{\mbox{#2}} \quad #3 \\[0pt]\begin{displaymath}\begin{array}{l}}{\end{array}\end{displaymath}}
\newcommand{\Lfunclause}[2]{ #1 \equiv #2 \\}
\newcommand{\Lnt}[1]{\mathit{#1}}
\newcommand{\Lmv}[1]{\mathit{#1}}
\newcommand{\Lkw}[1]{\mathbf{#1}}
\newcommand{\Lsym}[1]{#1}
\newcommand{\Lcom}[1]{\text{#1}}
\newcommand{\Ldrulename}[1]{\textsc{#1}}
\newcommand{\Lcomplu}[5]{\overline{#1}^{\,#2\in #3 #4 #5}}
\newcommand{\Lcompu}[3]{\overline{#1}^{\,#2<#3}}
\newcommand{\Lcomp}[2]{\overline{#1}^{\,#2}}
\newcommand{\Lgrammartabular}[1]{\begin{supertabular}{llcllllll}#1\end{supertabular}}
\newcommand{\Lmetavartabular}[1]{\begin{supertabular}{ll}#1\end{supertabular}}
\newcommand{\Lrulehead}[3]{$#1$ & & $#2$ & & & \multicolumn{2}{l}{#3}}
\newcommand{\Lprodline}[6]{& & $#1$ & $#2$ & $#3 #4$ & $#5$ & $#6$}
\newcommand{\Lfirstprodline}[6]{\Lprodline{#1}{#2}{#3}{#4}{#5}{#6}}
\newcommand{\Llongprodline}[2]{& & $#1$ & \multicolumn{4}{l}{$#2$}}
\newcommand{\Lfirstlongprodline}[2]{\Llongprodline{#1}{#2}}
\newcommand{\Lbindspecprodline}[6]{\Lprodline{#1}{#2}{#3}{#4}{#5}{#6}}
\newcommand{\Lprodnewline}{\\}
\newcommand{\Linterrule}{\\[5.0mm]}
\newcommand{\Lafterlastrule}{\\}
\newcommand{\Lmetavars}{
\Lmetavartabular{
 $ \Lmv{vars} ,\, \Lmv{x} ,\, \Lmv{y} ,\, \Lmv{z} ,\, \Lmv{w} $ &  \\
 $ \Lmv{ivar} ,\, \Lmv{i} ,\, \Lmv{j} ,\, \Lmv{l} $ &  \\
}}

\newcommand{\LA}{
\Lrulehead{\Lnt{A}  ,\ \Lnt{B}  ,\ \Lnt{C}  ,\ \Lnt{D}}{::=}{}\Lprodnewline
\Lfirstprodline{|}{ I }{}{}{}{}\Lprodnewline
\Lprodline{|}{ \Lnt{A}  \otimes  \Lnt{B} }{}{}{}{}\Lprodnewline
\Lprodline{|}{ \Lnt{A}  \rightharpoonup  \Lnt{B} }{}{}{}{}\Lprodnewline
\Lprodline{|}{ \Lnt{A}  \leftharpoonup  \Lnt{B} }{}{}{}{}\Lprodnewline
\Lprodline{|}{ !  \Lnt{A} }{}{}{}{}\Lprodnewline
\Lprodline{|}{ \kappa  \Lnt{A} }{}{}{}{}\Lprodnewline
\Lprodline{|}{\Lsym{(}  \Lnt{A}  \Lsym{)}} {\textsf{M}}{}{}{}\Lprodnewline
\Lprodline{|}{ \Lnt{A} } {\textsf{M}}{}{}{}}

\newcommand{\Lp}{
\Lrulehead{\Lnt{p}}{::=}{}\Lprodnewline
\Lfirstprodline{|}{ \mathsf{unit} }{}{}{}{}\Lprodnewline
\Lprodline{|}{\Lsym{-}}{}{}{}{}\Lprodnewline
\Lprodline{|}{\Lmv{x}}{}{}{}{}\Lprodnewline
\Lprodline{|}{ \Lnt{p_{{\mathrm{1}}}}  \otimes  \Lnt{p_{{\mathrm{2}}}} }{}{}{}{}}

\newcommand{\Lt}{
\Lrulehead{\Lnt{t}}{::=}{}\Lprodnewline
\Lfirstprodline{|}{\Lmv{x}}{}{}{}{}\Lprodnewline
\Lprodline{|}{ \mathsf{unit} }{}{}{}{}\Lprodnewline
\Lprodline{|}{ \mathsf{app}_l\, \Lnt{t_{{\mathrm{1}}}} \, \Lnt{t_{{\mathrm{2}}}} }{}{}{}{}\Lprodnewline
\Lprodline{|}{ \mathsf{app}_r\, \Lnt{t_{{\mathrm{1}}}} \, \Lnt{t_{{\mathrm{2}}}} }{}{}{}{}\Lprodnewline
\Lprodline{|}{ \Lnt{t_{{\mathrm{1}}}}  \otimes  \Lnt{t_{{\mathrm{2}}}} }{}{}{}{}\Lprodnewline
\Lprodline{|}{ \lambda_l  \Lmv{x} : \Lnt{A} . \Lnt{t} }{}{}{}{}\Lprodnewline
\Lprodline{|}{ \lambda_r  \Lmv{x} : \Lnt{A} . \Lnt{t} }{}{}{}{}\Lprodnewline
\Lprodline{|}{ \mathsf{let}\, \Lnt{t_{{\mathrm{1}}}} \,\mathsf{be}\, \Lnt{p} \,\mathsf{in}\, \Lnt{t_{{\mathrm{2}}}} }{}{}{}{}\Lprodnewline
\Lprodline{|}{ \mathsf{copy}\, \Lnt{t'} \,\mathsf{as}\, \Lmv{y} , \Lmv{z} \,\mathsf{in}\, \Lnt{t} }{}{}{}{}\Lprodnewline
\Lprodline{|}{ \mathsf{discard}\, \Lnt{t'} \,\mathsf{in}\, \Lnt{t} }{}{}{}{}\Lprodnewline
\Lprodline{|}{ \mathsf{discard}\,\vec{ \Lnt{t'} }\,\mathsf{in}\, \Lnt{t} }{}{}{}{}\Lprodnewline
\Lprodline{|}{ \mathsf{promote}_!\,\vec{ \Lmv{x} }\,\mathsf{for}\,\vec{ \Lmv{y} }\,\mathsf{in}\, \Lnt{t} }{}{}{}{}\Lprodnewline
\Lprodline{|}{ \mathsf{derelict}_!\, \Lnt{t} }{}{}{}{}\Lprodnewline
\Lprodline{|}{ \mathsf{promote}_\kappa\,\vec{ \Lmv{x} }\,\mathsf{for}\,\vec{ \Lmv{y} }\,\mathsf{in}\, \Lnt{t} }{}{}{}{}\Lprodnewline
\Lprodline{|}{ \mathsf{derelict}_\kappa\, \Lnt{t} }{}{}{}{}\Lprodnewline
\Lprodline{|}{ \mathsf{exchange_r}\, \Lmv{x} \,\mathsf{with}\, \Lmv{y} \,\mathsf{in}\, \Lnt{t} }{}{}{}{}\Lprodnewline
\Lprodline{|}{ \mathsf{exchange_l}\, \Lmv{x} \,\mathsf{with}\, \Lmv{y} \,\mathsf{in}\, \Lnt{t} }{}{}{}{}\Lprodnewline
\Lprodline{|}{ (  \Lnt{t}  ) } {\textsf{S}}{}{}{}\Lprodnewline
\Lprodline{|}{\Lsym{[}  \Lnt{t_{{\mathrm{1}}}}  \Lsym{/}  \Lmv{x}  \Lsym{]}  \Lnt{t_{{\mathrm{2}}}}} {\textsf{M}}{}{}{}\Lprodnewline
\Lprodline{|}{ \Lnt{t} } {\textsf{M}}{}{}{}}

\newcommand{\LG}{
\Lrulehead{\Gamma}{::=}{}\Lprodnewline
\Lfirstprodline{|}{ \cdot }{}{}{}{}\Lprodnewline
\Lprodline{|}{\Lnt{A}}{}{}{}{}\Lprodnewline
\Lprodline{|}{\Lmv{x}  \Lsym{:}  \Lnt{A}}{}{}{}{}\Lprodnewline
\Lprodline{|}{\Gamma  \Lsym{,}  \Gamma'}{}{}{}{}\Lprodnewline
\Lprodline{|}{ \vec{ \Lmv{x} } : ! \Gamma } {\textsf{M}}{}{}{}\Lprodnewline
\Lprodline{|}{ \vec{ \Lmv{x} } : \kappa  \Gamma } {\textsf{M}}{}{}{}\Lprodnewline
\Lprodline{|}{\Lsym{(}  \Gamma  \Lsym{)}} {\textsf{S}}{}{}{}\Lprodnewline
\Lprodline{|}{ !  \Gamma } {\textsf{M}}{}{}{}\Lprodnewline
\Lprodline{|}{ \kappa  \Gamma } {\textsf{M}}{}{}{}\Lprodnewline
\Lprodline{|}{ \Gamma } {\textsf{M}}{}{}{}}

\newcommand{\Lterminals}{
\Lrulehead{\Lnt{terminals}}{::=}{}\Lprodnewline
\Lfirstprodline{|}{ \otimes }{}{}{}{}\Lprodnewline
\Lprodline{|}{\Lsym{1}}{}{}{}{}\Lprodnewline
\Lprodline{|}{\Lsym{0}}{}{}{}{}\Lprodnewline
\Lprodline{|}{ \Box }{}{}{}{}\Lprodnewline
\Lprodline{|}{ \models }{}{}{}{}\Lprodnewline
\Lprodline{|}{ \vdash }{}{}{}{}\Lprodnewline
\Lprodline{|}{ \mathop{\star} }{}{}{}{}\Lprodnewline
\Lprodline{|}{ \gg }{}{}{}{}\Lprodnewline
\Lprodline{|}{ \Box }{}{}{}{}\Lprodnewline
\Lprodline{|}{ \lambda }{}{}{}{}\Lprodnewline
\Lprodline{|}{ \redto }{}{}{}{}\Lprodnewline
\Lprodline{|}{ \redto^* }{}{}{}{}\Lprodnewline
\Lprodline{|}{ \succeq }{}{}{}{}\Lprodnewline
\Lprodline{|}{ \preceq }{}{}{}{}\Lprodnewline
\Lprodline{|}{ \rightarrow }{}{}{}{}\Lprodnewline
\Lprodline{|}{ \leftarrow }{}{}{}{}\Lprodnewline
\Lprodline{|}{ \Yleft }{}{}{}{}\Lprodnewline
\Lprodline{|}{ \redto^* }{}{}{}{}\Lprodnewline
\Lprodline{|}{\Lsym{=}}{}{}{}{}\Lprodnewline
\Lprodline{|}{ \equiv }{}{}{}{}\Lprodnewline
\Lprodline{|}{ \cong }{}{}{}{}\Lprodnewline
\Lprodline{|}{ \mathop{\perp} }{}{}{}{}\Lprodnewline
\Lprodline{|}{ \top }{}{}{}{}\Lprodnewline
\Lprodline{|}{ \bullet }{}{}{}{}\Lprodnewline
\Lprodline{|}{ \mathop{\sim} }{}{}{}{}\Lprodnewline
\Lprodline{|}{ \lambda }{}{}{}{}\Lprodnewline
\Lprodline{|}{ \mu }{}{}{}{}\Lprodnewline
\Lprodline{|}{ \neq }{}{}{}{}\Lprodnewline
\Lprodline{|}{ \mathsf{FV} }{}{}{}{}\Lprodnewline
\Lprodline{|}{ \in }{}{}{}{}}

\newcommand{\Lformula}{
\Lrulehead{\Lnt{formula}}{::=}{}\Lprodnewline
\Lfirstprodline{|}{\Lnt{judgement}}{}{}{}{}\Lprodnewline
\Lprodline{|}{\Lnt{formula}  \preceq  \Lnt{formula'}} {\textsf{M}}{}{}{}\Lprodnewline
\Lprodline{|}{\Lsym{(}  \Lnt{formula}  \Lsym{)}} {\textsf{M}}{}{}{}\Lprodnewline
\Lprodline{|}{ \Lnt{formula}  \qquad  \Lnt{formula'} }{}{}{}{}\Lprodnewline
\Lprodline{|}{\Lsym{(}  \Lnt{formula}  \Lsym{)}}{}{}{}{}\Lprodnewline
\Lprodline{|}{ \Lmv{x}  \not\in \mathsf{FV}( \Lnt{t} ) }{}{}{}{}}

\newcommand{\LJtyp}{
\Lrulehead{\Lnt{Jtyp}}{::=}{}\Lprodnewline
\Lfirstprodline{|}{ \Gamma  \vdash  \Lnt{A} }{}{}{}{}\Lprodnewline
\Lprodline{|}{ \Gamma  \vdash  \Lnt{t}  :  \Lnt{A} }{}{}{}{}\Lprodnewline
\Lprodline{|}{ \Lnt{t_{{\mathrm{1}}}}  \rightsquigarrow  \Lnt{t_{{\mathrm{2}}}} }{}{}{}{}}

\newcommand{\Ljudgement}{
\Lrulehead{\Lnt{judgement}}{::=}{}\Lprodnewline
\Lfirstprodline{|}{\Lnt{Jtyp}}{}{}{}{}}

\newcommand{\LuserXXsyntax}{
\Lrulehead{\Lnt{user\_syntax}}{::=}{}\Lprodnewline
\Lfirstprodline{|}{\Lmv{vars}}{}{}{}{}\Lprodnewline
\Lprodline{|}{\Lmv{ivar}}{}{}{}{}\Lprodnewline
\Lprodline{|}{\Lnt{A}}{}{}{}{}\Lprodnewline
\Lprodline{|}{\Lnt{p}}{}{}{}{}\Lprodnewline
\Lprodline{|}{\Lnt{t}}{}{}{}{}\Lprodnewline
\Lprodline{|}{\Gamma}{}{}{}{}\Lprodnewline
\Lprodline{|}{\Lnt{terminals}}{}{}{}{}\Lprodnewline
\Lprodline{|}{\Lnt{formula}}{}{}{}{}}

\newcommand{\Lgrammar}{\Lgrammartabular{
\LA\Linterrule
\Lp\Linterrule
\Lt\Linterrule
\LG\Linterrule
\Lterminals\Linterrule
\Lformula\Linterrule
\LJtyp\Linterrule
\Ljudgement\Linterrule
\LuserXXsyntax\Lafterlastrule
}}

% defnss
% defns Jtyp
%% defn Log
\newcommand{\Ldruleax}[1]{\Ldrule[#1]{%
}{
 \Lnt{A}  \vdash  \Lnt{A} }{%
{\Ldrulename{ax}}{}%
}}


\newcommand{\LdruleUr}[1]{\Ldrule[#1]{%
}{
  \cdot   \vdash   I  }{%
{\Ldrulename{Ur}}{}%
}}


\newcommand{\Ldrulecut}[1]{\Ldrule[#1]{%
\Lpremise{  \Gamma_{{\mathrm{2}}}  \vdash  \Lnt{A}   \qquad   \Gamma_{{\mathrm{1}}}  \Lsym{,}  \Lnt{A}  \Lsym{,}  \Gamma_{{\mathrm{3}}}  \vdash  \Lnt{B}  }%
}{
 \Gamma_{{\mathrm{1}}}  \Lsym{,}  \Gamma_{{\mathrm{2}}}  \Lsym{,}  \Gamma_{{\mathrm{3}}}  \vdash  \Lnt{B} }{%
{\Ldrulename{cut}}{}%
}}


\newcommand{\LdruleUl}[1]{\Ldrule[#1]{%
\Lpremise{ \Gamma_{{\mathrm{1}}}  \Lsym{,}  \Gamma_{{\mathrm{2}}}  \vdash  \Lnt{A} }%
}{
 \Gamma_{{\mathrm{1}}}  \Lsym{,}   I   \Lsym{,}  \Gamma_{{\mathrm{2}}}  \vdash  \Lnt{A} }{%
{\Ldrulename{Ul}}{}%
}}


\newcommand{\LdruleTl}[1]{\Ldrule[#1]{%
\Lpremise{ \Gamma  \Lsym{,}  \Lnt{A}  \Lsym{,}  \Lnt{B}  \Lsym{,}  \Gamma'  \vdash  \Lnt{C} }%
}{
 \Gamma  \Lsym{,}   \Lnt{A}  \otimes  \Lnt{B}   \Lsym{,}  \Gamma'  \vdash  \Lnt{C} }{%
{\Ldrulename{Tl}}{}%
}}


\newcommand{\LdruleTr}[1]{\Ldrule[#1]{%
\Lpremise{  \Gamma_{{\mathrm{1}}}  \vdash  \Lnt{A}   \qquad   \Gamma_{{\mathrm{2}}}  \vdash  \Lnt{B}  }%
}{
 \Gamma_{{\mathrm{1}}}  \Lsym{,}  \Gamma_{{\mathrm{2}}}  \vdash   \Lnt{A}  \otimes  \Lnt{B}  }{%
{\Ldrulename{Tr}}{}%
}}


\newcommand{\LdruleIRl}[1]{\Ldrule[#1]{%
\Lpremise{  \Gamma_{{\mathrm{2}}}  \vdash  \Lnt{A}   \qquad   \Gamma_{{\mathrm{1}}}  \Lsym{,}  \Lnt{B}  \Lsym{,}  \Gamma_{{\mathrm{3}}}  \vdash  \Lnt{C}  }%
}{
 \Gamma_{{\mathrm{1}}}  \Lsym{,}   \Lnt{A}  \rightharpoonup  \Lnt{B}   \Lsym{,}  \Gamma_{{\mathrm{2}}}  \Lsym{,}  \Gamma_{{\mathrm{3}}}  \vdash  \Lnt{C} }{%
{\Ldrulename{IRl}}{}%
}}


\newcommand{\LdruleILl}[1]{\Ldrule[#1]{%
\Lpremise{  \Gamma_{{\mathrm{2}}}  \vdash  \Lnt{A}   \qquad   \Gamma_{{\mathrm{1}}}  \Lsym{,}  \Lnt{B}  \Lsym{,}  \Gamma_{{\mathrm{3}}}  \vdash  \Lnt{C}  }%
}{
 \Gamma_{{\mathrm{1}}}  \Lsym{,}  \Gamma_{{\mathrm{2}}}  \Lsym{,}   \Lnt{B}  \leftharpoonup  \Lnt{A}   \Lsym{,}  \Gamma_{{\mathrm{3}}}  \vdash  \Lnt{C} }{%
{\Ldrulename{ILl}}{}%
}}


\newcommand{\LdruleIRr}[1]{\Ldrule[#1]{%
\Lpremise{ \Gamma  \Lsym{,}  \Lnt{A}  \vdash  \Lnt{B} }%
}{
 \Gamma  \vdash   \Lnt{A}  \rightharpoonup  \Lnt{B}  }{%
{\Ldrulename{IRr}}{}%
}}


\newcommand{\LdruleILr}[1]{\Ldrule[#1]{%
\Lpremise{ \Lnt{A}  \Lsym{,}  \Gamma  \vdash  \Lnt{B} }%
}{
 \Gamma  \vdash   \Lnt{B}  \leftharpoonup  \Lnt{A}  }{%
{\Ldrulename{ILr}}{}%
}}


\newcommand{\LdruleC}[1]{\Ldrule[#1]{%
\Lpremise{ \Gamma_{{\mathrm{1}}}  \Lsym{,}   !  \Lnt{A}   \Lsym{,}  \Gamma_{{\mathrm{2}}}  \Lsym{,}   !  \Lnt{A}   \Lsym{,}  \Gamma_{{\mathrm{3}}}  \vdash  \Lnt{B} }%
}{
 \Gamma_{{\mathrm{1}}}  \Lsym{,}   !  \Lnt{A}   \Lsym{,}  \Gamma_{{\mathrm{2}}}  \Lsym{,}  \Gamma_{{\mathrm{3}}}  \vdash  \Lnt{B} }{%
{\Ldrulename{C}}{}%
}}


\newcommand{\LdruleW}[1]{\Ldrule[#1]{%
\Lpremise{ \Gamma_{{\mathrm{1}}}  \Lsym{,}  \Gamma_{{\mathrm{2}}}  \vdash  \Lnt{B} }%
}{
 \Gamma_{{\mathrm{1}}}  \Lsym{,}   !  \Lnt{A}   \Lsym{,}  \Gamma_{{\mathrm{2}}}  \vdash  \Lnt{B} }{%
{\Ldrulename{W}}{}%
}}


\newcommand{\LdruleBr}[1]{\Ldrule[#1]{%
\Lpremise{  !  \Gamma   \vdash  \Lnt{B} }%
}{
  !  \Gamma   \vdash   !  \Lnt{B}  }{%
{\Ldrulename{Br}}{}%
}}


\newcommand{\LdruleBl}[1]{\Ldrule[#1]{%
\Lpremise{ \Gamma_{{\mathrm{1}}}  \Lsym{,}  \Lnt{A}  \Lsym{,}  \Gamma_{{\mathrm{2}}}  \vdash  \Lnt{B} }%
}{
 \Gamma  \Lsym{,}   !  \Lnt{A}   \Lsym{,}  \Gamma_{{\mathrm{2}}}  \vdash  \Lnt{B} }{%
{\Ldrulename{Bl}}{}%
}}


\newcommand{\LdruleEr}[1]{\Ldrule[#1]{%
\Lpremise{  \kappa  \Gamma   \vdash  \Lnt{B} }%
}{
  \kappa  \Gamma   \vdash   \kappa  \Lnt{B}  }{%
{\Ldrulename{Er}}{}%
}}


\newcommand{\LdruleEl}[1]{\Ldrule[#1]{%
\Lpremise{ \Gamma_{{\mathrm{1}}}  \Lsym{,}  \Lnt{A}  \Lsym{,}  \Gamma_{{\mathrm{2}}}  \vdash  \Lnt{B} }%
}{
 \Gamma_{{\mathrm{1}}}  \Lsym{,}   \kappa  \Lnt{A}   \Lsym{,}  \Gamma_{{\mathrm{2}}}  \vdash  \Lnt{B} }{%
{\Ldrulename{El}}{}%
}}


\newcommand{\LdruleEOne}[1]{\Ldrule[#1]{%
\Lpremise{ \Gamma_{{\mathrm{1}}}  \Lsym{,}   \kappa  \Lnt{A}   \Lsym{,}  \Lnt{B}  \Lsym{,}  \Gamma_{{\mathrm{2}}}  \vdash  \Lnt{C} }%
}{
 \Gamma_{{\mathrm{1}}}  \Lsym{,}  \Lnt{B}  \Lsym{,}   \kappa  \Lnt{A}   \Lsym{,}  \Gamma_{{\mathrm{2}}}  \vdash  \Lnt{C} }{%
{\Ldrulename{E1}}{}%
}}


\newcommand{\LdruleETwo}[1]{\Ldrule[#1]{%
\Lpremise{ \Gamma_{{\mathrm{1}}}  \Lsym{,}  \Lnt{A}  \Lsym{,}   \kappa  \Lnt{B}   \Lsym{,}  \Gamma_{{\mathrm{2}}}  \vdash  \Lnt{C} }%
}{
 \Gamma_{{\mathrm{1}}}  \Lsym{,}   \kappa  \Lnt{B}   \Lsym{,}  \Lnt{A}  \Lsym{,}  \Gamma_{{\mathrm{2}}}  \vdash  \Lnt{C} }{%
{\Ldrulename{E2}}{}%
}}

\newcommand{\LdefnLog}[1]{\begin{Ldefnblock}[#1]{$ \Gamma  \vdash  \Lnt{A} $}{}
\Lusedrule{\Ldruleax{}}
\Lusedrule{\LdruleUr{}}
\Lusedrule{\Ldrulecut{}}
\Lusedrule{\LdruleUl{}}
\Lusedrule{\LdruleTl{}}
\Lusedrule{\LdruleTr{}}
\Lusedrule{\LdruleIRl{}}
\Lusedrule{\LdruleILl{}}
\Lusedrule{\LdruleIRr{}}
\Lusedrule{\LdruleILr{}}
\Lusedrule{\LdruleC{}}
\Lusedrule{\LdruleW{}}
\Lusedrule{\LdruleBr{}}
\Lusedrule{\LdruleBl{}}
\Lusedrule{\LdruleEr{}}
\Lusedrule{\LdruleEl{}}
\Lusedrule{\LdruleEOne{}}
\Lusedrule{\LdruleETwo{}}
\end{Ldefnblock}}

%% defn JType
\newcommand{\LdruleTXXvar}[1]{\Ldrule[#1]{%
}{
 \Lmv{x}  \Lsym{:}  \Lnt{A}  \vdash  \Lmv{x}  :  \Lnt{A} }{%
{\Ldrulename{T\_var}}{}%
}}


\newcommand{\LdruleTXXUr}[1]{\Ldrule[#1]{%
}{
  \cdot   \vdash   \mathsf{unit}   :   I  }{%
{\Ldrulename{T\_Ur}}{}%
}}


\newcommand{\LdruleTXXcut}[1]{\Ldrule[#1]{%
\Lpremise{  \Gamma_{{\mathrm{2}}}  \vdash  \Lnt{t_{{\mathrm{1}}}}  :  \Lnt{A}   \qquad   \Gamma_{{\mathrm{1}}}  \Lsym{,}  \Lmv{x}  \Lsym{:}  \Lnt{A}  \Lsym{,}  \Gamma_{{\mathrm{3}}}  \vdash  \Lnt{t_{{\mathrm{2}}}}  :  \Lnt{B}  }%
}{
 \Gamma_{{\mathrm{1}}}  \Lsym{,}  \Gamma_{{\mathrm{2}}}  \Lsym{,}  \Gamma_{{\mathrm{3}}}  \vdash  \Lsym{[}  \Lnt{t_{{\mathrm{1}}}}  \Lsym{/}  \Lmv{x}  \Lsym{]}  \Lnt{t_{{\mathrm{2}}}}  :  \Lnt{B} }{%
{\Ldrulename{T\_cut}}{}%
}}


\newcommand{\LdruleTXXUl}[1]{\Ldrule[#1]{%
\Lpremise{ \Gamma_{{\mathrm{1}}}  \Lsym{,}  \Gamma_{{\mathrm{2}}}  \vdash  \Lnt{t}  :  \Lnt{A} }%
}{
 \Gamma_{{\mathrm{1}}}  \Lsym{,}  \Lmv{x}  \Lsym{:}   I   \Lsym{,}  \Gamma_{{\mathrm{2}}}  \vdash   \mathsf{let}\, \Lmv{x} \,\mathsf{be}\,  \mathsf{unit}  \,\mathsf{in}\, \Lnt{t}   :  \Lnt{A} }{%
{\Ldrulename{T\_Ul}}{}%
}}


\newcommand{\LdruleTXXTl}[1]{\Ldrule[#1]{%
\Lpremise{ \Gamma  \Lsym{,}  \Lmv{x}  \Lsym{:}  \Lnt{A}  \Lsym{,}  \Lmv{y}  \Lsym{:}  \Lnt{B}  \Lsym{,}  \Gamma'  \vdash  \Lnt{t}  :  \Lnt{C} }%
}{
 \Gamma  \Lsym{,}  \Lmv{z}  \Lsym{:}   \Lnt{A}  \otimes  \Lnt{B}   \Lsym{,}  \Gamma'  \vdash   \mathsf{let}\, \Lmv{z} \,\mathsf{be}\,  \Lmv{x}  \otimes  \Lmv{y}  \,\mathsf{in}\, \Lnt{t}   :  \Lnt{C} }{%
{\Ldrulename{T\_Tl}}{}%
}}


\newcommand{\LdruleTXXTr}[1]{\Ldrule[#1]{%
\Lpremise{  \Gamma_{{\mathrm{1}}}  \vdash  \Lnt{t_{{\mathrm{1}}}}  :  \Lnt{A}   \qquad   \Gamma_{{\mathrm{2}}}  \vdash  \Lnt{t_{{\mathrm{2}}}}  :  \Lnt{B}  }%
}{
 \Gamma_{{\mathrm{1}}}  \Lsym{,}  \Gamma_{{\mathrm{2}}}  \vdash    \Lnt{t_{{\mathrm{1}}}}  \otimes  \Lnt{t_{{\mathrm{2}}}}    :    \Lnt{A}  \otimes  \Lnt{B}   }{%
{\Ldrulename{T\_Tr}}{}%
}}


\newcommand{\LdruleTXXIRl}[1]{\Ldrule[#1]{%
\Lpremise{  \Gamma_{{\mathrm{2}}}  \vdash  \Lnt{t_{{\mathrm{1}}}}  :  \Lnt{A}   \qquad   \Gamma_{{\mathrm{1}}}  \Lsym{,}  \Lmv{x}  \Lsym{:}  \Lnt{B}  \Lsym{,}  \Gamma_{{\mathrm{3}}}  \vdash  \Lnt{t_{{\mathrm{2}}}}  :  \Lnt{C}  }%
}{
 \Gamma_{{\mathrm{1}}}  \Lsym{,}  \Lmv{z}  \Lsym{:}   \Lnt{A}  \rightharpoonup  \Lnt{B}   \Lsym{,}  \Gamma_{{\mathrm{2}}}  \Lsym{,}  \Gamma_{{\mathrm{3}}}  \vdash  \Lsym{[}   \mathsf{app}_r\, \Lmv{z} \, \Lnt{t_{{\mathrm{1}}}}   \Lsym{/}  \Lmv{x}  \Lsym{]}  \Lnt{t_{{\mathrm{2}}}}  :  \Lnt{C} }{%
{\Ldrulename{T\_IRl}}{}%
}}


\newcommand{\LdruleTXXILl}[1]{\Ldrule[#1]{%
\Lpremise{  \Gamma_{{\mathrm{2}}}  \vdash  \Lnt{t_{{\mathrm{1}}}}  :  \Lnt{A}   \qquad   \Gamma_{{\mathrm{1}}}  \Lsym{,}  \Lmv{x}  \Lsym{:}  \Lnt{B}  \Lsym{,}  \Gamma_{{\mathrm{3}}}  \vdash  \Lnt{t_{{\mathrm{2}}}}  :  \Lnt{C}  }%
}{
 \Gamma_{{\mathrm{1}}}  \Lsym{,}  \Gamma_{{\mathrm{2}}}  \Lsym{,}  \Lmv{z}  \Lsym{:}   \Lnt{B}  \leftharpoonup  \Lnt{A}   \Lsym{,}  \Gamma_{{\mathrm{3}}}  \vdash  \Lsym{[}   \mathsf{app}_l\, \Lmv{z} \, \Lnt{t_{{\mathrm{1}}}}   \Lsym{/}  \Lmv{x}  \Lsym{]}  \Lnt{t_{{\mathrm{2}}}}  :  \Lnt{C} }{%
{\Ldrulename{T\_ILl}}{}%
}}


\newcommand{\LdruleTXXIRr}[1]{\Ldrule[#1]{%
\Lpremise{ \Gamma  \Lsym{,}  \Lmv{x}  \Lsym{:}  \Lnt{A}  \vdash  \Lnt{t}  :  \Lnt{B} }%
}{
 \Gamma  \vdash   \lambda_r  \Lmv{x} : \Lnt{A} . \Lnt{t}   :   \Lnt{A}  \rightharpoonup  \Lnt{B}  }{%
{\Ldrulename{T\_IRr}}{}%
}}


\newcommand{\LdruleTXXILr}[1]{\Ldrule[#1]{%
\Lpremise{ \Lmv{x}  \Lsym{:}  \Lnt{A}  \Lsym{,}  \Gamma  \vdash  \Lnt{t}  :  \Lnt{B} }%
}{
 \Gamma  \vdash   \lambda_l  \Lmv{x} : \Lnt{A} . \Lnt{t}   :   \Lnt{B}  \leftharpoonup  \Lnt{A}  }{%
{\Ldrulename{T\_ILr}}{}%
}}


\newcommand{\LdruleTXXC}[1]{\Ldrule[#1]{%
\Lpremise{ \Gamma_{{\mathrm{1}}}  \Lsym{,}  \Lmv{x}  \Lsym{:}   !  \Lnt{A}   \Lsym{,}  \Gamma_{{\mathrm{2}}}  \Lsym{,}  \Lmv{y}  \Lsym{:}   !  \Lnt{A}   \Lsym{,}  \Gamma_{{\mathrm{3}}}  \vdash  \Lnt{t}  :  \Lnt{B} }%
}{
 \Gamma_{{\mathrm{1}}}  \Lsym{,}  \Lmv{z}  \Lsym{:}   !  \Lnt{A}   \Lsym{,}  \Gamma_{{\mathrm{2}}}  \Lsym{,}  \Gamma_{{\mathrm{3}}}  \vdash   \mathsf{copy}\, \Lmv{x} \,\mathsf{as}\, \Lmv{y} , \Lmv{z} \,\mathsf{in}\, \Lnt{t}   :  \Lnt{B} }{%
{\Ldrulename{T\_C}}{}%
}}


\newcommand{\LdruleTXXW}[1]{\Ldrule[#1]{%
\Lpremise{ \Gamma_{{\mathrm{1}}}  \Lsym{,}  \Gamma_{{\mathrm{2}}}  \vdash  \Lnt{t}  :  \Lnt{B} }%
}{
 \Gamma_{{\mathrm{1}}}  \Lsym{,}  \Lmv{x}  \Lsym{:}   !  \Lnt{A}   \Lsym{,}  \Gamma_{{\mathrm{2}}}  \vdash   \mathsf{discard}\, \Lmv{x} \,\mathsf{in}\, \Lnt{t}   :  \Lnt{B} }{%
{\Ldrulename{T\_W}}{}%
}}


\newcommand{\LdruleTXXBr}[1]{\Ldrule[#1]{%
\Lpremise{  \vec{ \Lmv{x} } : ! \Gamma   \vdash  \Lnt{t}  :  \Lnt{B} }%
}{
  \vec{ \Lmv{y} } : ! \Gamma   \vdash   \mathsf{promote}_!\,\vec{ \Lmv{y} }\,\mathsf{for}\,\vec{ \Lmv{x} }\,\mathsf{in}\, \Lnt{t}   :   !  \Lnt{B}  }{%
{\Ldrulename{T\_Br}}{}%
}}


\newcommand{\LdruleTXXBl}[1]{\Ldrule[#1]{%
\Lpremise{ \Gamma_{{\mathrm{1}}}  \Lsym{,}  \Lmv{x}  \Lsym{:}  \Lnt{A}  \Lsym{,}  \Gamma_{{\mathrm{2}}}  \vdash  \Lnt{t}  :  \Lnt{B} }%
}{
 \Gamma_{{\mathrm{1}}}  \Lsym{,}  \Lmv{y}  \Lsym{:}   !  \Lnt{A}   \Lsym{,}  \Gamma_{{\mathrm{2}}}  \vdash  \Lsym{[}   \mathsf{derelict}_!\, \Lmv{y}   \Lsym{/}  \Lmv{x}  \Lsym{]}  \Lnt{t}  :  \Lnt{B} }{%
{\Ldrulename{T\_Bl}}{}%
}}


\newcommand{\LdruleTXXEr}[1]{\Ldrule[#1]{%
\Lpremise{  \vec{ \Lmv{x} } : \kappa  \Gamma   \vdash  \Lnt{t}  :  \Lnt{B} }%
}{
  \vec{ \Lmv{y} } : \kappa  \Gamma   \vdash   \mathsf{promote}_\kappa\,\vec{ \Lmv{y} }\,\mathsf{for}\,\vec{ \Lmv{x} }\,\mathsf{in}\, \Lnt{t}   :   \kappa  \Lnt{B}  }{%
{\Ldrulename{T\_Er}}{}%
}}


\newcommand{\LdruleTXXEl}[1]{\Ldrule[#1]{%
\Lpremise{ \Gamma_{{\mathrm{1}}}  \Lsym{,}  \Lmv{x}  \Lsym{:}  \Lnt{A}  \Lsym{,}  \Gamma_{{\mathrm{2}}}  \vdash  \Lnt{t}  :  \Lnt{B} }%
}{
 \Gamma_{{\mathrm{1}}}  \Lsym{,}  \Lmv{y}  \Lsym{:}   \kappa  \Lnt{A}   \Lsym{,}  \Gamma_{{\mathrm{2}}}  \vdash  \Lsym{[}   \mathsf{derelict}_\kappa\, \Lmv{y}   \Lsym{/}  \Lmv{x}  \Lsym{]}  \Lnt{t}  :  \Lnt{B} }{%
{\Ldrulename{T\_El}}{}%
}}


\newcommand{\LdruleTXXEOne}[1]{\Ldrule[#1]{%
\Lpremise{ \Gamma_{{\mathrm{1}}}  \Lsym{,}  \Lmv{x}  \Lsym{:}   \kappa  \Lnt{A}   \Lsym{,}  \Lmv{y}  \Lsym{:}  \Lnt{B}  \Lsym{,}  \Gamma_{{\mathrm{2}}}  \vdash  \Lnt{t}  :  \Lnt{C} }%
}{
 \Gamma_{{\mathrm{1}}}  \Lsym{,}  \Lmv{y}  \Lsym{:}  \Lnt{B}  \Lsym{,}  \Lmv{x}  \Lsym{:}   \kappa  \Lnt{A}   \Lsym{,}  \Gamma_{{\mathrm{2}}}  \vdash   \mathsf{exchange_l}\, \Lmv{x} \,\mathsf{with}\, \Lmv{y} \,\mathsf{in}\, \Lnt{t}   :  \Lnt{C} }{%
{\Ldrulename{T\_E1}}{}%
}}


\newcommand{\LdruleTXXETwo}[1]{\Ldrule[#1]{%
\Lpremise{ \Gamma_{{\mathrm{1}}}  \Lsym{,}  \Lmv{x}  \Lsym{:}  \Lnt{A}  \Lsym{,}  \Lmv{y}  \Lsym{:}   \kappa  \Lnt{B}   \Lsym{,}  \Gamma_{{\mathrm{2}}}  \vdash  \Lnt{t}  :  \Lnt{C} }%
}{
 \Gamma_{{\mathrm{1}}}  \Lsym{,}  \Lmv{y}  \Lsym{:}   \kappa  \Lnt{B}   \Lsym{,}  \Lmv{x}  \Lsym{:}  \Lnt{A}  \Lsym{,}  \Gamma_{{\mathrm{2}}}  \vdash   \mathsf{exchange_r}\, \Lmv{x} \,\mathsf{with}\, \Lmv{y} \,\mathsf{in}\, \Lnt{t}   :  \Lnt{C} }{%
{\Ldrulename{T\_E2}}{}%
}}

\newcommand{\LdefnJType}[1]{\begin{Ldefnblock}[#1]{$ \Gamma  \vdash  \Lnt{t}  :  \Lnt{A} $}{}
\Lusedrule{\LdruleTXXvar{}}
\Lusedrule{\LdruleTXXUr{}}
\Lusedrule{\LdruleTXXcut{}}
\Lusedrule{\LdruleTXXUl{}}
\Lusedrule{\LdruleTXXTl{}}
\Lusedrule{\LdruleTXXTr{}}
\Lusedrule{\LdruleTXXIRl{}}
\Lusedrule{\LdruleTXXILl{}}
\Lusedrule{\LdruleTXXIRr{}}
\Lusedrule{\LdruleTXXILr{}}
\Lusedrule{\LdruleTXXC{}}
\Lusedrule{\LdruleTXXW{}}
\Lusedrule{\LdruleTXXBr{}}
\Lusedrule{\LdruleTXXBl{}}
\Lusedrule{\LdruleTXXEr{}}
\Lusedrule{\LdruleTXXEl{}}
\Lusedrule{\LdruleTXXEOne{}}
\Lusedrule{\LdruleTXXETwo{}}
\end{Ldefnblock}}

%% defn ReW
\newcommand{\LdruleRXXAlpha}[1]{\Ldrule[#1]{%
\Lpremise{ \Lmv{y}  \not\in \mathsf{FV}( \Lnt{t} ) }%
}{
 \Lsym{[}  \Lmv{y}  \Lsym{/}  \Lmv{x}  \Lsym{]}  \Lnt{t}  \rightsquigarrow  \Lnt{t} }{%
{\Ldrulename{R\_Alpha}}{}%
}}


\newcommand{\LdruleRXXEtal}[1]{\Ldrule[#1]{%
}{
 \Lnt{t}  \rightsquigarrow   \lambda_l  \Lmv{x} : \Lnt{A} .  \mathsf{app}_l\, \Lnt{t} \, \Lmv{x}   }{%
{\Ldrulename{R\_Etal}}{}%
}}


\newcommand{\LdruleRXXEtar}[1]{\Ldrule[#1]{%
}{
 \Lnt{t}  \rightsquigarrow   \lambda_r  \Lmv{x} : \Lnt{A} .  \mathsf{app}_r\, \Lnt{t} \, \Lmv{x}   }{%
{\Ldrulename{R\_Etar}}{}%
}}


\newcommand{\LdruleRXXBetal}[1]{\Ldrule[#1]{%
}{
  \mathsf{app}_l\,  (   \lambda_l  \Lmv{x} : \Lnt{A} . \Lnt{t_{{\mathrm{2}}}}   )  \, \Lnt{t_{{\mathrm{1}}}}   \rightsquigarrow  \Lsym{[}  \Lnt{t_{{\mathrm{1}}}}  \Lsym{/}  \Lmv{x}  \Lsym{]}  \Lnt{t_{{\mathrm{2}}}} }{%
{\Ldrulename{R\_Betal}}{}%
}}


\newcommand{\LdruleRXXBetar}[1]{\Ldrule[#1]{%
}{
  \mathsf{app}_r\,  (   \lambda_r  \Lmv{x} : \Lnt{A} . \Lnt{t_{{\mathrm{2}}}}   )  \, \Lnt{t_{{\mathrm{1}}}}   \rightsquigarrow  \Lsym{[}  \Lnt{t_{{\mathrm{1}}}}  \Lsym{/}  \Lmv{x}  \Lsym{]}  \Lnt{t_{{\mathrm{2}}}} }{%
{\Ldrulename{R\_Betar}}{}%
}}


\newcommand{\LdruleRXXBetaU}[1]{\Ldrule[#1]{%
}{
  \mathsf{let}\, \Lnt{t_{{\mathrm{1}}}} \,\mathsf{be}\,  \mathsf{unit}  \,\mathsf{in}\, \Lsym{[}   \mathsf{unit}   \Lsym{/}  \Lmv{z}  \Lsym{]}  \Lnt{t_{{\mathrm{2}}}}   \rightsquigarrow  \Lsym{[}  \Lnt{t_{{\mathrm{1}}}}  \Lsym{/}  \Lmv{z}  \Lsym{]}  \Lnt{t_{{\mathrm{2}}}} }{%
{\Ldrulename{R\_BetaU}}{}%
}}


\newcommand{\LdruleRXXBetaTOne}[1]{\Ldrule[#1]{%
}{
  \mathsf{let}\,  \Lnt{t_{{\mathrm{1}}}}  \otimes  \Lnt{t_{{\mathrm{2}}}}  \,\mathsf{be}\,  \Lmv{x}  \otimes  \Lmv{y}  \,\mathsf{in}\, \Lnt{t}   \rightsquigarrow  \Lsym{[}  \Lnt{t_{{\mathrm{1}}}}  \Lsym{/}  \Lmv{x}  \Lsym{]}  \Lsym{[}  \Lnt{t_{{\mathrm{2}}}}  \Lsym{/}  \Lmv{y}  \Lsym{]}  \Lnt{t} }{%
{\Ldrulename{R\_BetaT1}}{}%
}}


\newcommand{\LdruleRXXBetaTTwo}[1]{\Ldrule[#1]{%
}{
  \mathsf{let}\, \Lnt{t_{{\mathrm{1}}}} \,\mathsf{be}\,  \Lmv{x}  \otimes  \Lmv{y}  \,\mathsf{in}\, \Lsym{[}   \Lmv{x}  \otimes  \Lmv{y}   \Lsym{/}  \Lmv{z}  \Lsym{]}  \Lnt{t_{{\mathrm{2}}}}   \rightsquigarrow  \Lsym{[}  \Lnt{t_{{\mathrm{1}}}}  \Lsym{/}  \Lmv{x}  \Lsym{]}  \Lnt{t_{{\mathrm{2}}}} }{%
{\Ldrulename{R\_BetaT2}}{}%
}}


\newcommand{\LdruleRXXNatU}[1]{\Ldrule[#1]{%
}{
 \Lsym{[}   \mathsf{let}\, \Lnt{t_{{\mathrm{1}}}} \,\mathsf{be}\,  \mathsf{unit}  \,\mathsf{in}\, \Lnt{t_{{\mathrm{2}}}}   \Lsym{/}  \Lmv{z}  \Lsym{]}  \Lnt{t_{{\mathrm{3}}}}  \rightsquigarrow   \mathsf{let}\, \Lnt{t_{{\mathrm{1}}}} \,\mathsf{be}\,  \mathsf{unit}  \,\mathsf{in}\, \Lsym{[}  \Lnt{t_{{\mathrm{2}}}}  \Lsym{/}  \Lmv{z}  \Lsym{]}  \Lnt{t_{{\mathrm{3}}}}  }{%
{\Ldrulename{R\_NatU}}{}%
}}


\newcommand{\LdruleRXXNatT}[1]{\Ldrule[#1]{%
}{
 \Lsym{[}   \mathsf{let}\, \Lnt{t_{{\mathrm{1}}}} \,\mathsf{be}\,  \Lmv{x}  \otimes  \Lmv{y}  \,\mathsf{in}\, \Lnt{t_{{\mathrm{2}}}}   \Lsym{/}  \Lmv{z}  \Lsym{]}  \Lnt{t_{{\mathrm{3}}}}  \rightsquigarrow   \mathsf{let}\, \Lnt{t_{{\mathrm{1}}}} \,\mathsf{be}\,  \Lmv{x}  \otimes  \Lmv{y}  \,\mathsf{in}\, \Lsym{[}  \Lnt{t_{{\mathrm{2}}}}  \Lsym{/}  \Lmv{z}  \Lsym{]}  \Lnt{t_{{\mathrm{3}}}}  }{%
{\Ldrulename{R\_NatT}}{}%
}}


\newcommand{\LdruleRXXLetU}[1]{\Ldrule[#1]{%
}{
  \mathsf{let}\,  \mathsf{unit}  \,\mathsf{be}\,  \mathsf{unit}  \,\mathsf{in}\, \Lnt{t}   \rightsquigarrow  \Lnt{t} }{%
{\Ldrulename{R\_LetU}}{}%
}}


\newcommand{\LdruleRXXBetaDR}[1]{\Ldrule[#1]{%
}{
  \mathsf{derelict}_!\,  (   \mathsf{promote}_!\,\vec{ \Lmv{y} }\,\mathsf{for}\,\vec{ \Lmv{x} }\,\mathsf{in}\, \Lnt{t_{{\mathrm{1}}}}   )    \rightsquigarrow  \Lnt{t_{{\mathrm{1}}}} }{%
{\Ldrulename{R\_BetaDR}}{}%
}}


\newcommand{\LdruleRXXBetaDI}[1]{\Ldrule[#1]{%
}{
  \mathsf{discard}\,  (   \mathsf{promote}_!\,\vec{ \Lmv{y} }\,\mathsf{for}\,\vec{ \Lmv{x} }\,\mathsf{in}\, \Lnt{t_{{\mathrm{1}}}}   )  \,\mathsf{in}\, \Lnt{t_{{\mathrm{2}}}}   \rightsquigarrow   \mathsf{discard}\,\vec{ \Lmv{y} }\,\mathsf{in}\, \Lnt{t_{{\mathrm{2}}}}  }{%
{\Ldrulename{R\_BetaDI}}{}%
}}


\newcommand{\LdruleRXXBetaC}[1]{\Ldrule[#1]{%
}{
  \mathsf{copy}\,  (   \mathsf{promote}_!\,\vec{ \Lmv{y} }\,\mathsf{for}\,\vec{ \Lmv{x} }\,\mathsf{in}\, \Lnt{t_{{\mathrm{1}}}}   )  \,\mathsf{as}\, \Lmv{w} , \Lmv{z} \,\mathsf{in}\, \Lnt{t_{{\mathrm{2}}}}   \rightsquigarrow  \Lsym{[}   \mathsf{promote}_!\,\vec{ \Lmv{y} }\,\mathsf{for}\,\vec{ \Lmv{x} }\,\mathsf{in}\, \Lnt{t_{{\mathrm{1}}}}   \Lsym{/}  \Lmv{w}  \Lsym{]}  \Lsym{[}   \mathsf{promote}_!\,\vec{ \Lmv{y} }\,\mathsf{for}\,\vec{ \Lmv{x} }\,\mathsf{in}\, \Lnt{t_{{\mathrm{1}}}}   \Lsym{/}  \Lmv{z}  \Lsym{]}  \Lnt{t_{{\mathrm{2}}}} }{%
{\Ldrulename{R\_BetaC}}{}%
}}


\newcommand{\LdruleRXXLet}[1]{\Ldrule[#1]{%
\Lpremise{ \Lnt{t_{{\mathrm{1}}}}  \rightsquigarrow  \Lnt{t_{{\mathrm{2}}}} }%
}{
  \mathsf{let}\, \Lmv{x} \,\mathsf{be}\, \Lnt{p} \,\mathsf{in}\, \Lnt{t_{{\mathrm{1}}}}   \rightsquigarrow   \mathsf{let}\, \Lmv{x} \,\mathsf{be}\, \Lnt{p} \,\mathsf{in}\, \Lnt{t_{{\mathrm{2}}}}  }{%
{\Ldrulename{R\_Let}}{}%
}}


\newcommand{\LdruleRXXTOne}[1]{\Ldrule[#1]{%
\Lpremise{ \Lnt{t_{{\mathrm{1}}}}  \rightsquigarrow  \Lnt{t'_{{\mathrm{1}}}} }%
}{
  \Lnt{t_{{\mathrm{1}}}}  \otimes  \Lnt{t_{{\mathrm{2}}}}   \rightsquigarrow   \Lnt{t'_{{\mathrm{1}}}}  \otimes  \Lnt{t_{{\mathrm{2}}}}  }{%
{\Ldrulename{R\_T1}}{}%
}}


\newcommand{\LdruleRXXTTwo}[1]{\Ldrule[#1]{%
\Lpremise{ \Lnt{t_{{\mathrm{2}}}}  \rightsquigarrow  \Lnt{t'_{{\mathrm{2}}}} }%
}{
  \Lnt{t_{{\mathrm{1}}}}  \otimes  \Lnt{t_{{\mathrm{2}}}}   \rightsquigarrow   \Lnt{t_{{\mathrm{1}}}}  \otimes  \Lnt{t'_{{\mathrm{2}}}}  }{%
{\Ldrulename{R\_T2}}{}%
}}


\newcommand{\LdruleRXXAlOne}[1]{\Ldrule[#1]{%
\Lpremise{ \Lnt{t_{{\mathrm{1}}}}  \rightsquigarrow  \Lnt{t'_{{\mathrm{1}}}} }%
}{
  \mathsf{app}_l\, \Lnt{t_{{\mathrm{1}}}} \, \Lnt{t_{{\mathrm{2}}}}   \rightsquigarrow   \mathsf{app}_l\, \Lnt{t'_{{\mathrm{1}}}} \, \Lnt{t_{{\mathrm{2}}}}  }{%
{\Ldrulename{R\_Al1}}{}%
}}


\newcommand{\LdruleRXXAlTwo}[1]{\Ldrule[#1]{%
\Lpremise{ \Lnt{t_{{\mathrm{2}}}}  \rightsquigarrow  \Lnt{t'_{{\mathrm{2}}}} }%
}{
  \mathsf{app}_l\, \Lnt{t_{{\mathrm{1}}}} \, \Lnt{t_{{\mathrm{2}}}}   \rightsquigarrow   \mathsf{app}_l\, \Lnt{t_{{\mathrm{1}}}} \, \Lnt{t'_{{\mathrm{2}}}}  }{%
{\Ldrulename{R\_Al2}}{}%
}}


\newcommand{\LdruleRXXArOne}[1]{\Ldrule[#1]{%
\Lpremise{ \Lnt{t_{{\mathrm{1}}}}  \rightsquigarrow  \Lnt{t'_{{\mathrm{1}}}} }%
}{
  \mathsf{app}_r\, \Lnt{t_{{\mathrm{1}}}} \, \Lnt{t_{{\mathrm{2}}}}   \rightsquigarrow   \mathsf{app}_r\, \Lnt{t'_{{\mathrm{1}}}} \, \Lnt{t_{{\mathrm{2}}}}  }{%
{\Ldrulename{R\_Ar1}}{}%
}}


\newcommand{\LdruleRXXArTwo}[1]{\Ldrule[#1]{%
\Lpremise{ \Lnt{t_{{\mathrm{2}}}}  \rightsquigarrow  \Lnt{t'_{{\mathrm{2}}}} }%
}{
  \mathsf{app}_r\, \Lnt{t_{{\mathrm{1}}}} \, \Lnt{t_{{\mathrm{2}}}}   \rightsquigarrow   \mathsf{app}_r\, \Lnt{t_{{\mathrm{1}}}} \, \Lnt{t'_{{\mathrm{2}}}}  }{%
{\Ldrulename{R\_Ar2}}{}%
}}


\newcommand{\LdruleRXXLl}[1]{\Ldrule[#1]{%
\Lpremise{ \Lnt{t}  \rightsquigarrow  \Lnt{t'} }%
}{
  \lambda_l  \Lmv{x} : \Lnt{A} . \Lnt{t}   \rightsquigarrow   \lambda_l  \Lmv{x} : \Lnt{A} . \Lnt{t'}  }{%
{\Ldrulename{R\_Ll}}{}%
}}


\newcommand{\LdruleRXXLr}[1]{\Ldrule[#1]{%
\Lpremise{ \Lnt{t}  \rightsquigarrow  \Lnt{t'} }%
}{
  \lambda_r  \Lmv{x} : \Lnt{A} . \Lnt{t}   \rightsquigarrow   \lambda_r  \Lmv{x} : \Lnt{A} . \Lnt{t'}  }{%
{\Ldrulename{R\_Lr}}{}%
}}

\newcommand{\LdefnReW}[1]{\begin{Ldefnblock}[#1]{$ \Lnt{t_{{\mathrm{1}}}}  \rightsquigarrow  \Lnt{t_{{\mathrm{2}}}} $}{}
\Lusedrule{\LdruleRXXAlpha{}}
\Lusedrule{\LdruleRXXEtal{}}
\Lusedrule{\LdruleRXXEtar{}}
\Lusedrule{\LdruleRXXBetal{}}
\Lusedrule{\LdruleRXXBetar{}}
\Lusedrule{\LdruleRXXBetaU{}}
\Lusedrule{\LdruleRXXBetaTOne{}}
\Lusedrule{\LdruleRXXBetaTTwo{}}
\Lusedrule{\LdruleRXXNatU{}}
\Lusedrule{\LdruleRXXNatT{}}
\Lusedrule{\LdruleRXXLetU{}}
\Lusedrule{\LdruleRXXBetaDR{}}
\Lusedrule{\LdruleRXXBetaDI{}}
\Lusedrule{\LdruleRXXBetaC{}}
\Lusedrule{\LdruleRXXLet{}}
\Lusedrule{\LdruleRXXTOne{}}
\Lusedrule{\LdruleRXXTTwo{}}
\Lusedrule{\LdruleRXXAlOne{}}
\Lusedrule{\LdruleRXXAlTwo{}}
\Lusedrule{\LdruleRXXArOne{}}
\Lusedrule{\LdruleRXXArTwo{}}
\Lusedrule{\LdruleRXXLl{}}
\Lusedrule{\LdruleRXXLr{}}
\end{Ldefnblock}}


\newcommand{\LdefnsJtyp}{
\LdefnLog{}\LdefnJType{}\LdefnReW{}}

\newcommand{\Ldefnss}{
\LdefnsJtyp
}

\newcommand{\Lall}{\Lmetavars\\[0pt]
\Lgrammar\\[5.0mm]
\Ldefnss}



\begin{document}

\maketitle

\section*{Introduction}
 Lambek  introduced his homonymous calculus (originally called the  `Syntactic Calculus') for  proposed applications in Linguistics.  
 However the calculus got much of its cult following  and reputation by being a convenient, well-behaved prototype of a Gentzen sequent calculus without any structural rules. 

This note recalls a Dialectica model of the Lambek Calculus presented
by the first author in the Amsterdam Colloquium in 1991 \cite{depaiva1991}. Here, like then, we approach
the Lambek Calculus from the perspective of Linear Logic, so we are interested in the basic  sequent calculus with no structural rules, except associativity of tensors. In that
earlier work we took for granted the syntax of the calculus and only
discussed the exciting possibilities of  categorical models of linear-logic-like systems.  Many years later we find that the work on models is still interesting and novel  and
that it might inform some of the most recent work on the relationship
between categorial grammars and notions of distributional
semantics \cite{coecke2013}.

Moreover, the type theoretical notions that were left undiscussed are now more amenable to verification, using some new automated tools. Since the Amsterdam Colloquium proceedings were never
published,
%and not even the author had a copy of the paper. 
 we have
decided to revisit some of the mathematical work,  using the new tools that have been developed for type theory and sequent proof systems in
the time that elapsed. Thus, we implement the calculus in Agda \cite{bove2009} and we
use \texttt{Ott} \cite{Sewell:2010} to check that we do not have trivial
mistakes in our term systems. 

The goal is to show that  our  implementation can shed new light on some of the issues that remained
open.  Mostly we wanted to check the correctness of the semantic proposals put forward since Szabo's seminal book \cite{szabo1978} and, for future work, on the applicability and fit of the original systems to their intended uses.

\subsection*{Overview}
The Syntactic Calculus was first introduced by Joachim Lambek in 1958 \cite{Lambek1958}. Since then the  rechristened  Lambek Calculus has had as its main motivation providing an explanation of the mathematics of sentence structure, starting from the author's algebraic intuitions. The Lambek calculus is the core of logical Categorial Grammar.  The first use of
the term “categorial grammar” seems to be in the title of Bar-Hillel,
Gaifman and Shamir (1960), but categorial grammar began with
Ajdukiewicz (1935) quite a few years earlier. After a period of
ostracism, around 1980 the Lambek Calculus was taken up by logicians
interested in Computational Linguistics, especially the ones
interested in Categorial Grammars. %theories such as Categorial .
% The calculus was considered ``like a logical system" but much too
% weak to be taken seriously as a logic. To quote from van
% Benthem \begin{quotation}One fundamental system of this kind is the
% so-called `Lambek Calculus' whose type-change rules show a close
% analogy with the inference rules of constructive propositional
% logic.  \end{quotation}

The work on Categorial Grammar was given a serious impulse by the
advent of Girard's Linear Logic  at the end of the 1980s.  Girard \cite{Girard:1987}
showed that there is a full embedding, preserving proofs, of
Intuitionistic Logic into Linear Logic with a modality ``!". This
meant that Linear Logic, while paying attention to resources, could
always code up faithfully Classical Logic and hence one could, as
Girard put it, `have one's cake and eat it', paying attention to
resources, if desired, but always forgetting this accounting, if
preferred. This meant also that several new
systems of resource logics were conceived and developed and these
refined resource logics were applied to several areas of Computer
Science.

In Computational Linguistics, the Lambek calculus 
has seen a significant number of works written
about it,  apart from a number of monographs that deal with
logical and linguistic aspects of the generalized type-logical
approach.  
%For general background on the type-logical approach, there are the monographs of Moortgat, Morril, Carpenter and Steedman. 
For a shorter introduction, see Moortgat's entry on the Stanford Encyclopedia of Philosophy  on Type Logical Grammar \cite{MoortgatSEP}.
Type Logical Grammar situates the type-logical approach within the
framework of Montague's Universal Grammar and presents detailed
linguistic analyses for a substantive fragment of syntactic and
semantic phenomena in the grammar of English.  Type Logical Semantics offers a general introduction to natural language semantics studied from a type-logical perspective.
% Moortgat's Categorial Type Logics,
% in the Handbook of Logic ad
% Language


% However, fashion turned against these systems with the rise of
% probabilistic and machine learning systems.

This meant that a series of systems, implemented or not, were devised
that used the Lambek Calculus or its variants as their
basis. These systems can be as expressive as Intuitionistic Logic and the claim is that they are more precise i.e. they make finer
distinctions. Some of the landscape of calculi can be depicted as follows:
\[
\begin{array}{lll}
  \bfig
  \Ctriangle/-``-/<400,400>[
    \textbf{LJ}_{\textbf{imp}}`
    \textbf{LP}_{\textbf{cont}}`
    \textbf{LP};``]

  \Dtriangle(400,0)/`-`-/<400,400>[
    \textbf{LJ}_{\textbf{imp}}`
    \textbf{LP}_{\textbf{weak}}`
    \textbf{LP};``]

  \morphism(400,0)<0,-300>[\textbf{LP}`\textbf{L};]
  \efig
  & \quad &
  \bfig
  \Ctriangle/-``-/<400,400>[
    \textbf{ILL}`
    \textbf{LA}`
    \textbf{L};``]

  \Dtriangle(400,0)/`-`-/<400,400>[
    \textbf{ILL}`
    \textbf{LP}`
    \textbf{L};``]
  \efig
\end{array}
\]

From the beginning it was clear that the Lambek
Calculus is the multiplicative fragment of non-commutative
Intuitionistic Linear Logic. In the diagrams  $\bf L$ stands for the Lambek calculus, as expounded in \cite{Lambek1958} but with the  unit $I$  added for  the tensor connective
 (there was a certain amount of dispute on that, as the original
system did not introduce the constant corresponding to the nullary
case of the tensor product, here written as $I$). The system $\bf{LP}$ is the Lambek calculus with permutation, sometimes called the van Benthem calculus. We are calling $\bf{LA}$ the Lambek calculus with additives, the more usual algebraic connectives corresponding to meet and join. Hence adding both permutation and additives to the Lambek calculus we get to intuitonistic linear logic. On the other diagram to the Lambek calculus with permutation we add either weakening ($\bf{LP_{weak}}$) or contraction $\bf{LP_{weak}}$) or both to get the implicational fragment of Intuitionsitic Propositional Logic.


There several
interesting questions, considered for Linear Logic, that could also be asked of the Lambek Calculus or its extensions.  
One of them, posed by Morrill et al is
whether we can extend the Lambek calculus with a modality that does
for the structural rule of \textit{(exchange)} what the modality
\textit{of course} `!' does for the rules of \textit{(weakening)} and
\textit{(contraction)}.  A preliminary proposal, which answers this
question affirmatively, is set forward in this paper. The answer was
provided in semantical terms in the first version of this work. Here
we provide also the more syntactic description of these
modalities. Building up from work of  Ciabattoni, Galatos and Terui in \cite{Ciabattoni2012} and others that describe how to transform systems of axioms into cut-free sequent rules, we aim to refine the
algebraization of proof theory.

%amount of work in Type Logical The Proof Theory of the systems
%considered should be investigated in future work. Another warning is
%that the perspective of this note is basically from Category Theory
%as a branch of Mathematics, so words like categories and functors are
%always meant in their mathematical, rather than linguistical or
%philosophical sense.


We first recall Linear Logic and provide the transformations to show
that the Lambek Calculus \textsf{L} really is the multiplicative
fragment of non-commutative Intuitionistic Linear Logic.  Then we
describe the usual String Semantics for the Lambek Calculus \textsf{L}
and generalize it, using a categorical perspective in the second
section.  The third section recalls our Dialectica model for the
Lambek Calculus.  Finally, in the fourth section we discuss modalities
and some untidiness of the Curry-Howard correspondence for the
fragments of Linear Logic in question.

\section{The Lambek Calculus}

The Lambek Calculus, formerly the Syntactic Calculus $\sf L$, due to
J.  Lambek \cite{Lambek1958}, was created to capture the logical
structure of sentences.  Lambek introduced what we think of as a
substructural logic with an operator denoting concatenation,
$[[A (x) B]]$, and two implications relating the order of phrases,
$[[A <- B]]$ and $[[A -> B]]$.  The first implication corresponds to a
phrase of type $[[A]]$ when followed by a phrase of type $[[B]]$, and
the second is a phrase of type $[[B]]$ when proceeded by a phrase of
type $[[A]]$.

The Lambek Calculus can be presented as a
non-commutative intuitionistic multiplicative linear logic.  The
syntax of formulas and contexts of the logic are as follows:

\[
\begin{array}{lll}
  \text{(formulas)} & [[A]],[[B]],[[C]] ::= [[I]] \mid [[A (x) B]] \mid [[A <- B]] \mid [[A -> B]]\\
  \text{(contexts)} & [[G]] ::= [[A]] \mid [[G1,G2]]\\
\end{array}
\]

We denote mapping the modalities over an arbitrary context by $[[!{ G
}]]$ and $[[E{ G }]]$.  The inference rules are defined in
Figure~\ref{fig:L}.

\begin{figure}
  \begin{mdframed}
    \begin{mathpar}
      \Ldruleax{} \and
      \LdruleUr{} \and      
    \Ldrulecut{} \and
    \LdruleUl{} \and
    \LdruleTl{} \and
    \LdruleTr{} \and
    \LdruleIRl{} \and
    \LdruleILl{} \and
    \LdruleIRr{} \and
    \LdruleILr{}     
  \end{mathpar}
  \end{mdframed}
    
  \caption{The Lambek Calculus: L}
  \label{fig:L}
\end{figure}

Because the operator $[[A (x) B]]$ denotes the type of concatenations
the types $[[A (x) B]]$ and $[[B (x) A]]$ are not equivalent, and
hence, \textsf{L} is non-commutative which explains why implication
must be broken up into two operators $[[A <- B]]$ and $[[A -> B]]$.
In the following subsections we give two extensions of L: one with the
well-known modality of-course of linear logic which adds weakening and
contraction, and a second with a new modality adding exchange.

\section{
%Adding Weakening and Contraction 
Extensions to the Lambek Calculus}
\label{subsec:the_lambek_calculus_with_the_weakening_and_contraction_modality}

The linear modality, $[[! A]]$, read ``of-course $[[A]]$'' was first
proposed by Girard \cite{Girard:1987} as a means of encoding
non-linear logic in both classical and intuitionistic forms in linear
logic.  For example, non-linear implication $[[A -> B]]$ is usually
encoded into linear logic by $[[!A]] \multimap [[B]]$. Since we have
based L on non-commutative intuitionistic linear logic it is
straightforward to add the of-course modality to L.  The rules for the
of-course modality are defined by the following rules:
\[
\begin{array}{lll}
  \LdruleC{} & \LdruleW{}\\
  \\
  \LdruleBr{} & \LdruleBl{}
\end{array}
\]
The rules $\Ldrulename{C}$ and $\Ldrulename{W}$ add contraction and
weakening to L in a controlled way.  Then the other two rules allow
for linear formulas to be injected into the modality; and essentially
correspond to the rules for necessitation found in S4\cite{bierman2000}.  Thus,
under the of-course modality the logic becomes non-linear. We will see
in Section~\ref{sec:dialectica_lambek_spaces} that these rules define
a comonad.  We call the extension of L with the of-course modality
$\LB$.
% subsection the_lambek_calculus_with_the_weakening_and_contraction_modality (end)

\subsection{Adding Exchange to the Lambek Calculus}
\label{subsec:the_lambek_calculus_with_the_exchange_modality}
As we remarked above, one leading question of the Lambek Calculus is
can exchange be added in a similar way to weakening and contraction?
That is, can we add a new modality that adds the exchange rule to L in
a controlled way?  The answer to this question is positive, and the
rules for this new modality are as follows:
\[
\begin{array}{lll}
  \LdruleEr{} & \LdruleEl{}\\
  \\
  \LdruleEOne{} & \LdruleETwo{} 
\end{array}
\]
The first two rules are similar to of-course, but the last two add
exchange to formulas under the $\kappa$-modality.  We call L with the
exchange modality $\LE$.  Thus, unlike intuitionistic linear logic
where any two formulas can be exchanged $\LE$ restricts exchange to
only formulas under the exchange modality.  Just like of-course
the exchange modality is modeled categorically as a comonad; see
Section~\ref{sec:dialectica_lambek_spaces}.
% subsection the_lambek_calculus_with_the_exchange_modality (end)

\section{Algebraic Semantics}

In Lambek's original paper \cite{Lambek1958} introducing his calculus
L, albeit without modalities, he introduced an algebraic semantics
that is now called the String Semantics for L.  The semantics begins
with a non-empty set of expressions denoted $\V$, and then by modeling
formulas of L by subsets of $\V$ it proceeds by defining operations on
these subsets, which correspond to the logical connectives of L.  Each
operation is defined as follows (using our notation for the logical
connectives):
\[
\begin{array}{c}
  [[A (x) B]] = \{xy \in \V \mid x \in A \text{ and } y \in B\}\\ \relax
  [[A <- B]] = \{x \in \V \mid \text{for all } y \in B, xy \in A\}\\ \relax
  [[A -> B]] = \{y \in \V \mid \text{for all } x \in A, xy \in B\}\\ \relax
\end{array}
\]
Let $\N = \pow{\V}$ be the powerset of $\V$.  Then we can view each of
the above definitions as binary operations with type $\N \times \N \mto
\N$. In fact, $\N$ has a natural order induced by set containment, and
concatenation, $[[A (x) B]]$, gives $\N$ a non-commutative monoidal
structure, where the unit $[[I]] = \{\epsilon\}$ is the set containing
the empty sequence:
\[
\begin{array}{lll}
  \text{(associativity)} & [[A (x) (B (x) C)]] = [[(A (x) B) (x) C]]\\
  \text{(unit)} & [[A (x) I]] = [[A]] = [[I (x) A]]\\
  \text{(non-commutativity)} & [[A (x) B]] \neq [[B (x) A]]\text{, in general}
\end{array}
\]

There happens to be a more general structure underlying the previous
semantics.  We now make this structure explicit using the tools
developed by Hyland and de Paiva \cite{Hyland:1991}. The ordering on
$\N$ induces a poset $(\N, \subseteq)$, but even more so, $\N$ is also
a monoid $(\N, \otimes, I)$, but that is not all, these two structures
are compatible, that is, given $A \subseteq B$ the following hold:
\[
  \begin{array}{cll}
    [[A (x) C]] \subseteq [[B (x) C]]\text{, for all } C \in \N\\ \relax
    [[C (x) A]] \subseteq [[C (x) B]]\text{, for all } C \in \N\\
  \end{array}
\]
Abstracting this structure out yields what can be called an ordered
non-commutative monoid. Other names can also be used, like FL-algebras, for Full Lambek algebras. The literature is extense, \cite{Pentus1995} discusses some of it.

\begin{definition}
  \label{def:ordered-non-comm-monoid}
  An \textbf{ordered non-commutative monoid}, $(M,\leq,\circ,e)$, is a
  poset $(M,\leq)$ with a given compatible monoidal structure $(M,
  \circ, e)$.  That is, a set $M$ equipped with a binary relation,
  $\leq : M \times M \mto \mathsf{2}$, satisfying:
  \begin{center}
    \begin{tabular}{cll}
    \text{(reflexivity)} & $a \leq a$ for all $a \in M$\\
    \text{(transitivity)} &  $a \leq b$ and $b \leq c$, implies that $a
    \leq c$ for all $a, b , c \in M$\\
    \text{(antisymmetry)} &  $a \leq b$ and $b \leq a$, implies that $a = b$
    \end{tabular}
  \end{center}
  together with a monoidal multiplication $\circ : M \times M \mto M$
  and a distinguished object $e \in M$ satisfying the following:
  \begin{center}
    \begin{tabular}{cll}
    \text{(associativity)} & $a \circ (b \circ c) = (a \circ b) \circ c$\\
    \text{(identity)} & $e \circ a = a = a \circ e$
    \end{tabular}
  \end{center}
  The structures are compatible in the sense that, if $a \leq b$, then
  the following hold:
  \begin{center}
    \begin{tabular}{lll}
      $a \circ c \leq b \circ c$ for any $c \in M$\\
      $c \circ a \leq c \circ b$ for any $c \in M$\\
    \end{tabular}
  \end{center}
\end{definition}
It is easy to see that the previous definition accounts for all of the
structure we have described so far, and thus, we may conclude that
$(\N, \subseteq, \otimes, I)$ is an ordered non-commutative monoid,
however, this definition is not able to model the implication
operations $[[A <- B]]$ and $[[A -> B]]$.  To do this we need to
understand how implication relates to the ordered non-commutative
monoid structure.  Notice that the following hold:
\[
\begin{array}{lll}
  [[A (x) (A -> B)]] \subseteq B\\ \relax
  [[(A <- B) (x) A]] \subseteq B\\ \relax
\end{array}
\]
Furthermore, there are no larger objects of $\N$ with these
properties.  Abstracting this results in the notion of a biclosed
poset.
\begin{definition}
  \label{def:biclosed-poset}
  Suppose $(M, \leq, \circ, e)$ is an ordered non-commutative monoid.
  If there exists a largest $x \in M$ such that $a \circ x \leq b$ for
  any $a, b \in M$, then we denote $x$ by $a \lto b$ and called it
  the \textbf{left-pseudocomplement} of $a$ w.r.t $b$.  Additionally,
  if there exists a largest $x \in M$ such that $x \circ a \leq b$ for
  any $a, b \in M$, then we denote $x$ by $a \rto b$ and called it
  the \textbf{right-pseudocomplement} of $a$ w.r.t $b$.

  A \textbf{biclosed poset}, $(M, \leq, \circ, e, \lto, \rto)$, is an  ordered non-commutative monoid, $(M, \leq, \circ, e)$, such that $a
  \lto b$ and $a \rto b$ exist for any $a,b \in M$.
\end{definition}

At this point we have everything we need to model the Lambek Calculus L without modalities.
\begin{lemma}
  \label{lemma:biclosed-poset-model-L}
  Any biclosed poset $(M, \leq, \circ, e, \lto, \rto)$ is a model for  the Lambek Calculus L without modalities.
\end{lemma}
\begin{proof}
  First suppose we have an assignment $(-)^0$ which assigns to each
  formula of L an element of $M$.  Then if $[[G |- A]]$ holds we
  show that $([[G]])^0 \leq ([[A]])^0$.  This proof can easily be
  completed by induction on the form $[[G |- A]]$.  
\end{proof}

\subsection{Two Example Algebraic Models}
\label{subsec:two_example_algebraic_models}

We show that the set of invertible matrices over the set of natural
numbers ($\mathbb{N}$) is a biclosed poset, and that $\mathbb{N}$ is
itself a biclosed poset.

\subsubsection{The Set of Invertible Matrices}
\label{subsec:the_set_of_invertible_matrices}

% subsubsection the_set_of_invertible_matrices (end)

\subsubsection{The Set of Natural Numbers}
\label{subsec:the_set_of_natural_numbers}

% subsubsection the_set_of_natural_numbers (end)
% subsection two_example_algebraic_models (end)

As hinted before these algebraic structrues have been studied before
and, despite the order that we used here, the structure underlying the
connection between the left and right implications and the tensor,
which has traditionally been called a residuated structure is the most
important one. It is worth noticing that surprising examples of these
structures crop up in the work of Lawvere \cite{lawvere1973}. We
recall two of these surprising examples, as they will be useful in the
sequel.

\section{Categorical Models}

Historically, the Lambek calculus arose from its intended categorical
model, biclosed symmetric monoidal categories, as discussed by Lambek
in \cite{lambek1988}. Clearly all this happened almost three decades
before Girard introduced Linear Logic, hence there were no modalities
or exponentials in this setting.


Categorically, one models the of-course modality as a functor endowed
with the structure of a comonad $(!, \delta_A, \varepsilon_A)$ with
some additional structure.  That is, there are maps
$\delta_A : [[!A]] \mto [[!!A]]$ and $\varepsilon_A : [[!A]] \mto [[A]]$
subject to a few coherence diagrams, and maps $c_A : [[!A]] \mto
[[h(!A) (x) !A]]$ and $w_A : [[!A]] \mto [[I]]$.  Using this structure we can
interpret the rules of the of-course modality.

Consider the rule $\Ldrulename{C}$ from Section~\ref{subsec:the_lambek_calculus_with_the_weakening_and_contraction_modality},
and suppose we have a map $[[G]]
\otimes [[(h(!A) (x) !A)]] \mto^{f} [[B]]$, then we can obtain a new
map $[[G]] \otimes [[!A]] \mto^{\id_{[[G]]} \otimes c_A} [[G]] \otimes
[[(h(!A) (x) !A)]] \mto^f [[B]]$.  The rule $\Ldrulename{W}$ is
similar, but we start with a map $[[G]] \mto^f [[B]]$ and then we can
define the map $[[G]] \otimes [[!A]] \mto^{\id_{[[G]]} \otimes w_A}
[[G]] \otimes [[I]] \mto^\cong [[G]] \mto^{f} [[B]]$.  Notice that the
previous map exploits the fact that $[[I]]$ is the unit for tensor.
Now consider the rule $\Ldrulename{Bl}$, and suppose we have a map
$[[!{G}]] \mto^f [[B]]$, then we may obtain a second map using the
fact that of-course is a functor $[[!{G}]] \mto^{\delta} [[!{!{G}}]]
\mto^{!f} [[!B]]$.  Finally, consider the rule $\Ldrulename{Br}$ and
suppose we have a map $[[G]] \otimes [[A]] \mto^f [[B]]$, then we can
construct the map $[[G]] \otimes [[!A]] \mto^{\id_{[[G]]} \otimes
  \varepsilon_A} [[G]] \otimes [[A]] \mto^f [[B]]$.

This analysis tells us a few things about interpreting logics into
categorical models.  Sequents, $[[G |- B]]$, are interpreted as
morphisms, $[[G]] \mto^f [[B]]$, where $[[G]]$ is $[[I]]$ if it is
empty, or it is the tensor product of the interpretations of its
formulas.  Then interpreting inference rules amounts to starting with
the morphisms corresponding to the premises, and then building a map
corresponding to the conclusion.  The cut-elimination procedure is
defined by a set of equations between derivations, and hence, in the
model corresponds to equations between morphisms.  The various
coherence diagrams relating the structure of the model enforce that
these equations hold.

We can similarly interpret the rules for the exchange modality.
That is, as a functor endowed with the structure of a second comonad,
but also with the maps $e_1 : [[A (x) E B]] \mto [[h(E B) (x) A]]$ and
$e_2 : [[h(E A) (x) B]] \mto [[B (x) E A]]$.  Then, each of the
inference rules for the exchange modality can easily be interpreted
into the model.

\subsection{Dialectica Lambek Spaces}
\label{sec:dialectica_lambek_spaces}

A sound and complete categorical model of each of the Lambek calculi
we have been exploring can be given in a modification to de Paiva's
dialectica categories \cite{depaiva1990}.  Originally, dialectica
categories arose from de Paiva's thesis work on the categorical model
of G\"odel's Dialectica interpretation.  In fact, dialectica
categories are one of the first sound categorical models of
intuitionistic linear logic with the linear modalities.  We show in
this section that they can be adapted to a sound a complete model,
called dialectica Lambek spaces, for the Lambek calculus with both the
exchange and of-course modalities.  Due to the complexities of working
with dialectica categories we have formally verified this section in
the proof assistant Agda\footnote{The complete formalization can be
  found online at
  \url{https://github.com/heades/dialectica-spaces/blob/Lambek/NCDialSets.agda}.}
\cite{bove2009}.

Dialectica categories arise as constructions over a given monoidal
category.  Suppose $\cat{C}$ is such a category.  Then in complete
generality the objects of the dialectica category over $\cat{C}$ are
triples $(U, X, \alpha)$ where $U$ and $X$ are objects of $\cat{C}$,
and $\alpha : A \mto U \otimes X$ is a subobject of the tensor product
in $\cat{C}$ of $U$ and $X$.  Thus, we can think of $\alpha$ as a
relation over $U \otimes X$.  If we specialize the category $\cat{C}$
to the category of sets and functions, $\Set$, then we obtain what is
called a dialectica space. Dialectica spaces are a useful model of
full intuitionistic linear logic \cite{Hyland:1993}.

Now morphisms between objects $(U, X, \alpha)$ and $(V, Y, \beta)$ are
pairs $(f, F)$ where $f : U \mto V$ and $F : Y \mto X$ are morphisms
of $\cat{C}$ such that the following pullback condition holds:
\[
(U \otimes F)^{-1}(\alpha) \leq (f \otimes Y)^{-1}(\beta)
\]
In dialectica spaces this condition becomes the following:
\[
\forall u \in U.\forall y \in Y. \alpha(u , F(y)) \leq \beta(f(u), y)
\]
The latter reveals that we can think of the condition on morphisms as
a weak adjoint condition.  Finally, through some nontrivial reasoning
on this structure we can show that this is indeed a category; for the
details see the formal development.  Dialectica categories are related
to the Chu construction \cite{?} and to $\text{GAME}_{\kappa_1}$
\cite{?}.

To some extent the underlying category $\cat{C}$ controls the kind of
structure we can expect in the dialectica category over $\cat{C}$.  de
Paiva later showed \cite{?} that by changing the relation used in
objects and the relation used in the weak adjoint condition also
controls the type of structure we obtain in the dialectica category,
but the structure of the underlying category and the structure of the
underlying relations must be compatible.  She shows that one can
abstract this notion of relation out as a parameter in the dialectica
construction, and we denote this by $\mathsf{Dial}_{L}(\cat{C})$ where
$L$ is the structure controlling the relations used in the
construction.  For example, $\mathsf{Dial}_{\mathsf{2}}(\Set)$ is the
category of dialectica spaces.  Thus, we can see dialectica categories
as really a framework of categorical models of various logics.
Depending on which category we start with and which structure we use
for the relations in the construction we will obtain different models
for different logics.

The underlying category we will choose here is the category $\Set$,
but the structure we will define our relations over will be biclosed
posets (Definition~\ref{def:biclosed-poset}).
\begin{definition}
  \label{def:dialectica-lambek-spaces}
  Suppose $(M, \leq, \circ, e, \lto, \rto)$ is a biclosed poset. Then
  we define the category of \textbf{dialectica Lambek spaces},
  $\mathsf{Dial}_M(\Set)$, as follows:
  \begin{itemize}
  \item[-] objects, or dialectica Lambek spaces, are triples $(U, X,
    \alpha)$ where $U$ and $X$ are sets, and $\alpha : U \times X \mto
    M$ is a generalized relation over $M$, and

  \item[-] maps that are pairs $(f, F) : (U , X, \alpha) \mto (V , Y ,
    \beta)$ where $f : U \mto V$, and $F : Y \mto X$ are functions
    such that the following weak adjointness condition holds:
    \[
    \forall u \in U.\forall y \in Y. \alpha(u , F(y)) \leq \beta(f(u), y)
    \]
  \end{itemize}
\end{definition}
Notice here that the biclosed poset is used as the target of the
relations in objects, but also as the relation in the weak adjoint
condition on morphisms.  This will allow the structure of the biclosed
poset to lift up into $\Dial{M}{\Set}$.

We claim that $\Dial{M}{\Set}$ is a model of the Lambek calculus with
modalities.  First, we show that it is a model of the Lambek calculus
without modalities.  Thus, we must show that $\Dial{M}{\Set}$ is
monoidal biclosed.

\begin{definition}
  \label{def:dial-monoidal-structure}
  Suppose $(U, X, \alpha)$ and $(V, Y, \beta)$ are two objects of
  $\Dial{M}{\Set}$. Then their tensor product is defined as follows:
  \[
  (U, X, \alpha) \otimes (V, Y, \beta) = (U \times V, (V \to X) \times (U \to Y), \alpha \otimes \beta)
  \]
  where $- \to -$ is the function space from $\Set$, and $(\alpha
  \otimes \beta)((u, v), (f, g)) = \alpha(u, f(v)) \circ \beta(g(u), v)$.
\end{definition}

\noindent
The identity of the tensor product just defined is $I = (\mathbbm{1},
\mathbbm{1}, e)$, where $\mathbbm{1}$ is the terminal object in
$\Set$, and $e$ is the unit of the biclosed poset.  It is
straightforward to show that the tensor product is functorial, one can
define the left and right unitors, and the associator for tensor; see
the formalization for the definitions.  In addition, all of the usual
monoidal diagrams hold \cite{depaiva1990}.  Take note of the fact that
this tensor product is indeed non-commutative, because the
non-commutative multiplication of the biclosed poset is used to define
the relation of the tensor product.

The tensor product has two right adjoints making $\Dial{M}{\Set}$
biclosed.
\begin{definition}
  \label{def:dial-is-biclosed}
  Suppose $(U, X, \alpha)$ and $(V, Y, \beta)$ are two objects of
  $\Dial{M}{\Set}$. Then two internal-homs can be defined as follows:
  \[
  \begin{array}{lll}
    (U, X, \alpha) \lto (V, Y, \beta) = ((U \to V) \times (Y \to X), U \times Y, \alpha \lto \beta)\\
    (V, Y, \beta) \rto (U, X, \alpha) = ((U \to V) \times (Y \to X), U \times Y, \alpha \rto \beta)\\
  \end{array}
  \]
\end{definition}
These two definitions are functorial, where the first is contravariant
in the first argument and covariant in the second, but the second
internal-hom is covariant in the first argument and contravariant in
the second.  The relations in the previous two definitions prevent
these two from collapsing into the same object, because of the use of
the left and right pseudo-compliments. It is straightforward to show
that the following bijections hold:
\[
\begin{array}{lll}
  \Hom{}{A \otimes B}{C} \cong \Hom{}{B}{A \lto C}\\
  \Hom{}{A \otimes B}{C} \cong \Hom{}{A}{C \rto B}\\
\end{array}
\]
Therefore, $\Dial{M}{\Set}$ is biclosed, and we obtain the following
result.
\begin{theorem}
  \label{thm:sound-lambek}
  $\Dial{M}{\Set}$ is a sound and complete model for the Lambek
  calculus $L$ without modalities.
\end{theorem}

\subsubsection{Modalities}
\label{subsec:modalities}
We now extend $\Dial{M}{\Set}$ with two modalities: the usual
modality, of-course, denoted $!A$, and the exchange modality denoted
$\kappa A$.  However, we must first extended biclosed posets to
include an exchange operation.

\begin{definition}
  \label{def:biclosed-exchange}
  A \textbf{biclosed poset with exchange} is a poset $(M, \leq, \circ,
  e, \lto, \rto)$ equipped with an unary operation $\kappa : M \to M$
  satisfying the following:
  \[
  \begin{array}{clr}
    \text{(Compatibility)} & a \leq b \text{ implies } \kappa a \leq \kappa b \text{ for all } a,b,c \in M\\
    \text{(Minimality)} & \kappa a \leq a \text{ for all } a \in M\\
    \text{(Duplication)} & \kappa a \leq \kappa\kappa a \text{ for all } a \in M\\
    \text{(Left Exchange)} & \kappa a \circ b \leq b \circ \kappa a \text{ for all } a, b \in M\\
    \text{(Right Exchange)} & a \circ \kappa b \leq \kappa b \circ a \text{ for all } a, b \in M\\
  \end{array}
  \]
\end{definition}
\noindent
Compatibility results in $\kappa : M \to M$ being a functor in the
biclosed poset, and the remainder of the axioms imply that $\kappa$ is
a comonad extending the biclosed poset with left and right exchange.

We can now define the two modalities in $\Dial{M}{\Set}$ where $M$ is
a biclosed poset with exchange; clearly we know $\Dial{M}{\Set}$ is
also a model of the Lambek calculus without modalities by
Theorem~\ref{thm:sound-lambek} because $M$ is a biclosed poset.
\begin{definition}
  \label{def:modalities-dial}
  Suppose $(U, X, \alpha)$ is an object of $\Dial{M}{\Set}$ where $M$
  is a biclosed poset with exchange. Then the \textbf{of-course} and
  \textbf{exchange} modalities can be defined as follows:
  \[
  \begin{array}{lll}
    ! (U, X, \alpha) = (U, U \to X^*, !\alpha)\\
    \kappa (U, X, \alpha) = (U, X, \kappa \alpha)\\
  \end{array}
  \]
  where $X^*$ is the free commutative monoid on $X$, $(!\alpha)(u, f)
  = \alpha(u, x_1) \circ \cdots \circ \alpha(u, x_i)$ for $f(u) =
  (x_1, \ldots, x_i)$, and $(\kappa \alpha)(u, x) = \kappa (\alpha(u,
  x))$.
\end{definition}
This definition highlights a fundamental difference between the two
modalities.  The definition of the exchange modality relies on an
extension of biclosed posets with essentially the exchange modality in
the category of posets.  However, the of-course modality is defined by
the structure already present in $\Dial{M}{\Set}$, specifically, the
structure of $\Set$.

Both of the modalities have the structure of a comonad.  That is,
there are monoidal natural transformations $\varepsilon_! : !A \mto
A$, $\varepsilon_\kappa : \kappa A \mto A$, $\delta_! : !A \mto !!A$,
and $\delta_\kappa : \kappa A \mto \kappa\kappa A$ which satisfy the
appropriate diagrams; see the formalization for the full
proofs. Furthermore, these comonads come equipped with arrows $e : A
\mto I$, $d : A \mto A \otimes A$, $\beta L : \kappa A \otimes B \mto B
\otimes \kappa A$, and $\beta R : A \otimes \kappa B \mto \kappa B
\otimes A$.  Thus, we arrive at the following result.

\begin{theorem}
  \label{thm:sound-dial-exchange-!}
  Suppose $M$ is a biclosed poset with exchange.  Then
  $\Dial{M}{\Set}$ is a sound and complete model for the Lambek
  calculi $L_!$, $L_\kappa$, and $L_{!\kappa}$.
\end{theorem}
% subsubsection modalities (end)
% end subsection dialectica_lambek_spaces (end)
% end section categorical_models (end)

\section{Type Theory for Lambek Systems}
\label{sec:typed_lambek_calculi}
In this section we introduce typed calculi for each of the logics
discussed so far.  Each type system is based on the term assignment
for Full Intuitionistic Linear Logic introduced by Hyland and de Paiva
\cite{Hyland:1993}.  We show that they are all strongly normalizing
and confluent, but we do not give full detailed proofs of each of
these properties, because they are straightforward consequences of the
proofs of strong normalization and confluence for intuitionistic
linear logic.  In fact, we will reference Bierman's thesis often
within this section.  The reader may wish to review Section~3.5 on
page 88 of \cite{Bierman:1994}.

\subsection{The Typed Lambek Calculus: $\lambda\text{L}$}
\label{subsec:the_typed_lambek_calculus_lambda-l}

The first system we cover is the Lambek calculus without
modalities. This system can be seen as the initial core of each of the
other systems we introduce below, and thus, we will simply extend the
results here to those other systems.  

The syntax for patterns, terms, and contexts are described by the
following grammar:
\[
\begin{array}{cll}
  \text{(patterns)} & [[p]] := [[h]] \mid [[x]] \mid [[punit]] \mid [[p1 (x) p2]]\\
  \text{(terms)}    & [[t]] := [[x]] \mid [[unit]] \mid [[t1 (x) t2]] \mid [[\l x : A.t]] \mid [[\r x : A.t]] \mid
  [[appl t1 t2]] \mid \\ & [[appr t1 t2]] \mid [[let t1 be p in t2]]\\
  \text{(contexts)} & [[G]] := [[.]] \mid [[x : A]] \mid [[G1,G2]]\\
\end{array}
\]
Contexts are sequences of pairs of free variables and types.  Patterns
are only used in the let-expression which is itself used to eliminate
logical connectives within the left rules of L.  All variables in the
pattern of a let-expression are bound.  The remainder of the terms are
straightforward.

The typing rules can be found in Figure~\ref{fig:typed-L} and the
reduction rules in Figure~\ref{fig:rewrite-L}.
\begin{figure}[h]
  \begin{mdframed}
    \begin{mathpar}
      \LdruleTXXvar{} \and
      \LdruleTXXUr{} \and
      \LdruleTXXcut{} \and
      \LdruleTXXUl{} \and
      \LdruleTXXTl{} \and
      \LdruleTXXTr{} \and
      \LdruleTXXIRl{} \and
      \LdruleTXXILl{} \and
      \LdruleTXXIRr{} \and
      \LdruleTXXILr{} 
    \end{mathpar}
  \end{mdframed}
  \caption{Typing Rules for The Typed Lambek Calculus: $\lambda\text{L}$}
  \label{fig:typed-L}
\end{figure}
\begin{figure}[h]
  \begin{mdframed}
    \begin{mathpar}
      %% \LdruleRXXAlpha{} \and
      \LdruleRXXBetal{} \and
      \LdruleRXXBetar{} \and
      \LdruleRXXBetaU{} \and
      \LdruleRXXBetaTOne{} \and
      \LdruleRXXBetaTTwo{} \and
      \LdruleRXXNatU{} \and
      \LdruleRXXNatT{} \and
      \LdruleRXXLetU{} \and
      %% \LdruleRXXLet{} \and
      %% \LdruleRXXTOne{} \and
      %% \LdruleRXXTTwo{} \and
      %% \LdruleRXXAlOne{} \and
      %% \LdruleRXXAlTwo{} \and
      %% \LdruleRXXArOne{} \and
      %% \LdruleRXXArTwo{} \and
      %% \LdruleRXXLl{} \and
      %% \LdruleRXXLr{} 
    \end{mathpar}
  \end{mdframed}
  \caption{Rewriting Rules for The Typed Lambek Calculus: $\lambda\text{L}$}
  \label{fig:rewrite-L}
\end{figure}
The typing rules are as one might expect.  Finally, the reduction
rules were extracted from the cut-elimination procedure for L.  We
denote the reflexive and transitive closure of the $[[~>]]$ by
$[[~*>]]$.  We call a term with no $\beta$-redexes a normal form,
and we denote normal forms by $[[n]]$.  In the interest of space we
omit the congruence rules from the definition of the reduction
relation; we will do this for each calculi introduced throughout this
section.  The other typed calculi we introduce below will be
extensions of $\lambda\text{L}$, thus, we do not reintroduce these
rules each time for readability.

\textbf{Strong normalization.}  It is well known that intuitionistic
linear logic (ILL) is strongly normalizing, for example, see Bierman's
thesis \cite{Bierman:1994} or Benton's beautiful embedding of ILL into
system F \cite{Benton:1995c}.  It is fairly straightforward to define
a reduction preserving embedding of $\lambda\text{L}$ -- and as we
will see the other calculi can be as well -- into ILL.  Intuitionistic
linear logic can be obtained from $\lambda\text{L}$ by replacing the
rules $\Ldrulename{T\_IRl}$, $\Ldrulename{T\_ILl}$,
$\Ldrulename{T\_IRr}$, and $\Ldrulename{T\_ILr}$ with the following
two rules:
\[
\begin{array}{cccccccc}
  \LdruleTXXIl{} & \quad \LdruleTXXIr{}
\end{array}
\]
In addition, contexts are considered multisets, and hence, exchange is
handled implicitly.

At this point we define the following embeddings.
\begin{definition}
  \label{def:lambda-L_to_ILL}
  We embed types and terms of $\lambda\text{L}$ into ILL as follows:
  \begin{center}
    \begin{math}
      \begin{array}{lrllllllllllllllllll}
        \text{Types:}\\
        & [[I e]] & = & [[I]]\\
        & [[(A (x) B) e]] & = & [[A e (x) h(B e)]]\\
        & [[(A -> B) e]] & = & [[A e -o h(B e)]]\\
        & [[(A <- B) e]] & = & [[A e -o h(B e)]]\\                    
        \\
        \text{Terms:}\\
        & [[x e]] & = & [[x]]\\
        & [[unit e]] & = & [[unit]]\\
        & [[{t1 (x) t2} e]] & = & [[t1 e (x) h{t2 e}]]\\
        & [[{let t1 be p in t2} e]] & = & [[let t1 e be p in h{t2 e}]]\\
        & [[{\l x : A.t} e]] & = & [[\ x : A.h{t e}]]\\
        & [[{\r x : A.t} e]] & = & [[\ x : A.h{t e}]]\\
        & [[{appl t1 t2} e]] & = & [[h{t1 e} h{t2 e}]]\\
        & [[{appr t1 t2} e]] & = & [[h{t1 e} h{t2 e}]]\\
      \end{array}
    \end{math}
  \end{center}
  The previous embeddings can be extended to contexts in the
  straightforward way, and to sequents as follows:
  \[ [[(G |- t : A) e]] = [[G e |- t e : h(A e)]]\]
\end{definition}
\noindent
This embedding preserves typing and reduction.
\begin{lemma}[Type Preserving Embedding]
  \label{lemma:type_preserving_embedding}
  If $[[G |- t : A]]$ in $\lambda\text{L}$, then
  $[[G e |- t e : h(A e)]]$ in ILL.
\end{lemma}
\begin{proof}
  This holds by straightforward induction on the form of the assumed
  typing derivation.
\end{proof}

\begin{lemma}[Reduction Preserving Embedding]
  \label{lemma:reduction_preserving_embedding}
  If $[[t1 ~> t2]]$ in $\lambda\text{L}$, then $[[t1 e ~> h{t2 e}]]$
  in ILL.
\end{lemma}
\begin{proof}
  This holds by straightforward induction on the form of the assumed
  reduction derivation.
\end{proof}
\noindent
The previous two results imply that $\lambda\text{L}$ is strongly
normalizing, because ILL is.
\begin{corollary}[Strong Normalization]
  \label{corollary:strong_normalization}
  If $[[G |- t : A]]$, then $[[t]]$ is strongly normalizing.
\end{corollary}
\begin{proof}
  Suppose $[[t]]$ is not strongly normalizing.  Then there exists at
  least one infinite descending chain starting with $[[t]]$ of the
  form $[[t]] \redto [[t']] \redto [[t'']] \redto
  \cdots$. Lemma~\ref{lemma:reduction_preserving_embedding} implies
  that there must also be an infinite descending chain of the form
  $[[t e]] \redto [[t' e]] \redto [[t'' e]] \redto \cdots$, but we
  know this is impossible, because ILL is strongly normalizing.
\end{proof}

\textbf{Confluence.} The Church-Rosser property is well known to hold
for ILL modulo commuting conversions, for example, see Theorem~19 of
\cite{Bierman:1994} on page 96.  Since $\lambda\text{L}$ is
essentially a subsystem of ILL, it is straightforward, albeit
lengthly, to simply redefine Bierman's candidates (Definitions~17, 19,
20, and 21 of Section 5.1 of ibid.), and then carry out a similar proof
as Bierman's (Theorem~19 on page 96 of ibid.).  However, giving the
details here would not be fruitful nor enlightening, and hence, we
omit the proof.  Thus, we have the following:
\begin{theorem}[Confluence]
  \label{thm:confluence}
  The reduction relation, $[[~>]]$, modulo the commuting conversions
  is confluent.
\end{theorem}
% subsection the_typed_lambek_calculi_lambda-l (end)

\subsection{The Typed Lambek Calculus: $\lambda\text{L}_!$}
\label{subsec:the_typed_lambek_calculus:lambda-l!}
The calculus we introduce in this section is an extension of
$\lambda\text{L}$ with the of-course modality $[[!A]]$.  This
extension follows from ILL exactly.  The syntax of types and terms of
$\lambda\text{L}$ are extended as follows:
\[
\begin{array}{cllllll}
  \text{(types)}    & [[A]] & := & \cdots \mid [[! A]]\\
  \text{(terms)}    & [[t]] & := & \cdots \mid [[copy t' as t1 , t2 in t]] \mid [[discard t' in t]]
  \mid [[promote! t' for t'' in t]] \mid \\ & & & [[derelict! t]]\\
\end{array}
\]
The new type and terms are what one might expect, and are the
traditional syntax used for the of-course modality.  We add the
following typing rules to $\lambda\text{L}$:
\begin{mathpar}
      \LdruleTXXC{} \and
      \LdruleTXXW{} \and
      \LdruleTXXBr{} \and
      \LdruleTXXBl{} 
\end{mathpar}
\begin{figure}
  \small
  \begin{mdframed}
    \begin{mathpar}
      \LdruleRXXBetaDR{} \and
      \LdruleRXXBetaDI{} \and
      \LdruleRXXBetaC{} \and
      \LdruleRXXNatD{} \and
      \LdruleRXXNatC{} 
    \end{mathpar}
  \end{mdframed}
  \caption{Rewriting Rules for The Typed Lambek Calculus: $\lambda\text{L}_!$}
  \label{fig:rewrite-LB}
\end{figure}
Finally, the reduction rules can be found in
Figure~\ref{fig:rewrite-LB}.  The equality used in the
$\Ldrulename{R\_BetaC}$ rule is definitional, meaning, that the rule
simply gives the terms on the right side of the equation the name on
the left side, and nothing more.  This makes the rule easier to read.

\textbf{Strong normalization.} Showing strong normalization for
$\lambda\text{L}_!$ easily follows by a straightforward extension of
the embedding we gave for $\lambda\text{L}$.
\begin{definition}
  \label{def:embed-lambda-L!-in-ILL}
  The following is an extension of the embedding of $\lambda\text{L}$
  into ILL resulting in an embedding of types and terms of
  $\lambda\text{L}_!$ into ILL:
  \begin{center}
    \begin{math}
      \begin{array}{lrllllllllllllllllll}
        \text{Types:}\\
        & [[(! A) e]] & = & [[! h(A e)]]\\        
        \\
        \text{Terms:}\\
        & [[{copy t' as t1 , t2 in t} e]] & = & [[copy t' e as t1 e , t2 e in h{t e}]]\\
        & [[{discard t' in t} e]] & = & [[discard t' e in h{t e}]]\\
        & [[{promote! t' for t'' in t} e]] & = & [[promote! t' e for t'' e in h{t e}]]\\
        & [[{derelict! t} e]] & = & [[derelict! h{t e}]]\\
      \end{array}
    \end{math}
  \end{center}
\end{definition}

\noindent
Just as before this embedding is type preserving and reduction preserving.
\begin{lemma}[Type Preserving Embedding]
  \label{lemma:type_preserving_embedding_lambdaL!}
  If $[[G |- t : A]]$ in $\lambda\text{L}_!$, then
  $[[G e |- t e : h(A e)]]$ in ILL.
\end{lemma}
\begin{proof}
  This holds by straightforward induction on the form of the assumed
  typing derivation.
\end{proof}

\begin{lemma}[Reduction Preserving Embedding]
  \label{lemma:reduction_preserving_embedding_lambdaL!}
  If $[[t1 ~> t2]]$ in $\lambda\text{L}_!$, then $[[t1 e ~> h{t2 e}]]$
  in ILL.
\end{lemma}
\begin{proof}
  This holds by straightforward induction on the form of the assumed
  reduction derivation.
\end{proof}
\noindent
Finally, we can use the previous two results to conclude strong
normalization, because ILL is.
\begin{corollary}[Strong Normalization]
  \label{corollary:strong_normalization_lambdaL!}
  If $[[G |- t : A]]$, then $[[t]]$ is strongly normalizing.
\end{corollary}

\textbf{Confluence.} The Church-Rosser property also holds for
$\lambda\text{L}_!$, and can be shown by straightforwardly applying a
slightly modified version of Bierman's proof \cite{Bierman:1994} just
as we did for $\lambda\text{L}$.  Thus, we have the following:
\begin{theorem}[Confluence]
  \label{thm:confluence}
  The reduction relation, $[[~>]]$, modulo the commuting conversions
  is confluent.
\end{theorem}
% subsection the_typed_lambek_calculus:lambda-l! (end)

\subsection{The Typed Lambek Calculus: $\lambda\text{L}_\kappa$}
\label{subsec:the_typed_lambek_calculus:lambda-l-kappa}

The next calculus we introduce is also an extension of
$\lambda\text{L}$ with a modality that adds exchange to
$\lambda\text{L}_\kappa$ denoted $[[E A]]$.  It is perhaps the most
novel of the calculi we have introduced so far.  The syntax of types
and terms of $\lambda\text{L}$ are extended as follows:
\[
\begin{array}{cllllll}
  \text{(types)}    & [[A]] & := & \cdots \mid [[E A]]\\
  \text{(terms)}    & [[t]] & := & \cdots \mid [[lexchange t1,t2 with x,y in t3]] \mid [[rexchange t1,t2 with x,y in t3]] \mid 
  \\ & & & [[promoteE t' for t'' in t]] \mid [[derelictE t]]\\
\end{array}
\]
The syntax for types has been extended to include the exchange
modality, and the syntax of terms follow suit.  The terms $[[lexchange
    t1,t2 with x,y in t3]]$ and $[[rexchange t1,t2 with x,y in t3]]$
are used to explicitly track uses of exchange throughout proofs.  We
add the following typing rules to $\lambda\text{L}$:
\begin{mathpar}
  \LdruleTXXEOne{} \and
  \LdruleTXXETwo{} \and
  \LdruleTXXEr{} \and
  \LdruleTXXEl{} 
\end{mathpar}

\begin{figure}[h]
  \small
  \begin{mdframed}
    \begin{mathpar}      
      \LdruleRXXBetaEDR{} \and
      \LdruleRXXNatEl{} \and
      \LdruleRXXNatEr{} 
    \end{mathpar}
  \end{mdframed}
  \caption{Rewriting Rules for The Typed Lambek Calculus: $\lambda\text{L}_\kappa$}
  \label{fig:rewrite-LE}
\end{figure}
\noindent
The reduction rules are in Figure~\ref{fig:rewrite-LE}, and are vary
similar to the rules from $\lambda\text{L}_!$.

\textbf{Strong normalization.}  Similarly, we show that we can embed
$\lambda\text{L}_\kappa$ into ILL, but the embedding is a bit more
interesting.

\begin{definition}
  \label{def:embed-lambda-L!-in-ILL}
  The following is an extension of the embedding of $\lambda\text{L}$
  into ILL resulting in an embedding of types and terms of
  $\lambda\text{L}_\kappa$ into ILL:
  \begin{center}
    \begin{math}
      \begin{array}{lrllllllllllllllllll}
        \text{Types:}\\
        & [[(E A) e]] & = & [[! h(A e)]]\\        
        \\
        \text{Terms:}\\
        & [[{lexchange t1 , t2 with x , y in t3} e]] & = & [[ [t2 e/x][t1 e/y]h{t3 e}]]\\
        & [[{rexchange t1 , t2 with x , y in t3} e]] & = & [[ [t2 e/x][t1 e/y]h{t3 e}]]\\
        & [[{promoteE t' for t'' in t} e]] & = & [[promote! t' e for t'' e in h{t e}]]\\
        & [[{derelictE t} e]] & = & [[derelict! h{t e}]]\\
      \end{array}
    \end{math}
  \end{center}
\end{definition}
The embedding just defined translates the exchange modality into the
of-course modality of ILL.  We do this so as to preserve the comonadic
structure of the exchange modality.  One might think that we could
simply translate the exchange modality to the identity, but as Benton
showed \cite{Benton:1995c}, this would result in an embedding that
does not preserve reductions.  Furthermore, the left and right
exchange terms are translated away completely, but this works because
ILL contains exchange in general, and hence, does not need to be
tracked explicitly.  It is now straightforward to prove the following
results.
\begin{lemma}[Type Preserving Embedding]
  \label{lemma:type_preserving_embedding_lambdaLk}
  If $[[G |- t : A]]$ in $\lambda\text{L}_!$, then
  $[[G e |- t e : h(A e)]]$ in ILL.
\end{lemma}
\begin{proof}
  This holds by straightforward induction on the form of the assumed
  typing derivation.
\end{proof}
\begin{lemma}[Reduction Preserving Embedding]
  \label{lemma:reduction_preserving_embedding_lambdaLk}
  If $[[t1 ~> t2]]$ in $\lambda\text{L}_!$, then $[[t1 e ~> h{t2 e}]]$
  in ILL.
\end{lemma}
\begin{proof}
  This holds by straightforward induction on the form of the assumed
  reduction derivation.
\end{proof}
\noindent
Finally, we can use the previous two results to conclude strong
normalization, because ILL is.
\begin{corollary}[Strong Normalization]
  \label{corollary:strong_normalization_lambdaLk}
  If $[[G |- t : A]]$, then $[[t]]$ is strongly normalizing.
\end{corollary}

\textbf{Confluence.} The Church-Rosser property also holds for
$\lambda\text{L}_\kappa$, and the proof is nearly identical to the
proof that $\lambda\text{L}_!$ is confluent.  Thus, we have the
following:
\begin{theorem}[Confluence]
  \label{thm:confluence-lambdaLk}
  The reduction relation, $[[~>]]$, modulo the commuting conversions
  is confluent.
\end{theorem}
% subsection the_typed_lambek_calculus:lambda-l-kappa (end)

\subsection{The Typed Lambek Calculus: $\lambda\text{L}_{!\kappa}$}
\label{subsec:the_typed_lambek_calculus:lambda-l-!kappa}
If we combine all three of the previous typed Lambek calculi, then we
obtain the typed Lambek calculus $\lambda\text{L}_{!\kappa}$.  We do
not explicitly define the syntax, typing rules, and reduction rules
here, because we would be repeating what we have already defined in
the previous sections.  The benefits of this system is that it
provides the structure of the Lambek calculus with the benefits of
having exchange, and the of-course modality, but both are carefully
tracked within the proofs.

Strong normalization for this calculus can be proved similarly to the
previous calculi by simply merging the embeddings together.  Thus,
both modalities of $\lambda\text{L}_{!\kappa}$ would merge into the
of-course modality of ILL.  The Church-Rosser property also holds for
$\lambda\text{L}_{!\kappa}$ by extending the proof of confluence for
ILL by Bierman \cite{Bierman:1994} just as we did for the other
systems.  Thus, we have the following results.

\begin{theorem}[Strong Normalization]
  \label{theorem:strong_normalization_lambdaL!k}
  If $[[G |- t : A]]$, then $[[t]]$ is strongly normalizing.
\end{theorem}

\begin{theorem}[Confluence]
  \label{thm:confluence-lambdaL!k}
  The reduction relation, $[[~>]]$, modulo the commuting conversions
  is confluent.
\end{theorem}
% subsection the_typed_lambek_calculus:lambda-l-!kappa (end)
% section type_theories (end)


\section{Conclusion}

\bibliographystyle{plain}
\bibliography{references}

%% \appendix
%% \section{Term Annotated Cut Elimination Procedure for L}
%% \label{sec:cut_elimination_procedure_for_l}
%% \begin{itemize}
\item[Case:] commuting conversion cut vs cut (first case).
  The following proof
  \begin{center}
    \begin{math}
      $$\mprset{flushleft}
      \inferrule* [right=cut] {
        $$\mprset{flushleft}
        \inferrule* [right=] {
          \inferrule* [right=,vdots=1.5em,fraction=\,] {
            \,
          }{\pi_1}
        }{ \Gamma_{{\mathrm{1}}}  \vdash  \Lnt{t}  :  \Lnt{A} }
        \\
        $$\mprset{flushleft}
        \inferrule* [right=cut] {
          $$\mprset{flushleft}
          \inferrule* [right=] {
            \inferrule* [right=,vdots=1.5em,fraction=\,] {
              \,
            }{\pi_2}
          }{ \Lmv{x}  \Lsym{:}  \Lnt{A}  \Lsym{,}  \Gamma_{{\mathrm{2}}}  \vdash  \Lnt{t_{{\mathrm{1}}}}  :  \Lnt{B} }
          \\
          $$\mprset{flushleft}
          \inferrule* [right=] {
            \inferrule* [right=,vdots=1.5em,fraction=\,] {
              \,
            }{\pi_3}
          }{ \Lmv{y}  \Lsym{:}  \Lnt{B}  \Lsym{,}  \Gamma_{{\mathrm{3}}}  \vdash  \Lnt{t'}  :  \Lnt{C} }
        }{ \Lmv{x}  \Lsym{:}  \Lnt{A}  \Lsym{,}  \Gamma_{{\mathrm{2}}}  \Lsym{,}  \Gamma_{{\mathrm{3}}}  \vdash  \Lsym{[}  \Lnt{t_{{\mathrm{1}}}}  \Lsym{/}  \Lmv{y}  \Lsym{]}  \Lnt{t'}  :  \Lnt{C} }
      }{ \Gamma_{{\mathrm{1}}}  \Lsym{,}  \Gamma_{{\mathrm{2}}}  \Lsym{,}  \Gamma_{{\mathrm{3}}}  \vdash  \Lsym{[}  \Lnt{t}  \Lsym{/}  \Lmv{x}  \Lsym{]}  \Lsym{[}  \Lnt{t_{{\mathrm{1}}}}  \Lsym{/}  \Lmv{y}  \Lsym{]}  \Lnt{t'}  :  \Lnt{C} }
    \end{math}
  \end{center}
  is transformed into the proof
  \begin{center}
    \begin{math}
      $$\mprset{flushleft}
      \inferrule* [right=cut] {
        $$\mprset{flushleft}
        \inferrule* [right=] {
          $$\mprset{flushleft}
          \inferrule* [right=] {
            \inferrule* [right=,vdots=1.5em,fraction=\,] {
              \,
            }{\pi_1}               
          }{ \Gamma_{{\mathrm{1}}}  \vdash  \Lnt{t}  :  \Lnt{A} }
          \\
          $$\mprset{flushleft}
          \inferrule* [right=] {
            \inferrule* [right=,vdots=1.5em,fraction=\,] {
              \,
            }{\pi_2}
          }{ \Lmv{x}  \Lsym{:}  \Lnt{A}  \Lsym{,}  \Gamma_{{\mathrm{2}}}  \vdash  \Lnt{t_{{\mathrm{1}}}}  :  \Lnt{B} }
        }{ \Gamma_{{\mathrm{1}}}  \Lsym{,}  \Gamma_{{\mathrm{2}}}  \vdash  \Lsym{[}  \Lnt{t}  \Lsym{/}  \Lmv{x}  \Lsym{]}  \Lnt{t_{{\mathrm{1}}}}  :  \Lnt{B} }
        \\
        $$\mprset{flushleft}
        \inferrule* [right=] {
          \inferrule* [right=,vdots=1.5em,fraction=\,] {
            \,
          }{\pi_3}
        }{ \Lmv{y}  \Lsym{:}  \Lnt{B}  \Lsym{,}  \Gamma_{{\mathrm{3}}}  \vdash  \Lnt{t'}  :  \Lnt{C} }
      }{ \Gamma_{{\mathrm{1}}}  \Lsym{,}  \Gamma_{{\mathrm{2}}}  \Lsym{,}  \Gamma_{{\mathrm{3}}}  \vdash  \Lsym{[}  \Lsym{[}  \Lnt{t}  \Lsym{/}  \Lmv{x}  \Lsym{]}  \Lnt{t_{{\mathrm{1}}}}  \Lsym{/}  \Lmv{y}  \Lsym{]}  \Lnt{t'}  :  \Lnt{C} }
    \end{math}
  \end{center}

\item[Case:] commuting conversion cut vs. cut (second case).  The second commuting conversion on cut begins with the proof
  \begin{center}
    \begin{math}
      $$\mprset{flushleft}
      \inferrule* [right=cut] {
        $$\mprset{flushleft}
        \inferrule* [right=] {
          \inferrule* [right=,vdots=1.5em,fraction=\,] {
            \,
          }{\pi_1}
        }{ \Gamma  \vdash  \Lnt{t}  :  \Lnt{A} }
        \\
        $$\mprset{flushleft}
        \inferrule* [right=cut] {
          $$\mprset{flushleft}
          \inferrule* [right=] {
            \inferrule* [right=,vdots=1.5em,fraction=\,] {
              \,
            }{\pi_2}
          }{ \Gamma'  \vdash  \Lnt{t'}  :  \Lnt{B} }
          \\
          $$\mprset{flushleft}
          \inferrule* [right=] {
            \inferrule* [right=,vdots=1.5em,fraction=\,] {
              \,
            }{\pi_3}
          }{ \Lmv{x}  \Lsym{:}  \Lnt{A}  \Lsym{,}  \Gamma_{{\mathrm{2}}}  \Lsym{,}  \Lmv{y}  \Lsym{:}  \Lnt{B}  \Lsym{,}  \Gamma_{{\mathrm{3}}}  \vdash  \Lnt{t''}  :  \Lnt{C} }
        }{ \Gamma_{{\mathrm{1}}}  \Lsym{,}  \Lmv{x}  \Lsym{:}  \Lnt{A}  \Lsym{,}  \Gamma_{{\mathrm{2}}}  \Lsym{,}  \Gamma'  \Lsym{,}  \Gamma_{{\mathrm{3}}}  \vdash  \Lsym{[}  \Lnt{t'}  \Lsym{/}  \Lmv{y}  \Lsym{]}  \Lnt{t''}  :  \Lnt{C} }
      }{ \Gamma_{{\mathrm{1}}}  \Lsym{,}  \Gamma  \Lsym{,}  \Gamma_{{\mathrm{2}}}  \Lsym{,}  \Gamma'  \Lsym{,}  \Gamma_{{\mathrm{3}}}  \vdash  \Lsym{[}  \Lnt{t}  \Lsym{/}  \Lmv{x}  \Lsym{]}  \Lsym{[}  \Lnt{t'}  \Lsym{/}  \Lmv{y}  \Lsym{]}  \Lnt{t''}  :  \Lnt{C} }
    \end{math}
  \end{center}
  is transformed into the following proof:
  \begin{center}
    \begin{math}
      $$\mprset{flushleft}
      \inferrule* [right=cut] {
        $$\mprset{flushleft}
        \inferrule* [right=] {
          \inferrule* [right=,vdots=1.5em,fraction=\,] {
            \,
          }{\pi_2}
        }{ \Gamma'  \vdash  \Lnt{t'}  :  \Lnt{B} }
        \\
        $$\mprset{flushleft}
        \inferrule* [right=cut] {
          $$\mprset{flushleft}
          \inferrule* [right=] {
            \inferrule* [right=,vdots=1.5em,fraction=\,] {
              \,
            }{\pi_1}
          }{ \Gamma  \vdash  \Lnt{t}  :  \Lnt{A} }      
          \\
          $$\mprset{flushleft}
          \inferrule* [right=] {
            \inferrule* [right=,vdots=1.5em,fraction=\,] {
              \,
            }{\pi_3}
          }{ \Gamma_{{\mathrm{1}}}  \Lsym{,}  \Lmv{x}  \Lsym{:}  \Lnt{A}  \Lsym{,}  \Gamma_{{\mathrm{2}}}  \Lsym{,}  \Lmv{y}  \Lsym{:}  \Lnt{B}  \Lsym{,}  \Gamma_{{\mathrm{3}}}  \vdash  \Lnt{t''}  :  \Lnt{C} }
        }{ \Gamma_{{\mathrm{1}}}  \Lsym{,}  \Gamma  \Lsym{,}  \Gamma_{{\mathrm{2}}}  \Lsym{,}  \Lmv{y}  \Lsym{:}  \Lnt{B}  \Lsym{,}  \Gamma_{{\mathrm{3}}}  \vdash  \Lsym{[}  \Lnt{t}  \Lsym{/}  \Lmv{x}  \Lsym{]}  \Lnt{t''}  :  \Lnt{C} }
      }{ \Gamma_{{\mathrm{1}}}  \Lsym{,}  \Gamma  \Lsym{,}  \Gamma_{{\mathrm{2}}}  \Lsym{,}  \Gamma'  \Lsym{,}  \Gamma_{{\mathrm{3}}}  \vdash  \Lsym{[}  \Lnt{t'}  \Lsym{/}  \Lmv{y}  \Lsym{]}  \Lsym{[}  \Lnt{t}  \Lsym{/}  \Lmv{x}  \Lsym{]}  \Lnt{t''}  :  \Lnt{C} }
    \end{math}
  \end{center}
  

\item[Case:] the $\eta$-expansion cases: tensor.  
  The proof
  \begin{center}
    \begin{math}
      $$\mprset{flushleft}
      \inferrule* [right=Var] {
        \,
      }{ \Lmv{x}  \Lsym{:}   \Lnt{A}  \otimes  \Lnt{B}   \vdash  \Lmv{x}  :   \Lnt{A}  \otimes  \Lnt{B}  }
    \end{math}
  \end{center}
  is transformed into the proof
  \begin{center}
    \begin{math}
      $$\mprset{flushleft}
      \inferrule* [right=Tl] {
        $$\mprset{flushleft}
        \inferrule* [right=Tr] {
          $$\mprset{flushleft}
          \inferrule* [right=Var] {
            \,
          }{ \Lmv{y}  \Lsym{:}  \Lnt{A}  \vdash  \Lmv{y}  :  \Lnt{A} }
          \\
          $$\mprset{flushleft}
          \inferrule* [right=Var] {
            \,
          }{ \Lmv{z}  \Lsym{:}  \Lnt{B}  \vdash  \Lmv{z}  :  \Lnt{B} }
        }{ \Lmv{y}  \Lsym{:}  \Lnt{A}  \Lsym{,}  \Lmv{z}  \Lsym{:}  \Lnt{B}  \vdash   \Lmv{y}  \otimes  \Lmv{z}   :   \Lnt{A}  \otimes  \Lnt{B}  }
      }{ \Lmv{x}  \Lsym{:}   \Lnt{A}  \otimes  \Lnt{B}   \vdash   \mathsf{let}\, \Lmv{x} \,\mathsf{be}\,  \Lmv{y}  \otimes  \Lmv{z}  \,\mathsf{in}\,  (   \Lmv{y}  \otimes  \Lmv{z}   )    :   \Lnt{A}  \otimes  \Lnt{B}  }
    \end{math}
  \end{center}

\item[Case:] the $\eta$-expansion cases: implication.
  The proof
  \begin{center}
    \begin{math}
      $$\mprset{flushleft}
      \inferrule* [right=Var] {
        \,
      }{ \Lmv{x}  \Lsym{:}   \Lnt{A}  \rightharpoonup  \Lnt{B}   \vdash  \Lmv{x}  :   \Lnt{A}  \rightharpoonup  \Lnt{B}  }
    \end{math}
  \end{center}
  transforms into the proof 
  \begin{center}
    \begin{math}
      $$\mprset{flushleft}
      \inferrule* [right=IRr] {
        $$\mprset{flushleft}
        \inferrule* [right=IRl] {
          $$\mprset{flushleft}
          \inferrule* [right=Var] {
            \,
          }{ \Lmv{y}  \Lsym{:}  \Lnt{A}  \vdash  \Lmv{y}  :  \Lnt{A} }
          \\
          $$\mprset{flushleft}
          \inferrule* [right=Var] {
            \,
          }{ \Lmv{z}  \Lsym{:}  \Lnt{B}  \vdash  \Lmv{z}  :  \Lnt{B} }
        }{ \Lmv{y}  \Lsym{:}  \Lnt{A}  \Lsym{,}  \Lmv{x}  \Lsym{:}   \Lnt{A}  \rightharpoonup  \Lnt{B}   \vdash   \mathsf{app}_r\, \Lmv{x} \, \Lmv{y}   :  \Lnt{B} }
      }{ \Lmv{x}  \Lsym{:}   \Lnt{A}  \rightharpoonup  \Lnt{B}   \vdash   \lambda_r  \Lmv{y} : \Lnt{A} .  \mathsf{app}_r\, \Lmv{x} \, \Lmv{y}    :   \Lnt{A}  \rightharpoonup  \Lnt{B}  }
    \end{math}  
  \end{center}

\item[Case:] the $\eta$-expansion cases: implication.
  The proof
  \begin{center}
    \begin{math}
      $$\mprset{flushleft}
      \inferrule* [right=Var] {
        \,
      }{ \Lmv{x}  \Lsym{:}   \Lnt{B}  \leftharpoonup  \Lnt{A}   \vdash  \Lmv{x}  :   \Lnt{B}  \leftharpoonup  \Lnt{A}  }
    \end{math}
  \end{center}
  transforms into the proof 
  \begin{center}
    \begin{math}
      $$\mprset{flushleft}
      \inferrule* [right=ILr] {
        $$\mprset{flushleft}
        \inferrule* [right=ILl] {
          $$\mprset{flushleft}
          \inferrule* [right=Var] {
            \,
          }{ \Lmv{y}  \Lsym{:}  \Lnt{A}  \vdash  \Lmv{y}  :  \Lnt{A} }
          \\
          $$\mprset{flushleft}
          \inferrule* [right=Var] {
            \,
          }{ \Lmv{z}  \Lsym{:}  \Lnt{B}  \vdash  \Lmv{z}  :  \Lnt{B} }
        }{ \Lmv{y}  \Lsym{:}  \Lnt{A}  \Lsym{,}  \Lmv{x}  \Lsym{:}   \Lnt{B}  \leftharpoonup  \Lnt{A}   \vdash   \mathsf{app}_l\, \Lmv{x} \, \Lmv{y}   :  \Lnt{B} }
      }{ \Lmv{x}  \Lsym{:}   \Lnt{B}  \leftharpoonup  \Lnt{A}   \vdash   \lambda_l  \Lmv{y} : \Lnt{A} .  \mathsf{app}_l\, \Lmv{x} \, \Lmv{y}    :   \Lnt{A}  \leftharpoonup  \Lnt{B}  }
    \end{math}  
  \end{center}  
  
\item[Case:] the $\eta$-expansion cases: tensor unit.
  The proof
  \begin{center}
    \begin{math}
      $$\mprset{flushleft}
      \inferrule* [right=Var] {
        \,
      }{ \Lmv{x}  \Lsym{:}   I   \vdash  \Lmv{x}  :   I  }
    \end{math}
  \end{center}
  transforms into the proof
  \begin{center}
    \begin{math}
      $$\mprset{flushleft}
      \inferrule* [right=Ul] {
        $$\mprset{flushleft}
        \inferrule* [right=Ur] {
          \,
        }{  \cdot   \vdash   \mathsf{unit}   :   I  }
      }{ \Lmv{x}  \Lsym{:}   I   \vdash   \mathsf{let}\, \Lmv{x} \,\mathsf{be}\,  \mathsf{unit}  \,\mathsf{in}\,  \mathsf{unit}    :   I  }
    \end{math}
  \end{center}

\item[Case:] the axiom steps: the axiom step. The proof 
  \begin{center}
    \begin{math}
      $$\mprset{flushleft}
      \inferrule* [right=cut] {
        $$\mprset{flushleft}
        \inferrule* [right=Var] {
          \,
        }{ \Lmv{x}  \Lsym{:}  \Lnt{A}  \vdash  \Lmv{x}  :  \Lnt{A} }
        \\
        $$\mprset{flushleft}
        \inferrule* [right=] {
          \inferrule* [right=,vdots=1.5em,fraction=\,] {
            \,
          }{\pi}          
        }{ \Lmv{y}  \Lsym{:}  \Lnt{A}  \Lsym{,}  \Gamma  \vdash  \Lnt{t}  :  \Lnt{C} }
      }{ \Lmv{x}  \Lsym{:}  \Lnt{A}  \Lsym{,}  \Gamma  \vdash  \Lsym{[}  \Lmv{x}  \Lsym{/}  \Lmv{y}  \Lsym{]}  \Lnt{t}  :  \Lnt{C} }
    \end{math}
  \end{center}
  transforms into the proof
  \begin{center}
    \begin{math}
      $$\mprset{flushleft}
      \inferrule* [right=] {
        \inferrule* [right=,vdots=1.5em,fraction=\,] {
          \,
        }{\pi}          
      }{ \Lmv{y}  \Lsym{:}  \Lnt{A}  \Lsym{,}  \Gamma  \vdash  \Lnt{t}  :  \Lnt{C} }
    \end{math}
  \end{center}

\item[Case:] the axiom steps: conclusion vs. axiom.
  The proof 
  \begin{center}
    \begin{math}
      $$\mprset{flushleft}
      \inferrule* [right=cut] {
        $$\mprset{flushleft}
        \inferrule* [right=] {
          \inferrule* [right=,vdots=1.5em,fraction=\,] {
            \,
          }{\pi}          
        }{ \Gamma  \vdash  \Lnt{t}  :  \Lnt{A} }
        \,
        $$\mprset{flushleft}
        \inferrule* [right=var] {
          \,
        }{ \Lmv{x}  \Lsym{:}  \Lnt{A}  \vdash  \Lmv{x}  :  \Lnt{A} }
      }{ \Gamma  \vdash  \Lsym{[}  \Lnt{t}  \Lsym{/}  \Lmv{x}  \Lsym{]}  \Lmv{x}  :  \Lnt{A} }
    \end{math}
  \end{center}
  transforms into 
  \begin{center}
    \begin{math}      
      \inferrule* [right=] {
        \inferrule* [right=,vdots=1.5em,fraction=\,] {
          \,
        }{\pi}          
      }{ \Gamma  \vdash  \Lnt{t}  :  \Lnt{A} }
    \end{math}
  \end{center}  
  
\item[Case:] principal formula vs. principal formula: tensor.
  The proof 
  \begin{center}
    \scriptsize
    \begin{math}
      $$\mprset{flushleft}
      \inferrule* [right=\footnotesize cut] {
        $$\mprset{flushleft}
        \inferrule* [right=\footnotesize Tr] {
          \inferrule* [right=] {
            \inferrule* [right=,vdots=1.5em,fraction=\,] {
              \,
            }{\pi_1}          
          }{ \Gamma_{{\mathrm{1}}}  \vdash  \Lnt{t_{{\mathrm{1}}}}  :  \Lnt{A} }
          \\
          \inferrule* [right=] {
            \inferrule* [right=,vdots=1.5em,fraction=\,] {
              \,
            }{\pi_2}          
          }{ \Gamma_{{\mathrm{2}}}  \vdash  \Lnt{t_{{\mathrm{2}}}}  :  \Lnt{B} }
        }{ \Gamma_{{\mathrm{1}}}  \Lsym{,}  \Gamma_{{\mathrm{2}}}  \vdash   \Lnt{t_{{\mathrm{1}}}}  \otimes  \Lnt{t_{{\mathrm{2}}}}   :   \Lnt{A}  \otimes  \Lnt{B}  }
        \\
        $$\mprset{flushleft}
        \inferrule* [right=\footnotesize Tl] {
          \inferrule* [right=] {
            \inferrule* [right=,vdots=1.5em,fraction=\,] {
              \,
            }{\pi_3}          
          }{ \Gamma_{{\mathrm{3}}}  \Lsym{,}  \Lmv{x}  \Lsym{:}  \Lnt{A}  \Lsym{,}  \Lmv{y}  \Lsym{:}  \Lnt{B}  \Lsym{,}  \Gamma_{{\mathrm{4}}}  \vdash  \Lnt{t_{{\mathrm{3}}}}  :  \Lnt{C} }
        }{ \Gamma_{{\mathrm{3}}}  \Lsym{,}  \Lmv{z}  \Lsym{:}   \Lnt{A}  \otimes  \Lnt{B}   \Lsym{,}  \Gamma_{{\mathrm{4}}}  \vdash   \mathsf{let}\, \Lmv{z} \,\mathsf{be}\,  \Lmv{x}  \otimes  \Lmv{y}  \,\mathsf{in}\, \Lnt{t_{{\mathrm{3}}}}   :  \Lnt{C} }
      }{ \Gamma_{{\mathrm{3}}}  \Lsym{,}  \Gamma_{{\mathrm{1}}}  \Lsym{,}  \Gamma_{{\mathrm{2}}}  \Lsym{,}  \Gamma_{{\mathrm{4}}}  \vdash  \Lsym{[}   \Lnt{t_{{\mathrm{1}}}}  \otimes  \Lnt{t_{{\mathrm{2}}}}   \Lsym{/}  \Lmv{z}  \Lsym{]}   \mathsf{let}\, \Lmv{z} \,\mathsf{be}\,  \Lmv{x}  \otimes  \Lmv{y}  \,\mathsf{in}\, \Lnt{t_{{\mathrm{3}}}}   :  \Lnt{C} }
    \end{math}
  \end{center}
  is transformed into the proof
  \begin{center}
    \begin{math}
      $$\mprset{flushleft}
      \inferrule* [right=cut] {
        \inferrule* [right=] {
          \inferrule* [right=,vdots=1.5em,fraction=\,] {
            \,
          }{\pi_1}          
        }{ \Gamma_{{\mathrm{1}}}  \vdash  \Lnt{t_{{\mathrm{1}}}}  :  \Lnt{A} }
        \\
        $$\mprset{flushleft}
        \inferrule* [right=cut] {
          \inferrule* [right=] {
            \inferrule* [right=,vdots=1.5em,fraction=\,] {
              \,
            }{\pi_2}          
          }{ \Gamma_{{\mathrm{2}}}  \vdash  \Lnt{t_{{\mathrm{2}}}}  :  \Lnt{B} }
          \\
          \inferrule* [right=] {
            \inferrule* [right=,vdots=1.5em,fraction=\,] {
              \,
            }{\pi_3}          
          }{ \Gamma_{{\mathrm{3}}}  \Lsym{,}  \Lmv{x}  \Lsym{:}  \Lnt{A}  \Lsym{,}  \Lmv{y}  \Lsym{:}  \Lnt{B}  \Lsym{,}  \Gamma_{{\mathrm{4}}}  \vdash  \Lnt{t_{{\mathrm{3}}}}  :  \Lnt{C} }
        }{ \Gamma_{{\mathrm{3}}}  \Lsym{,}  \Lmv{x}  \Lsym{:}  \Lnt{A}  \Lsym{,}  \Gamma_{{\mathrm{2}}}  \Lsym{,}  \Gamma_{{\mathrm{4}}}  \vdash  \Lsym{[}  \Lnt{t_{{\mathrm{2}}}}  \Lsym{/}  \Lmv{y}  \Lsym{]}  \Lnt{t_{{\mathrm{3}}}}  :  \Lnt{C} }
      }{ \Gamma_{{\mathrm{3}}}  \Lsym{,}  \Gamma_{{\mathrm{1}}}  \Lsym{,}  \Gamma_{{\mathrm{2}}}  \Lsym{,}  \Gamma_{{\mathrm{4}}}  \vdash  \Lsym{[}  \Lnt{t_{{\mathrm{1}}}}  \Lsym{/}  \Lmv{x}  \Lsym{]}  \Lsym{[}  \Lnt{t_{{\mathrm{2}}}}  \Lsym{/}  \Lmv{y}  \Lsym{]}  \Lnt{t_{{\mathrm{3}}}}  :  \Lnt{C} }
    \end{math}
  \end{center}

\item[Case:] principal formula vs. principal formula: implication.
  The proof
  \begin{center}
    \scriptsize
    \begin{math}
      $$\mprset{flushleft}
      \inferrule* [right=\scriptsize cut] {
        $$\mprset{flushleft}
        \inferrule* [right=\scriptsize IRr] {
          \inferrule* [right=] {
            \inferrule* [right=,vdots=1.5em,fraction=\,] {
              \,
            }{\pi_1}          
          }{ \Gamma  \Lsym{,}  \Lmv{x}  \Lsym{:}  \Lnt{A}  \vdash  \Lnt{t}  :  \Lnt{B} }
        }{ \Gamma  \vdash   \lambda_r  \Lmv{x} : \Lnt{A} . \Lnt{t}   :   \Lnt{A}  \rightharpoonup  \Lnt{B}  }
        \\
        $$\mprset{flushleft}
        \inferrule* [right=\scriptsize IRl] {
          \inferrule* [right=] {
            \inferrule* [right=,vdots=1.5em,fraction=\,] {
              \,
            }{\pi_2}          
          }{ \Gamma_{{\mathrm{2}}}  \vdash  \Lnt{t_{{\mathrm{1}}}}  :  \Lnt{A} }
          \\
          \inferrule* [right=] {
            \inferrule* [right=,vdots=1.5em,fraction=\,] {
              \,
            }{\pi_3}          
          }{ \Gamma_{{\mathrm{1}}}  \Lsym{,}  \Lmv{y}  \Lsym{:}  \Lnt{B}  \vdash  \Lnt{t_{{\mathrm{3}}}}  :  \Lnt{C} }
        }{ \Gamma_{{\mathrm{1}}}  \Lsym{,}  \Lmv{z}  \Lsym{:}   \Lnt{A}  \rightharpoonup  \Lnt{B}   \Lsym{,}  \Gamma_{{\mathrm{2}}}  \vdash  \Lsym{[}   \mathsf{app}_r\, \Lmv{z} \, \Lnt{t_{{\mathrm{1}}}}   \Lsym{/}  \Lmv{y}  \Lsym{]}  \Lnt{t_{{\mathrm{3}}}}  :  \Lnt{C} }
      }{ \Gamma_{{\mathrm{1}}}  \Lsym{,}  \Gamma  \Lsym{,}  \Gamma_{{\mathrm{2}}}  \vdash  \Lsym{[}   \lambda_r  \Lmv{x} : \Lnt{A} . \Lnt{t}   \Lsym{/}  \Lmv{z}  \Lsym{]}  \Lsym{[}   \mathsf{app}_r\, \Lmv{z} \, \Lnt{t_{{\mathrm{1}}}}   \Lsym{/}  \Lmv{y}  \Lsym{]}  \Lnt{t_{{\mathrm{3}}}}  :  \Lnt{C} }
    \end{math}
  \end{center}
  transforms into the proof
  \begin{center}
    \scriptsize
    \begin{math}
      $$\mprset{flushleft}
      \inferrule* [right=\scriptsize cut] {
        $$\mprset{flushleft}
        \inferrule* [right=\scriptsize cut] {
            \inferrule* [right=] {
              \inferrule* [right=,vdots=1.5em,fraction=\,] {
                \,
              }{\pi_2}          
            }{ \Gamma_{{\mathrm{2}}}  \vdash  \Lnt{t_{{\mathrm{1}}}}  :  \Lnt{A} }
            \\
            \inferrule* [right=] {
              \inferrule* [right=,vdots=1.5em,fraction=\,] {
                \,
              }{\pi_1}          
            }{ \Gamma  \Lsym{,}  \Lmv{x}  \Lsym{:}  \Lnt{A}  \vdash  \Lnt{t}  :  \Lnt{B} }            
          }{ \Gamma  \Lsym{,}  \Gamma_{{\mathrm{2}}}  \vdash  \Lsym{[}  \Lnt{t_{{\mathrm{1}}}}  \Lsym{/}  \Lmv{x}  \Lsym{]}  \Lnt{t}  :  \Lnt{B} }
          \\
          \inferrule* [right=] {
            \inferrule* [right=,vdots=1.5em,fraction=\,] {
              \,
            }{\pi_3}          
          }{ \Gamma_{{\mathrm{1}}}  \Lsym{,}  \Lmv{y}  \Lsym{:}  \Lnt{B}  \vdash  \Lnt{t_{{\mathrm{3}}}}  :  \Lnt{C} }
        }{ \Gamma_{{\mathrm{1}}}  \Lsym{,}  \Gamma  \Lsym{,}  \Gamma_{{\mathrm{2}}}  \vdash  \Lsym{[}  \Lsym{[}  \Lnt{t_{{\mathrm{1}}}}  \Lsym{/}  \Lmv{x}  \Lsym{]}  \Lnt{t}  \Lsym{/}  \Lmv{y}  \Lsym{]}  \Lnt{t_{{\mathrm{3}}}}  :  \Lnt{C} }
    \end{math}
  \end{center}

\item[Case:] principal formula vs. principal formula: implication.
  The proof
  \begin{center}
    \scriptsize
    \begin{math}
      $$\mprset{flushleft}
      \inferrule* [right=\scriptsize cut] {
        $$\mprset{flushleft}
        \inferrule* [right=\scriptsize ILr] {
          \inferrule* [right=] {
            \inferrule* [right=,vdots=1.5em,fraction=\,] {
              \,
            }{\pi_1}          
          }{ \Lmv{x}  \Lsym{:}  \Lnt{A}  \Lsym{,}  \Gamma  \vdash  \Lnt{t}  :  \Lnt{B} }
        }{ \Gamma  \vdash   \lambda_l  \Lmv{x} : \Lnt{A} . \Lnt{t}   :   \Lnt{B}  \leftharpoonup  \Lnt{A}  }
        \\
        $$\mprset{flushleft}
        \inferrule* [right=\scriptsize ILl] {
          \inferrule* [right=] {
            \inferrule* [right=,vdots=1.5em,fraction=\,] {
              \,
            }{\pi_2}          
          }{ \Gamma_{{\mathrm{2}}}  \vdash  \Lnt{t_{{\mathrm{1}}}}  :  \Lnt{A} }
          \\
          \inferrule* [right=] {
            \inferrule* [right=,vdots=1.5em,fraction=\,] {
              \,
            }{\pi_3}          
          }{ \Gamma_{{\mathrm{1}}}  \Lsym{,}  \Lmv{y}  \Lsym{:}  \Lnt{B}  \vdash  \Lnt{t_{{\mathrm{3}}}}  :  \Lnt{C} }
        }{ \Gamma_{{\mathrm{1}}}  \Lsym{,}  \Gamma_{{\mathrm{2}}}  \Lsym{,}  \Lmv{z}  \Lsym{:}   \Lnt{B}  \leftharpoonup  \Lnt{A}   \vdash  \Lsym{[}   \mathsf{app}_l\, \Lmv{z} \, \Lnt{t_{{\mathrm{1}}}}   \Lsym{/}  \Lmv{y}  \Lsym{]}  \Lnt{t_{{\mathrm{3}}}}  :  \Lnt{C} }
      }{ \Gamma_{{\mathrm{1}}}  \Lsym{,}  \Gamma_{{\mathrm{2}}}  \Lsym{,}  \Gamma  \vdash  \Lsym{[}   \lambda_l  \Lmv{x} : \Lnt{A} . \Lnt{t}   \Lsym{/}  \Lmv{z}  \Lsym{]}  \Lsym{[}   \mathsf{app}_l\, \Lmv{z} \, \Lnt{t_{{\mathrm{1}}}}   \Lsym{/}  \Lmv{y}  \Lsym{]}  \Lnt{t_{{\mathrm{3}}}}  :  \Lnt{C} }
    \end{math}
  \end{center}
  transforms into the proof
  \begin{center}
    \scriptsize
    \begin{math}
      $$\mprset{flushleft}
      \inferrule* [right=\scriptsize cut] {
        $$\mprset{flushleft}
        \inferrule* [right=\scriptsize cut] {
            \inferrule* [right=] {
              \inferrule* [right=,vdots=1.5em,fraction=\,] {
                \,
              }{\pi_2}          
            }{ \Gamma_{{\mathrm{2}}}  \vdash  \Lnt{t_{{\mathrm{1}}}}  :  \Lnt{A} }
            \\
            \inferrule* [right=] {
              \inferrule* [right=,vdots=1.5em,fraction=\,] {
                \,
              }{\pi_1}          
            }{ \Gamma  \Lsym{,}  \Lmv{x}  \Lsym{:}  \Lnt{A}  \vdash  \Lnt{t}  :  \Lnt{B} }            
          }{ \Gamma  \Lsym{,}  \Gamma_{{\mathrm{2}}}  \vdash  \Lsym{[}  \Lnt{t_{{\mathrm{1}}}}  \Lsym{/}  \Lmv{x}  \Lsym{]}  \Lnt{t}  :  \Lnt{B} }
          \\
          \inferrule* [right=] {
            \inferrule* [right=,vdots=1.5em,fraction=\,] {
              \,
            }{\pi_3}          
          }{ \Gamma_{{\mathrm{1}}}  \Lsym{,}  \Lmv{y}  \Lsym{:}  \Lnt{B}  \vdash  \Lnt{t_{{\mathrm{3}}}}  :  \Lnt{C} }
        }{ \Gamma_{{\mathrm{1}}}  \Lsym{,}  \Gamma  \Lsym{,}  \Gamma_{{\mathrm{2}}}  \vdash  \Lsym{[}  \Lsym{[}  \Lnt{t_{{\mathrm{1}}}}  \Lsym{/}  \Lmv{x}  \Lsym{]}  \Lnt{t}  \Lsym{/}  \Lmv{y}  \Lsym{]}  \Lnt{t_{{\mathrm{3}}}}  :  \Lnt{C} }
    \end{math}
  \end{center}

\item[Case:] promotion vs dereliction.
  The proof
  \begin{center}
    \scriptsize
    \begin{math}      
      $$\mprset{flushleft}
      \inferrule* [right=cut] {
        $$\mprset{flushleft}
        \inferrule* [right=Br] {
          $$\mprset{flushleft}
          \inferrule* [right=] {
            \inferrule* [right=,vdots=1.5em,fraction=\,] {
              \,
            }{\pi_1}          
          }{  \vec{ \Lmv{x} } : ! \Gamma   \vdash  \Lnt{t_{{\mathrm{1}}}}  :  \Lnt{A} }          
        }{  \vec{ \Lmv{y} } : ! \Gamma   \vdash   \mathsf{promote}_!\,  \vec{ \Lmv{y} }  \,\mathsf{for}\,\vec{ \Lmv{x} }\,\mathsf{in}\, \Lnt{t_{{\mathrm{1}}}}   :   !  \Lnt{A}  }
        \\
        $$\mprset{flushleft}
        \inferrule* [right=Bl] {
          $$\mprset{flushleft}
          \inferrule* [right=] {
            \inferrule* [right=,vdots=1.5em,fraction=\,] {
              \,
            }{\pi_2}          
          }{ \Gamma_{{\mathrm{1}}}  \Lsym{,}  \Lmv{w}  \Lsym{:}  \Lnt{A}  \Lsym{,}  \Gamma_{{\mathrm{2}}}  \vdash  \Lnt{t_{{\mathrm{2}}}}  :  \Lnt{B} }                      
        }{ \Gamma_{{\mathrm{1}}}  \Lsym{,}  \Lmv{z}  \Lsym{:}   !  \Lnt{A}   \Lsym{,}  \Gamma_{{\mathrm{2}}}  \vdash  \Lsym{[}   \mathsf{derelict}_!\, \Lmv{z}   \Lsym{/}  \Lmv{w}  \Lsym{]}  \Lnt{t_{{\mathrm{2}}}}  :  \Lnt{B} }
      }{ \Gamma_{{\mathrm{1}}}  \Lsym{,}   \vec{ \Lmv{y} } : ! \Gamma   \Lsym{,}  \Gamma_{{\mathrm{2}}}  \vdash  \Lsym{[}   \mathsf{promote}_!\,  \vec{ \Lmv{y} }  \,\mathsf{for}\,\vec{ \Lmv{x} }\,\mathsf{in}\, \Lnt{t_{{\mathrm{1}}}}   \Lsym{/}  \Lmv{z}  \Lsym{]}  \Lsym{[}   \mathsf{derelict}_!\, \Lmv{z}   \Lsym{/}  \Lmv{w}  \Lsym{]}  \Lnt{t_{{\mathrm{2}}}}  :  \Lnt{B} }
    \end{math}
  \end{center}
  transforms into the proof
  \begin{center}
    \begin{math}
      $$\mprset{flushleft}
      \inferrule* [right=cut] {        
        $$\mprset{flushleft}
        \inferrule* [right=] {
          \inferrule* [right=,vdots=1.5em,fraction=\,] {
            \,
          }{\pi_1}          
        }{  \vec{ \Lmv{x} } : ! \Gamma   \vdash  \Lnt{t_{{\mathrm{1}}}}  :  \Lnt{A} }          
        \\
        $$\mprset{flushleft}
        \inferrule* [right=] {
          \inferrule* [right=,vdots=1.5em,fraction=\,] {
            \,
          }{\pi_2}          
        }{ \Gamma_{{\mathrm{1}}}  \Lsym{,}  \Lmv{w}  \Lsym{:}  \Lnt{A}  \Lsym{,}  \Gamma_{{\mathrm{2}}}  \vdash  \Lnt{t_{{\mathrm{2}}}}  :  \Lnt{B} }                      
      }{ \Gamma_{{\mathrm{1}}}  \Lsym{,}   \vec{ \Lmv{x} } : ! \Gamma   \Lsym{,}  \Gamma_{{\mathrm{2}}}  \vdash  \Lsym{[}  \Lnt{t_{{\mathrm{1}}}}  \Lsym{/}  \Lmv{w}  \Lsym{]}  \Lnt{t_{{\mathrm{2}}}}  :  \Lnt{B} }
    \end{math}
  \end{center}

\item[Case:] promotion vs weakening.
  The proof
  \begin{center}
    \scriptsize
    \begin{math}      
      $$\mprset{flushleft}
      \inferrule* [right=cut] {
        $$\mprset{flushleft}
        \inferrule* [right=Br] {
          $$\mprset{flushleft}
          \inferrule* [right=] {
            \inferrule* [right=,vdots=1.5em,fraction=\,] {
              \,
            }{\pi_1}          
          }{  \vec{ \Lmv{x} } : ! \Gamma   \vdash  \Lnt{t_{{\mathrm{1}}}}  :  \Lnt{A} }          
        }{  \vec{ \Lmv{y} } : ! \Gamma   \vdash   \mathsf{promote}_!\,  \vec{ \Lmv{y} }  \,\mathsf{for}\,\vec{ \Lmv{x} }\,\mathsf{in}\, \Lnt{t_{{\mathrm{1}}}}   :   !  \Lnt{A}  }
        \\
        $$\mprset{flushleft}
        \inferrule* [right=W] {
          $$\mprset{flushleft}
          \inferrule* [right=] {
            \inferrule* [right=,vdots=1.5em,fraction=\,] {
              \,
            }{\pi_2}          
          }{ \Gamma_{{\mathrm{1}}}  \Lsym{,}  \Gamma_{{\mathrm{2}}}  \vdash  \Lnt{t_{{\mathrm{2}}}}  :  \Lnt{B} }                      
        }{ \Gamma_{{\mathrm{1}}}  \Lsym{,}  \Lmv{z}  \Lsym{:}   !  \Lnt{A}   \Lsym{,}  \Gamma_{{\mathrm{2}}}  \vdash   \mathsf{discard}\, \Lmv{z} \,\mathsf{in}\, \Lnt{t_{{\mathrm{2}}}}   :  \Lnt{B} }
      }{ \Gamma_{{\mathrm{1}}}  \Lsym{,}   \vec{ \Lmv{y} } : ! \Gamma   \Lsym{,}  \Gamma_{{\mathrm{2}}}  \vdash  \Lsym{[}   \mathsf{promote}_!\,  \vec{ \Lmv{y} }  \,\mathsf{for}\,\vec{ \Lmv{x} }\,\mathsf{in}\, \Lnt{t_{{\mathrm{1}}}}   \Lsym{/}  \Lmv{z}  \Lsym{]}   \mathsf{discard}\, \Lmv{z} \,\mathsf{in}\, \Lnt{t_{{\mathrm{2}}}}   :  \Lnt{B} }
    \end{math}
  \end{center}
  transforms into the proof
  \begin{center}
    \begin{math}
      $$\mprset{flushleft}
      \inferrule* [right=\scriptsize Series of Weakening W] {                
        $$\mprset{flushleft}
        \inferrule* [right=] {
          \inferrule* [right=,vdots=1.5em,fraction=\,] {
            \,
          }{\pi_2}          
        }{ \Gamma_{{\mathrm{1}}}  \Lsym{,}  \Gamma_{{\mathrm{2}}}  \vdash  \Lnt{t_{{\mathrm{2}}}}  :  \Lnt{B} }                      
      }{ \Gamma_{{\mathrm{1}}}  \Lsym{,}   \vec{ \Lmv{y} } : ! \Gamma   \Lsym{,}  \Gamma_{{\mathrm{2}}}  \vdash   \mathsf{discard}\,  \vec{ \Lmv{y} }  \,\mathsf{in}\, \Lnt{t_{{\mathrm{2}}}}   :  \Lnt{B} }
    \end{math}
  \end{center}

\item[Case:] promotion vs contraction.
  The proof
  \begin{center}
    \scriptsize
    \begin{math}      
      $$\mprset{flushleft}
      \inferrule* [right=cut] {
        $$\mprset{flushleft}
        \inferrule* [right=Br] {
          $$\mprset{flushleft}
          \inferrule* [right=] {
            \inferrule* [right=,vdots=1.5em,fraction=\,] {
              \,
            }{\pi_1}          
          }{  \vec{ \Lmv{x} } : ! \Gamma   \vdash  \Lnt{t_{{\mathrm{1}}}}  :  \Lnt{A} }          
        }{  \vec{ \Lmv{y} } : ! \Gamma   \vdash   \mathsf{promote}_!\,  \vec{ \Lmv{y} }  \,\mathsf{for}\,\vec{ \Lmv{x} }\,\mathsf{in}\, \Lnt{t_{{\mathrm{1}}}}   :   !  \Lnt{A}  }
        \\
        $$\mprset{flushleft}
        \inferrule* [right=C] {
          $$\mprset{flushleft}
          \inferrule* [right=] {
            \inferrule* [right=,vdots=1.5em,fraction=\,] {
              \,
            }{\pi_2}          
          }{ \Gamma_{{\mathrm{1}}}  \Lsym{,}  \Lmv{w_{{\mathrm{1}}}}  \Lsym{:}   !  \Lnt{A}   \Lsym{,}  \Lmv{w_{{\mathrm{2}}}}  \Lsym{:}   !  \Lnt{A}   \Lsym{,}  \Gamma_{{\mathrm{2}}}  \vdash  \Lnt{t_{{\mathrm{2}}}}  :  \Lnt{B} }                      
        }{ \Gamma_{{\mathrm{1}}}  \Lsym{,}  \Lmv{z}  \Lsym{:}   !  \Lnt{A}   \Lsym{,}  \Gamma_{{\mathrm{2}}}  \vdash   \mathsf{copy}\, \Lmv{z} \,\mathsf{as}\, \Lmv{w_{{\mathrm{1}}}} , \Lmv{w_{{\mathrm{2}}}} \,\mathsf{in}\, \Lnt{t_{{\mathrm{2}}}}   :  \Lnt{B} }
      }{ \Gamma_{{\mathrm{1}}}  \Lsym{,}   \vec{ \Lmv{y} } : ! \Gamma   \Lsym{,}  \Gamma_{{\mathrm{2}}}  \vdash  \Lsym{[}   \mathsf{promote}_!\,  \vec{ \Lmv{y} }  \,\mathsf{for}\,\vec{ \Lmv{x} }\,\mathsf{in}\, \Lnt{t_{{\mathrm{1}}}}   \Lsym{/}  \Lmv{z}  \Lsym{]}   \mathsf{copy}\, \Lmv{z} \,\mathsf{as}\, \Lmv{w_{{\mathrm{1}}}} , \Lmv{w_{{\mathrm{2}}}} \,\mathsf{in}\, \Lnt{t_{{\mathrm{2}}}}   :  \Lnt{B} }
    \end{math}
  \end{center}
  transforms into the proof
  \begin{center}
    \begin{math}      
      $$\mprset{flushleft}
      \inferrule* [right=C] {
        $$\mprset{flushleft}
      \inferrule* [right=\scriptsize cut] {
        $$\mprset{flushleft}
        \inferrule* [right=\scriptsize Br] {
          $$\mprset{flushleft}
          \inferrule* [right=] {
            \inferrule* [right=,vdots=1.5em,fraction=\,] {
              \,
            }{\pi_1}          
          }{  \vec{ \Lmv{x} } : ! \Gamma   \vdash  \Lnt{t_{{\mathrm{1}}}}  :  \Lnt{A} }          
        }{  \vec{ \Lmv{z} } : ! \Gamma   \vdash   \mathsf{promote}_!\,  \vec{ \Lmv{z} }  \,\mathsf{for}\,\vec{ \Lmv{x} }\,\mathsf{in}\, \Lnt{t_{{\mathrm{1}}}}   :   !  \Lnt{A}  }
        \\
        D
      }{ \Gamma_{{\mathrm{1}}}  \Lsym{,}   \vec{ \Lmv{z_{{\mathrm{1}}}} } : ! \Gamma   \Lsym{,}   \vec{ \Lmv{z_{{\mathrm{2}}}} } : ! \Gamma   \Lsym{,}  \Gamma_{{\mathrm{2}}}  \vdash  \Lsym{[}   \mathsf{promote}_!\,  \vec{ \Lmv{z_{{\mathrm{1}}}} }  \,\mathsf{for}\,\vec{ \Lmv{x} }\,\mathsf{in}\, \Lnt{t_{{\mathrm{1}}}}   \Lsym{/}  \Lmv{w_{{\mathrm{1}}}}  \Lsym{]}  \Lsym{[}   \mathsf{promote}_!\,  \vec{ \Lmv{z_{{\mathrm{2}}}} }  \,\mathsf{for}\,\vec{ \Lmv{x} }\,\mathsf{in}\, \Lnt{t_{{\mathrm{1}}}}   \Lsym{/}  \Lmv{w_{{\mathrm{2}}}}  \Lsym{]}  \Lnt{t_{{\mathrm{2}}}}  :  \Lnt{B} }
      }{ \Gamma_{{\mathrm{1}}}  \Lsym{,}   \vec{ \Lmv{y} } : ! \Gamma   \Lsym{,}  \Gamma_{{\mathrm{2}}}  \vdash   \mathsf{copy}\,  \vec{ \Lmv{y} }  \,\mathsf{as}\, \Lmv{z_{{\mathrm{1}}}} , \Lmv{z_{{\mathrm{2}}}} \,\mathsf{in}\, \Lsym{[}   \mathsf{promote}_!\,  \vec{ \Lmv{z_{{\mathrm{1}}}} }  \,\mathsf{for}\,\vec{ \Lmv{x} }\,\mathsf{in}\, \Lnt{t_{{\mathrm{1}}}}   \Lsym{/}  \Lmv{w_{{\mathrm{1}}}}  \Lsym{]}  \Lsym{[}   \mathsf{promote}_!\,  \vec{ \Lmv{z_{{\mathrm{2}}}} }  \,\mathsf{for}\,\vec{ \Lmv{x} }\,\mathsf{in}\, \Lnt{t_{{\mathrm{1}}}}   \Lsym{/}  \Lmv{w_{{\mathrm{2}}}}  \Lsym{]}  \Lnt{t_{{\mathrm{2}}}}   :  \Lnt{B} }
    \end{math}
  \end{center}
  where the definition $D$ is as follows:
  \begin{center}
    \begin{math}
      $$\mprset{flushleft}
        \inferrule* [right=\scriptsize cut] {
          $$\mprset{flushleft}
          \inferrule* [right=\scriptsize Br] {
            $$\mprset{flushleft}
            \inferrule* [right=] {
              \inferrule* [right=,vdots=1.5em,fraction=\,] {
                \,
              }{\pi_1}          
            }{  \vec{ \Lmv{x} } : ! \Gamma   \vdash  \Lnt{t_{{\mathrm{1}}}}  :  \Lnt{A} }          
          }{  \vec{ \Lmv{y} } : ! \Gamma   \vdash   \mathsf{promote}_!\,  \vec{ \Lmv{y} }  \,\mathsf{for}\,\vec{ \Lmv{x} }\,\mathsf{in}\, \Lnt{t_{{\mathrm{1}}}}   :   !  \Lnt{A}  }          
          \\
          $$\mprset{flushleft}
          \inferrule* [right=] {
            \inferrule* [right=,vdots=1.5em,fraction=\,] {
              \,
            }{\pi_2}          
          }{ \Gamma_{{\mathrm{1}}}  \Lsym{,}  \Lmv{w_{{\mathrm{1}}}}  \Lsym{:}   !  \Lnt{A}   \Lsym{,}  \Lmv{w_{{\mathrm{2}}}}  \Lsym{:}   !  \Lnt{A}   \Lsym{,}  \Gamma_{{\mathrm{2}}}  \vdash  \Lnt{t_{{\mathrm{2}}}}  :  \Lnt{B} }                      
        }{ \Gamma_{{\mathrm{1}}}  \Lsym{,}  \Lmv{w_{{\mathrm{1}}}}  \Lsym{:}   !  \Lnt{A}   \Lsym{,}   \vec{ \Lmv{y} } : ! \Gamma   \Lsym{,}  \Gamma_{{\mathrm{2}}}  \vdash  \Lsym{[}   \mathsf{promote}_!\,  \vec{ \Lmv{y} }  \,\mathsf{for}\,\vec{ \Lmv{x} }\,\mathsf{in}\, \Lnt{t_{{\mathrm{1}}}}   \Lsym{/}  \Lmv{w_{{\mathrm{2}}}}  \Lsym{]}  \Lnt{t_{{\mathrm{2}}}}  :  \Lnt{B} }                
    \end{math}
  \end{center}

\item[Case:] secondary conclusion: left introduction of implication.
  The proof 
  \begin{center}
    \small
    \begin{math}
      $$\mprset{flushleft}
      \inferrule* [right=cut] {
        $$\mprset{flushleft}
        \inferrule* [right=IRl] {
          \inferrule* [right=] {
            \inferrule* [right=,vdots=1.5em,fraction=\,] {
              \,
            }{\pi_1}          
          }{ \Gamma  \vdash  \Lnt{t_{{\mathrm{1}}}}  :  \Lnt{A} }
          \\
          \inferrule* [right=] {
            \inferrule* [right=,vdots=1.5em,fraction=\,] {
              \,
            }{\pi_2}          
          }{ \Gamma_{{\mathrm{1}}}  \Lsym{,}  \Lmv{x}  \Lsym{:}  \Lnt{B}  \Lsym{,}  \Gamma_{{\mathrm{2}}}  \vdash  \Lnt{t_{{\mathrm{2}}}}  :  \Lnt{C} }
        }{ \Gamma_{{\mathrm{1}}}  \Lsym{,}  \Lmv{y}  \Lsym{:}   \Lnt{A}  \rightharpoonup  \Lnt{B}   \Lsym{,}  \Gamma  \Lsym{,}  \Gamma_{{\mathrm{2}}}  \vdash  \Lsym{[}   \mathsf{app}_r\, \Lmv{y} \, \Lnt{t_{{\mathrm{1}}}}   \Lsym{/}  \Lmv{x}  \Lsym{]}  \Lnt{t_{{\mathrm{2}}}}  :  \Lnt{C} }
        \\      
        \inferrule* [right=] {
          \inferrule* [right=,vdots=1.5em,fraction=\,] {
            \,
          }{\pi_3}          
        }{ \Gamma_{{\mathrm{3}}}  \Lsym{,}  \Lmv{z}  \Lsym{:}  \Lnt{C}  \Lsym{,}  \Gamma_{{\mathrm{4}}}  \vdash  \Lnt{t_{{\mathrm{3}}}}  :  \Lnt{D} }
      }{ \Gamma_{{\mathrm{3}}}  \Lsym{,}  \Gamma_{{\mathrm{1}}}  \Lsym{,}  \Lmv{y}  \Lsym{:}   \Lnt{A}  \rightharpoonup  \Lnt{B}   \Lsym{,}  \Gamma  \Lsym{,}  \Gamma_{{\mathrm{2}}}  \Lsym{,}  \Gamma_{{\mathrm{4}}}  \vdash  \Lsym{[}  \Lsym{[}   \mathsf{app}_r\, \Lmv{y} \, \Lnt{t_{{\mathrm{1}}}}   \Lsym{/}  \Lmv{x}  \Lsym{]}  \Lnt{t_{{\mathrm{2}}}}  \Lsym{/}  \Lmv{z}  \Lsym{]}  \Lnt{t_{{\mathrm{3}}}}  :  \Lnt{D} }
    \end{math}
  \end{center}
  transforms into the proof
  \begin{center}
    \begin{math}      
      $$\mprset{flushleft}
      \inferrule* [right=IRl] {
        \inferrule* [right=] {
          \inferrule* [right=,vdots=1.5em,fraction=\,] {
            \,
          }{\pi_1}          
        }{ \Gamma  \vdash  \Lnt{t_{{\mathrm{1}}}}  :  \Lnt{A} }
        \\
        $$\mprset{flushleft}
        \inferrule* [right=cut] {
          \inferrule* [right=] {
            \inferrule* [right=,vdots=1.5em,fraction=\,] {
              \,
            }{\pi_2}          
          }{ \Gamma_{{\mathrm{1}}}  \Lsym{,}  \Lmv{x}  \Lsym{:}  \Lnt{B}  \Lsym{,}  \Gamma_{{\mathrm{2}}}  \vdash  \Lnt{t_{{\mathrm{2}}}}  :  \Lnt{C} }
          \\
          \inferrule* [right=] {
            \inferrule* [right=,vdots=1.5em,fraction=\,] {
              \,
            }{\pi_3}          
          }{ \Gamma_{{\mathrm{3}}}  \Lsym{,}  \Lmv{z}  \Lsym{:}  \Lnt{C}  \Lsym{,}  \Gamma_{{\mathrm{4}}}  \vdash  \Lnt{t_{{\mathrm{3}}}}  :  \Lnt{D} }
        }{ \Gamma_{{\mathrm{3}}}  \Lsym{,}  \Gamma_{{\mathrm{1}}}  \Lsym{,}  \Lmv{x}  \Lsym{:}  \Lnt{B}  \Lsym{,}  \Gamma_{{\mathrm{2}}}  \Lsym{,}  \Gamma_{{\mathrm{4}}}  \vdash  \Lsym{[}  \Lnt{t_{{\mathrm{2}}}}  \Lsym{/}  \Lmv{z}  \Lsym{]}  \Lnt{t_{{\mathrm{3}}}}  :  \Lnt{D} }
      }{ \Gamma_{{\mathrm{3}}}  \Lsym{,}  \Gamma_{{\mathrm{1}}}  \Lsym{,}  \Lmv{y}  \Lsym{:}   \Lnt{A}  \rightharpoonup  \Lnt{B}   \Lsym{,}  \Gamma  \Lsym{,}  \Gamma_{{\mathrm{2}}}  \Lsym{,}  \Gamma_{{\mathrm{4}}}  \vdash  \Lsym{[}   \mathsf{app}_r\, \Lmv{y} \, \Lnt{t_{{\mathrm{1}}}}   \Lsym{/}  \Lmv{x}  \Lsym{]}  \Lsym{[}  \Lnt{t_{{\mathrm{2}}}}  \Lsym{/}  \Lmv{z}  \Lsym{]}  \Lnt{t_{{\mathrm{3}}}}  :  \Lnt{D} }
    \end{math}
  \end{center}

\item[Case:] secondary conclusion: left introduction of implication.
  The proof 
  \begin{center}
    \begin{math}
      $$\mprset{flushleft}
      \inferrule* [right=cut] {
        $$\mprset{flushleft}
        \inferrule* [right=ILl] {
          \inferrule* [right=] {
            \inferrule* [right=,vdots=1.5em,fraction=\,] {
              \,
            }{\pi_1}          
          }{ \Gamma  \vdash  \Lnt{t_{{\mathrm{1}}}}  :  \Lnt{A} }
          \\
          \inferrule* [right=] {
            \inferrule* [right=,vdots=1.5em,fraction=\,] {
              \,
            }{\pi_2}          
          }{ \Gamma_{{\mathrm{1}}}  \Lsym{,}  \Lmv{x}  \Lsym{:}  \Lnt{B}  \Lsym{,}  \Gamma_{{\mathrm{2}}}  \vdash  \Lnt{t_{{\mathrm{2}}}}  :  \Lnt{C} }
        }{ \Gamma_{{\mathrm{1}}}  \Lsym{,}  \Gamma  \Lsym{,}  \Lmv{y}  \Lsym{:}   \Lnt{B}  \leftharpoonup  \Lnt{A}   \Lsym{,}  \Gamma_{{\mathrm{2}}}  \vdash  \Lsym{[}   \mathsf{app}_l\, \Lmv{y} \, \Lnt{t_{{\mathrm{1}}}}   \Lsym{/}  \Lmv{x}  \Lsym{]}  \Lnt{t_{{\mathrm{2}}}}  :  \Lnt{C} }
        \\      
        \inferrule* [right=] {
          \inferrule* [right=,vdots=1.5em,fraction=\,] {
            \,
          }{\pi_3}          
        }{ \Gamma_{{\mathrm{3}}}  \Lsym{,}  \Lmv{z}  \Lsym{:}  \Lnt{C}  \Lsym{,}  \Gamma_{{\mathrm{4}}}  \vdash  \Lnt{t_{{\mathrm{3}}}}  :  \Lnt{D} }
      }{ \Gamma_{{\mathrm{3}}}  \Lsym{,}  \Gamma_{{\mathrm{1}}}  \Lsym{,}  \Gamma  \Lsym{,}  \Lmv{y}  \Lsym{:}   \Lnt{B}  \leftharpoonup  \Lnt{A}   \Lsym{,}  \Gamma_{{\mathrm{2}}}  \Lsym{,}  \Gamma_{{\mathrm{4}}}  \vdash  \Lsym{[}  \Lsym{[}   \mathsf{app}_l\, \Lmv{y} \, \Lnt{t_{{\mathrm{1}}}}   \Lsym{/}  \Lmv{x}  \Lsym{]}  \Lnt{t_{{\mathrm{2}}}}  \Lsym{/}  \Lmv{z}  \Lsym{]}  \Lnt{t_{{\mathrm{3}}}}  :  \Lnt{D} }
    \end{math}
  \end{center}
  transforms into the proof
  \begin{center}
    \begin{math}      
      $$\mprset{flushleft}
      \inferrule* [right=ILl] {
        \inferrule* [right=] {
          \inferrule* [right=,vdots=1.5em,fraction=\,] {
            \,
          }{\pi_1}          
        }{ \Gamma  \vdash  \Lnt{t_{{\mathrm{1}}}}  :  \Lnt{A} }
        \\
        $$\mprset{flushleft}
        \inferrule* [right=cut] {
          \inferrule* [right=] {
            \inferrule* [right=,vdots=1.5em,fraction=\,] {
              \,
            }{\pi_2}          
          }{ \Gamma_{{\mathrm{1}}}  \Lsym{,}  \Lmv{x}  \Lsym{:}  \Lnt{B}  \Lsym{,}  \Gamma_{{\mathrm{2}}}  \vdash  \Lnt{t_{{\mathrm{2}}}}  :  \Lnt{C} }
          \\
          \inferrule* [right=] {
            \inferrule* [right=,vdots=1.5em,fraction=\,] {
              \,
            }{\pi_3}          
          }{ \Gamma_{{\mathrm{3}}}  \Lsym{,}  \Lmv{z}  \Lsym{:}  \Lnt{C}  \Lsym{,}  \Gamma_{{\mathrm{4}}}  \vdash  \Lnt{t_{{\mathrm{3}}}}  :  \Lnt{D} }
        }{ \Gamma_{{\mathrm{3}}}  \Lsym{,}  \Gamma_{{\mathrm{1}}}  \Lsym{,}  \Lmv{x}  \Lsym{:}  \Lnt{B}  \Lsym{,}  \Gamma_{{\mathrm{2}}}  \Lsym{,}  \Gamma_{{\mathrm{4}}}  \vdash  \Lsym{[}  \Lnt{t_{{\mathrm{2}}}}  \Lsym{/}  \Lmv{z}  \Lsym{]}  \Lnt{t_{{\mathrm{3}}}}  :  \Lnt{D} }
      }{ \Gamma_{{\mathrm{3}}}  \Lsym{,}  \Gamma_{{\mathrm{1}}}  \Lsym{,}  \Gamma  \Lsym{,}  \Lmv{y}  \Lsym{:}   \Lnt{B}  \leftharpoonup  \Lnt{A}   \Lsym{,}  \Gamma_{{\mathrm{2}}}  \Lsym{,}  \Gamma_{{\mathrm{4}}}  \vdash  \Lsym{[}   \mathsf{app}_l\, \Lmv{y} \, \Lnt{t_{{\mathrm{1}}}}   \Lsym{/}  \Lmv{x}  \Lsym{]}  \Lsym{[}  \Lnt{t_{{\mathrm{2}}}}  \Lsym{/}  \Lmv{z}  \Lsym{]}  \Lnt{t_{{\mathrm{3}}}}  :  \Lnt{D} }
    \end{math}
  \end{center}    

\item[Case:] secondary conclusion: left introduction of tensor.
  The proof 
  \begin{center}
    \begin{math}
      $$\mprset{flushleft}
      \inferrule* [right=cut] {
        $$\mprset{flushleft}
        \inferrule* [right=Tl] {
          \inferrule* [right=] {
            \inferrule* [right=,vdots=1.5em,fraction=\,] {
              \,
            }{\pi_1}          
          }{ \Gamma  \Lsym{,}  \Lmv{x}  \Lsym{:}  \Lnt{A}  \Lsym{,}  \Lmv{y}  \Lsym{:}  \Lnt{B}  \vdash  \Lnt{t}  :  \Lnt{C} }      
        }{ \Gamma  \Lsym{,}  \Lmv{z}  \Lsym{:}   \Lnt{A}  \otimes  \Lnt{B}   \vdash   \mathsf{let}\, \Lmv{z} \,\mathsf{be}\,  \Lmv{x}  \otimes  \Lmv{y}  \,\mathsf{in}\, \Lnt{t}   :  \Lnt{C} }
        \\
        \inferrule* [right=] {
          \inferrule* [right=,vdots=1.5em,fraction=\,] {
            \,
          }{\pi_2}          
        }{ \Gamma_{{\mathrm{1}}}  \Lsym{,}  \Lmv{w}  \Lsym{:}  \Lnt{C}  \Lsym{,}  \Gamma_{{\mathrm{2}}}  \vdash  \Lnt{t'}  :  \Lnt{D} }
      }{ \Gamma_{{\mathrm{1}}}  \Lsym{,}  \Gamma  \Lsym{,}  \Lmv{z}  \Lsym{:}   \Lnt{A}  \otimes  \Lnt{B}   \Lsym{,}  \Gamma_{{\mathrm{2}}}  \vdash  \Lsym{[}   \mathsf{let}\, \Lmv{z} \,\mathsf{be}\,  \Lmv{x}  \otimes  \Lmv{y}  \,\mathsf{in}\, \Lnt{t}   \Lsym{/}  \Lmv{w}  \Lsym{]}  \Lnt{t'}  :  \Lnt{D} }
    \end{math}
  \end{center}
  transforms into the proof
  \begin{center}
    \begin{math}
      $$\mprset{flushleft}
      \inferrule* [right=Tl] {
        $$\mprset{flushleft}
        \inferrule* [right=cut] {
          \inferrule* [right=] {
            \inferrule* [right=,vdots=1.5em,fraction=\,] {
              \,
            }{\pi_1}          
          }{ \Gamma  \Lsym{,}  \Lmv{x}  \Lsym{:}  \Lnt{A}  \Lsym{,}  \Lmv{y}  \Lsym{:}  \Lnt{B}  \vdash  \Lnt{t}  :  \Lnt{C} }      
          \\
          \inferrule* [right=] {
            \inferrule* [right=,vdots=1.5em,fraction=\,] {
              \,
            }{\pi_2}          
          }{ \Gamma_{{\mathrm{1}}}  \Lsym{,}  \Lmv{w}  \Lsym{:}  \Lnt{C}  \Lsym{,}  \Gamma_{{\mathrm{2}}}  \vdash  \Lnt{t'}  :  \Lnt{D} }
        }{ \Gamma_{{\mathrm{1}}}  \Lsym{,}  \Gamma  \Lsym{,}  \Lmv{x}  \Lsym{:}  \Lnt{A}  \Lsym{,}  \Lmv{y}  \Lsym{:}  \Lnt{B}  \Lsym{,}  \Gamma_{{\mathrm{2}}}  \vdash  \Lsym{[}  \Lnt{t}  \Lsym{/}  \Lmv{w}  \Lsym{]}  \Lnt{t'}  :  \Lnt{D} }      
      }{ \Gamma_{{\mathrm{1}}}  \Lsym{,}  \Gamma  \Lsym{,}  \Lmv{z}  \Lsym{:}   \Lnt{A}  \otimes  \Lnt{B}   \Lsym{,}  \Gamma_{{\mathrm{2}}}  \vdash   \mathsf{let}\, \Lmv{z} \,\mathsf{be}\,  \Lmv{x}  \otimes  \Lmv{y}  \,\mathsf{in}\, \Lsym{[}  \Lnt{t}  \Lsym{/}  \Lmv{w}  \Lsym{]}  \Lnt{t'}   :  \Lnt{D} }
    \end{math}
  \end{center}

\item[Case:] secondary conclusion: left introduction of tensor unit.
  The proof
  \begin{center}
    \begin{math}
      $$\mprset{flushleft}
      \inferrule* [right=cut] {
        $$\mprset{flushleft}
        \inferrule* [right=Ul] {
          \inferrule* [right=] {
            \inferrule* [right=,vdots=1.5em,fraction=\,] {
              \,
            }{\pi_1}          
          }{ \Gamma  \vdash  \Lnt{t}  :  \Lnt{C} }      
        }{ \Gamma  \Lsym{,}  \Lmv{x}  \Lsym{:}   I   \vdash   \mathsf{let}\, \Lmv{x} \,\mathsf{be}\,  \mathsf{unit}  \,\mathsf{in}\, \Lnt{t}   :  \Lnt{C} }
        \\
        \inferrule* [right=] {
          \inferrule* [right=,vdots=1.5em,fraction=\,] {
            \,
          }{\pi_2}          
        }{ \Gamma_{{\mathrm{1}}}  \Lsym{,}  \Lmv{w}  \Lsym{:}  \Lnt{C}  \Lsym{,}  \Gamma_{{\mathrm{2}}}  \vdash  \Lnt{t'}  :  \Lnt{D} }
      }{ \Gamma_{{\mathrm{1}}}  \Lsym{,}  \Gamma  \Lsym{,}  \Lmv{x}  \Lsym{:}   I   \Lsym{,}  \Gamma_{{\mathrm{2}}}  \vdash  \Lsym{[}   \mathsf{let}\, \Lmv{x} \,\mathsf{be}\,  \mathsf{unit}  \,\mathsf{in}\, \Lnt{t}   \Lsym{/}  \Lmv{w}  \Lsym{]}  \Lnt{t'}  :  \Lnt{D} }
    \end{math}
  \end{center}
  is transformed into the following:
  \begin{center}
    \begin{math}
      $$\mprset{flushleft}
        \inferrule* [right=Ul] {
          $$\mprset{flushleft}
          \inferrule* [right=cut] {
            \inferrule* [right=] {
              \inferrule* [right=,vdots=1.5em,fraction=\,] {
                \,
              }{\pi_1}          
            }{ \Gamma  \vdash  \Lnt{t}  :  \Lnt{C} }      
            \\
            \inferrule* [right=] {
              \inferrule* [right=,vdots=1.5em,fraction=\,] {
                \,
              }{\pi_2}          
            }{ \Gamma_{{\mathrm{1}}}  \Lsym{,}  \Lmv{w}  \Lsym{:}  \Lnt{C}  \Lsym{,}  \Gamma_{{\mathrm{2}}}  \vdash  \Lnt{t'}  :  \Lnt{D} }  
          }{ \Gamma_{{\mathrm{1}}}  \Lsym{,}  \Gamma  \Lsym{,}  \Gamma_{{\mathrm{2}}}  \vdash  \Lsym{[}  \Lnt{t}  \Lsym{/}  \Lmv{w}  \Lsym{]}  \Lnt{t'}  :  \Lnt{D} }
        }{ \Gamma_{{\mathrm{1}}}  \Lsym{,}  \Gamma  \Lsym{,}  \Lmv{x}  \Lsym{:}   I   \Lsym{,}  \Gamma_{{\mathrm{2}}}  \vdash   \mathsf{let}\, \Lmv{x} \,\mathsf{be}\,  \mathsf{unit}  \,\mathsf{in}\, \Lsym{[}  \Lnt{t}  \Lsym{/}  \Lmv{w}  \Lsym{]}  \Lnt{t'}   :  \Lnt{D} }
    \end{math}
  \end{center}

\item[Case:] secondary hypothesis: left introduction of tensor.
  The proof
  \begin{center}
    \begin{math}
      $$\mprset{flushleft}
      \inferrule* [right=cut] {
        \inferrule* [right=] {
          \inferrule* [right=,vdots=1.5em,fraction=\,] {
            \,
          }{\pi_1}          
        }{ \Gamma  \vdash  \Lnt{t}  :  \Lnt{A} }      
        \\
        $$\mprset{flushleft}
        \inferrule* [right=Tl] {
          \inferrule* [right=] {
            \inferrule* [right=,vdots=1.5em,fraction=\,] {
              \,
            }{\pi_2}          
          }{ \Gamma_{{\mathrm{1}}}  \Lsym{,}  \Lmv{x}  \Lsym{:}  \Lnt{A}  \Lsym{,}  \Gamma_{{\mathrm{2}}}  \Lsym{,}  \Lmv{y}  \Lsym{:}  \Lnt{B}  \Lsym{,}  \Lmv{z}  \Lsym{:}  \Lnt{C}  \Lsym{,}  \Gamma_{{\mathrm{3}}}  \vdash  \Lnt{t_{{\mathrm{1}}}}  :  \Lnt{D} }                  
        }{ \Gamma_{{\mathrm{1}}}  \Lsym{,}  \Lmv{x}  \Lsym{:}  \Lnt{A}  \Lsym{,}  \Gamma_{{\mathrm{2}}}  \Lsym{,}  \Lmv{w}  \Lsym{:}   \Lnt{B}  \otimes  \Lnt{C}   \Lsym{,}  \Gamma_{{\mathrm{3}}}  \vdash   \mathsf{let}\, \Lmv{w} \,\mathsf{be}\,  \Lmv{y}  \otimes  \Lmv{z}  \,\mathsf{in}\, \Lnt{t_{{\mathrm{1}}}}   :  \Lnt{D} }
      }{ \Gamma_{{\mathrm{1}}}  \Lsym{,}  \Gamma  \Lsym{,}  \Gamma_{{\mathrm{2}}}  \Lsym{,}  \Lmv{w}  \Lsym{:}   \Lnt{B}  \otimes  \Lnt{C}   \Lsym{,}  \Gamma_{{\mathrm{3}}}  \vdash  \Lsym{[}  \Lnt{t}  \Lsym{/}  \Lmv{x}  \Lsym{]}   \mathsf{let}\, \Lmv{w} \,\mathsf{be}\,  \Lmv{y}  \otimes  \Lmv{z}  \,\mathsf{in}\, \Lnt{t_{{\mathrm{1}}}}   :  \Lnt{D} }
    \end{math}
  \end{center}
  transforms into the proof
  \begin{center}
    \begin{math}
      $$\mprset{flushleft}
      \inferrule* [right=Tl] {
        $$\mprset{flushleft}
        \inferrule* [right=cut] {
          \inferrule* [right=] {
            \inferrule* [right=,vdots=1.5em,fraction=\,] {
              \,
            }{\pi_1}          
          }{ \Gamma  \vdash  \Lnt{t}  :  \Lnt{A} }      
          \\
          \inferrule* [right=] {
            \inferrule* [right=,vdots=1.5em,fraction=\,] {
              \,
            }{\pi_2}          
          }{ \Gamma_{{\mathrm{1}}}  \Lsym{,}  \Lmv{x}  \Lsym{:}  \Lnt{A}  \Lsym{,}  \Gamma_{{\mathrm{2}}}  \Lsym{,}  \Lmv{y}  \Lsym{:}  \Lnt{B}  \Lsym{,}  \Lmv{z}  \Lsym{:}  \Lnt{C}  \Lsym{,}  \Gamma_{{\mathrm{3}}}  \vdash  \Lnt{t_{{\mathrm{1}}}}  :  \Lnt{D} }                  
        }{ \Gamma_{{\mathrm{1}}}  \Lsym{,}  \Gamma  \Lsym{,}  \Gamma_{{\mathrm{2}}}  \Lsym{,}  \Lmv{y}  \Lsym{:}  \Lnt{B}  \Lsym{,}  \Lmv{z}  \Lsym{:}  \Lnt{C}  \Lsym{,}  \Gamma_{{\mathrm{3}}}  \vdash  \Lsym{[}  \Lnt{t}  \Lsym{/}  \Lmv{x}  \Lsym{]}  \Lnt{t_{{\mathrm{1}}}}  :  \Lnt{D} }
      }{ \Gamma_{{\mathrm{1}}}  \Lsym{,}  \Gamma  \Lsym{,}  \Gamma_{{\mathrm{2}}}  \Lsym{,}  \Lmv{w}  \Lsym{:}   \Lnt{B}  \otimes  \Lnt{C}   \Lsym{,}  \Gamma_{{\mathrm{3}}}  \vdash   \mathsf{let}\, \Lmv{w} \,\mathsf{be}\,  \Lmv{x}  \otimes  \Lmv{y}  \,\mathsf{in}\, \Lsym{[}  \Lnt{t}  \Lsym{/}  \Lmv{x}  \Lsym{]}  \Lnt{t_{{\mathrm{1}}}}   :  \Lnt{D} }
    \end{math}
  \end{center}

\item[Case:] secondary hypothesis: right introduction of tensor (first
  case).
  The proof
  \begin{center}
    \begin{math}
      $$\mprset{flushleft}
      \inferrule* [right=cut] {
        \inferrule* [right=] {
          \inferrule* [right=,vdots=1.5em,fraction=\,] {
            \,
          }{\pi_1}          
        }{ \Gamma  \vdash  \Lnt{t}  :  \Lnt{A} }      
        \\
        $$\mprset{flushleft}
        \inferrule* [right=Tr] {
          \inferrule* [right=] {
            \inferrule* [right=,vdots=1.5em,fraction=\,] {
              \,
            }{\pi_2}          
          }{ \Gamma_{{\mathrm{1}}}  \Lsym{,}  \Lmv{x}  \Lsym{:}  \Lnt{A}  \Lsym{,}  \Gamma_{{\mathrm{2}}}  \vdash  \Lnt{t_{{\mathrm{1}}}}  :  \Lnt{B} }      
          \\
          \inferrule* [right=] {
            \inferrule* [right=,vdots=1.5em,fraction=\,] {
              \,
            }{\pi_3}          
          }{ \Gamma_{{\mathrm{3}}}  \vdash  \Lnt{t_{{\mathrm{2}}}}  :  \Lnt{C} }      
        }{ \Gamma_{{\mathrm{1}}}  \Lsym{,}  \Lmv{x}  \Lsym{:}  \Lnt{A}  \Lsym{,}  \Gamma_{{\mathrm{2}}}  \Lsym{,}  \Gamma_{{\mathrm{3}}}  \vdash   \Lnt{t_{{\mathrm{1}}}}  \otimes  \Lnt{t_{{\mathrm{2}}}}   :   \Lnt{B}  \otimes  \Lnt{C}  }
      }{ \Gamma_{{\mathrm{1}}}  \Lsym{,}  \Gamma  \Lsym{,}  \Gamma_{{\mathrm{2}}}  \Lsym{,}  \Gamma_{{\mathrm{3}}}  \vdash  \Lsym{[}  \Lnt{t}  \Lsym{/}  \Lmv{x}  \Lsym{]}   (   \Lnt{t_{{\mathrm{1}}}}  \otimes  \Lnt{t_{{\mathrm{2}}}}   )   :   \Lnt{B}  \otimes  \Lnt{C}  }
    \end{math}
  \end{center}
  transforms into the proof
  \begin{center}
    \begin{math}
        $$\mprset{flushleft}
        \inferrule* [right=Tr] {
          $$\mprset{flushleft}
          \inferrule* [right=cut] {
            \inferrule* [right=] {
              \inferrule* [right=,vdots=1.5em,fraction=\,] {
                \,
              }{\pi_1}          
            }{ \Gamma  \vdash  \Lnt{t}  :  \Lnt{A} }      
            \\
            \inferrule* [right=] {
              \inferrule* [right=,vdots=1.5em,fraction=\,] {
                \,
              }{\pi_2}          
            }{ \Gamma_{{\mathrm{1}}}  \Lsym{,}  \Lmv{x}  \Lsym{:}  \Lnt{A}  \Lsym{,}  \Gamma_{{\mathrm{2}}}  \vdash  \Lnt{t_{{\mathrm{1}}}}  :  \Lnt{B} }      
          }{ \Gamma_{{\mathrm{1}}}  \Lsym{,}  \Gamma  \Lsym{,}  \Gamma_{{\mathrm{2}}}  \vdash  \Lsym{[}  \Lnt{t}  \Lsym{/}  \Lmv{x}  \Lsym{]}  \Lnt{t_{{\mathrm{1}}}}  :  \Lnt{B} }
          \\
          \inferrule* [right=] {
            \inferrule* [right=,vdots=1.5em,fraction=\,] {
              \,
            }{\pi_3}          
          }{ \Gamma_{{\mathrm{3}}}  \vdash  \Lnt{t_{{\mathrm{2}}}}  :  \Lnt{C} }      
        }{ \Gamma_{{\mathrm{1}}}  \Lsym{,}  \Gamma  \Lsym{,}  \Gamma_{{\mathrm{2}}}  \Lsym{,}  \Gamma_{{\mathrm{3}}}  \vdash    (  \Lsym{[}  \Lnt{t}  \Lsym{/}  \Lmv{x}  \Lsym{]}  \Lnt{t_{{\mathrm{1}}}}  )   \otimes  \Lnt{t_{{\mathrm{2}}}}   :   \Lnt{B}  \otimes  \Lnt{C}  }
    \end{math}
  \end{center}
  
\item[Case:] secondary hypothesis: right introduction of tensor
  (second case).
  The proof
  \begin{center}
    \begin{math}
      $$\mprset{flushleft}
      \inferrule* [right=cut] {
        \inferrule* [right=] {
          \inferrule* [right=,vdots=1.5em,fraction=\,] {
            \,
          }{\pi_1}          
        }{ \Gamma  \vdash  \Lnt{t}  :  \Lnt{A} }      
        \\
        $$\mprset{flushleft}
        \inferrule* [right=Tr] {
          \inferrule* [right=] {
            \inferrule* [right=,vdots=1.5em,fraction=\,] {
              \,
            }{\pi_2}          
          }{ \Gamma_{{\mathrm{1}}}  \vdash  \Lnt{t_{{\mathrm{1}}}}  :  \Lnt{B} }      
          \\
          \inferrule* [right=] {
            \inferrule* [right=,vdots=1.5em,fraction=\,] {
              \,
            }{\pi_3}          
          }{ \Gamma_{{\mathrm{2}}}  \Lsym{,}  \Lmv{x}  \Lsym{:}  \Lnt{A}  \Lsym{,}  \Gamma_{{\mathrm{3}}}  \vdash  \Lnt{t_{{\mathrm{2}}}}  :  \Lnt{C} }      
        }{ \Gamma_{{\mathrm{1}}}  \Lsym{,}  \Gamma_{{\mathrm{2}}}  \Lsym{,}  \Lmv{x}  \Lsym{:}  \Lnt{A}  \Lsym{,}  \Gamma_{{\mathrm{3}}}  \vdash   \Lnt{t_{{\mathrm{1}}}}  \otimes  \Lnt{t_{{\mathrm{2}}}}   :   \Lnt{B}  \otimes  \Lnt{C}  }
      }{ \Gamma_{{\mathrm{1}}}  \Lsym{,}  \Gamma_{{\mathrm{2}}}  \Lsym{,}  \Gamma  \Lsym{,}  \Gamma_{{\mathrm{3}}}  \vdash  \Lsym{[}  \Lnt{t}  \Lsym{/}  \Lmv{x}  \Lsym{]}   (   \Lnt{t_{{\mathrm{1}}}}  \otimes  \Lnt{t_{{\mathrm{2}}}}   )   :   \Lnt{B}  \otimes  \Lnt{C}  }
    \end{math}
  \end{center}
  transforms into the proof
  \begin{center}
    \begin{math}      
      $$\mprset{flushleft}
        \inferrule* [right=Tr] {
          \inferrule* [right=] {
            \inferrule* [right=,vdots=1.5em,fraction=\,] {
              \,
            }{\pi_2}          
          }{ \Gamma_{{\mathrm{1}}}  \vdash  \Lnt{t_{{\mathrm{1}}}}  :  \Lnt{B} }      
          \\
          $$\mprset{flushleft}
          \inferrule* [right=cut] {
            \inferrule* [right=] {
              \inferrule* [right=,vdots=1.5em,fraction=\,] {
                \,
              }{\pi_1}          
            }{ \Gamma  \vdash  \Lnt{t}  :  \Lnt{A} }      
            \\
            \inferrule* [right=] {
              \inferrule* [right=,vdots=1.5em,fraction=\,] {
                \,
              }{\pi_3}          
            }{ \Gamma_{{\mathrm{2}}}  \Lsym{,}  \Lmv{x}  \Lsym{:}  \Lnt{A}  \Lsym{,}  \Gamma_{{\mathrm{3}}}  \vdash  \Lnt{t_{{\mathrm{2}}}}  :  \Lnt{C} }      
          }{ \Gamma_{{\mathrm{2}}}  \Lsym{,}  \Gamma  \Lsym{,}  \Gamma_{{\mathrm{3}}}  \vdash  \Lsym{[}  \Lnt{t}  \Lsym{/}  \Lmv{x}  \Lsym{]}  \Lnt{t_{{\mathrm{2}}}}  :  \Lnt{C} }       
        }{ \Gamma_{{\mathrm{1}}}  \Lsym{,}  \Gamma_{{\mathrm{2}}}  \Lsym{,}  \Gamma  \Lsym{,}  \Gamma_{{\mathrm{3}}}  \vdash   \Lnt{t_{{\mathrm{1}}}}  \otimes   (  \Lsym{[}  \Lnt{t}  \Lsym{/}  \Lmv{x}  \Lsym{]}  \Lnt{t_{{\mathrm{2}}}}  )    :   \Lnt{B}  \otimes  \Lnt{C}  }
    \end{math}
  \end{center}  

\item[Case:] secondary hypothesis: left introduction of implication
  (first case).
  The proof
  \begin{center}
    \begin{math}
      $$\mprset{flushleft}
      \inferrule* [right=cut] {
        \inferrule* [right=] {
          \inferrule* [right=,vdots=1.5em,fraction=\,] {
            \,
          }{\pi_1}          
        }{ \Gamma  \vdash  \Lnt{t}  :  \Lnt{A} }      
        \\
        $$\mprset{flushleft}
        \inferrule* [right=IRl] {
          \inferrule* [right=] {
            \inferrule* [right=,vdots=1.5em,fraction=\,] {
              \,
            }{\pi_2}          
          }{ \Gamma_{{\mathrm{1}}}  \Lsym{,}  \Lmv{x}  \Lsym{:}  \Lnt{A}  \Lsym{,}  \Gamma_{{\mathrm{2}}}  \vdash  \Lnt{t_{{\mathrm{1}}}}  :  \Lnt{B} }      
          \\
          \inferrule* [right=] {
            \inferrule* [right=,vdots=1.5em,fraction=\,] {
              \,
            }{\pi_3}          
          }{ \Gamma_{{\mathrm{3}}}  \Lsym{,}  \Lmv{y}  \Lsym{:}  \Lnt{C}  \vdash  \Lnt{t_{{\mathrm{2}}}}  :  \Lnt{D} }      
        }{ \Gamma_{{\mathrm{3}}}  \Lsym{,}  \Lmv{z}  \Lsym{:}   \Lnt{B}  \rightharpoonup  \Lnt{C}   \Lsym{,}  \Gamma_{{\mathrm{1}}}  \Lsym{,}  \Lmv{x}  \Lsym{:}  \Lnt{A}  \Lsym{,}  \Gamma_{{\mathrm{2}}}  \vdash  \Lsym{[}   \mathsf{app}_r\, \Lmv{z} \, \Lnt{t_{{\mathrm{1}}}}   \Lsym{/}  \Lmv{y}  \Lsym{]}  \Lnt{t_{{\mathrm{2}}}}  :  \Lnt{D} }
      }{ \Gamma_{{\mathrm{3}}}  \Lsym{,}  \Lmv{z}  \Lsym{:}   \Lnt{B}  \rightharpoonup  \Lnt{C}   \Lsym{,}  \Gamma_{{\mathrm{1}}}  \Lsym{,}  \Gamma  \Lsym{,}  \Gamma_{{\mathrm{2}}}  \vdash  \Lsym{[}  \Lnt{t}  \Lsym{/}  \Lmv{x}  \Lsym{]}  \Lsym{[}   \mathsf{app}_r\, \Lmv{z} \, \Lnt{t_{{\mathrm{1}}}}   \Lsym{/}  \Lmv{y}  \Lsym{]}  \Lnt{t_{{\mathrm{2}}}}  :  \Lnt{D} }
    \end{math}
  \end{center}
  transforms into the proof
  \begin{center}
    \begin{math}
      $$\mprset{flushleft}
      \inferrule* [right=IRl] {
        $$\mprset{flushleft}
        \inferrule* [right=cut] {
          \inferrule* [right=] {
            \inferrule* [right=,vdots=1.5em,fraction=\,] {
              \,
            }{\pi_1}          
          }{ \Gamma  \vdash  \Lnt{t}  :  \Lnt{A} }      
          \\
          \inferrule* [right=] {
            \inferrule* [right=,vdots=1.5em,fraction=\,] {
              \,
            }{\pi_2}          
          }{ \Gamma_{{\mathrm{1}}}  \Lsym{,}  \Lmv{x}  \Lsym{:}  \Lnt{A}  \Lsym{,}  \Gamma_{{\mathrm{2}}}  \vdash  \Lnt{t_{{\mathrm{1}}}}  :  \Lnt{B} }      
        }{ \Gamma_{{\mathrm{1}}}  \Lsym{,}  \Gamma  \Lsym{,}  \Gamma_{{\mathrm{2}}}  \vdash  \Lsym{[}  \Lnt{t}  \Lsym{/}  \Lmv{x}  \Lsym{]}  \Lnt{t_{{\mathrm{1}}}}  :  \Lnt{B} }
        \\
        \inferrule* [right=] {
          \inferrule* [right=,vdots=1.5em,fraction=\,] {
            \,
          }{\pi_3}          
        }{ \Gamma_{{\mathrm{3}}}  \Lsym{,}  \Lmv{y}  \Lsym{:}  \Lnt{C}  \vdash  \Lnt{t_{{\mathrm{2}}}}  :  \Lnt{D} }      
      }{ \Gamma_{{\mathrm{3}}}  \Lsym{,}  \Lmv{z}  \Lsym{:}   \Lnt{B}  \rightharpoonup  \Lnt{C}   \Lsym{,}  \Gamma_{{\mathrm{1}}}  \Lsym{,}  \Gamma  \Lsym{,}  \Gamma_{{\mathrm{2}}}  \vdash  \Lsym{[}   \mathsf{app}_r\, \Lmv{z} \, \Lsym{[}  \Lnt{t}  \Lsym{/}  \Lmv{x}  \Lsym{]}  \Lnt{t_{{\mathrm{1}}}}   \Lsym{/}  \Lmv{y}  \Lsym{]}  \Lnt{t_{{\mathrm{2}}}}  :  \Lnt{D} }
    \end{math}
  \end{center}

\item[Case:] secondary hypothesis: left introduction of implication
  (first case).
  The proof
  \begin{center}
    \begin{math}
      $$\mprset{flushleft}
      \inferrule* [right=cut] {
        \inferrule* [right=] {
          \inferrule* [right=,vdots=1.5em,fraction=\,] {
            \,
          }{\pi_1}          
        }{ \Gamma  \vdash  \Lnt{t}  :  \Lnt{A} }      
        \\
        $$\mprset{flushleft}
        \inferrule* [right=ILl] {
          \inferrule* [right=] {
            \inferrule* [right=,vdots=1.5em,fraction=\,] {
              \,
            }{\pi_2}          
          }{ \Gamma_{{\mathrm{1}}}  \Lsym{,}  \Lmv{x}  \Lsym{:}  \Lnt{A}  \Lsym{,}  \Gamma_{{\mathrm{2}}}  \vdash  \Lnt{t_{{\mathrm{1}}}}  :  \Lnt{B} }      
          \\
          \inferrule* [right=] {
            \inferrule* [right=,vdots=1.5em,fraction=\,] {
              \,
            }{\pi_3}          
          }{ \Gamma_{{\mathrm{3}}}  \Lsym{,}  \Lmv{y}  \Lsym{:}  \Lnt{C}  \vdash  \Lnt{t_{{\mathrm{2}}}}  :  \Lnt{D} }      
        }{ \Gamma_{{\mathrm{3}}}  \Lsym{,}  \Gamma_{{\mathrm{1}}}  \Lsym{,}  \Lmv{x}  \Lsym{:}  \Lnt{A}  \Lsym{,}  \Gamma_{{\mathrm{2}}}  \Lsym{,}  \Lmv{z}  \Lsym{:}   \Lnt{C}  \leftharpoonup  \Lnt{B}   \vdash  \Lsym{[}   \mathsf{app}_l\, \Lmv{z} \, \Lnt{t_{{\mathrm{1}}}}   \Lsym{/}  \Lmv{y}  \Lsym{]}  \Lnt{t_{{\mathrm{2}}}}  :  \Lnt{D} }
      }{ \Gamma_{{\mathrm{3}}}  \Lsym{,}  \Gamma_{{\mathrm{1}}}  \Lsym{,}  \Gamma  \Lsym{,}  \Gamma_{{\mathrm{2}}}  \Lsym{,}  \Lmv{z}  \Lsym{:}   \Lnt{C}  \leftharpoonup  \Lnt{B}   \vdash  \Lsym{[}  \Lnt{t}  \Lsym{/}  \Lmv{x}  \Lsym{]}  \Lsym{[}   \mathsf{app}_l\, \Lmv{z} \, \Lnt{t_{{\mathrm{1}}}}   \Lsym{/}  \Lmv{y}  \Lsym{]}  \Lnt{t_{{\mathrm{2}}}}  :  \Lnt{D} }
    \end{math}
  \end{center}
  transforms into the proof
  \begin{center}
    \begin{math}
      $$\mprset{flushleft}
      \inferrule* [right=ILl] {
        $$\mprset{flushleft}
        \inferrule* [right=cut] {
          \inferrule* [right=] {
            \inferrule* [right=,vdots=1.5em,fraction=\,] {
              \,
            }{\pi_1}          
          }{ \Gamma  \vdash  \Lnt{t}  :  \Lnt{A} }      
          \\
          \inferrule* [right=] {
            \inferrule* [right=,vdots=1.5em,fraction=\,] {
              \,
            }{\pi_2}          
          }{ \Gamma_{{\mathrm{1}}}  \Lsym{,}  \Lmv{x}  \Lsym{:}  \Lnt{A}  \Lsym{,}  \Gamma_{{\mathrm{2}}}  \vdash  \Lnt{t_{{\mathrm{1}}}}  :  \Lnt{B} }      
        }{ \Gamma_{{\mathrm{1}}}  \Lsym{,}  \Gamma  \Lsym{,}  \Gamma_{{\mathrm{2}}}  \vdash  \Lsym{[}  \Lnt{t}  \Lsym{/}  \Lmv{x}  \Lsym{]}  \Lnt{t_{{\mathrm{1}}}}  :  \Lnt{B} }
        \\
        \inferrule* [right=] {
          \inferrule* [right=,vdots=1.5em,fraction=\,] {
            \,
          }{\pi_3}          
        }{ \Gamma_{{\mathrm{3}}}  \Lsym{,}  \Lmv{y}  \Lsym{:}  \Lnt{C}  \vdash  \Lnt{t_{{\mathrm{2}}}}  :  \Lnt{D} }      
      }{ \Gamma_{{\mathrm{3}}}  \Lsym{,}  \Gamma_{{\mathrm{1}}}  \Lsym{,}  \Gamma  \Lsym{,}  \Gamma_{{\mathrm{2}}}  \Lsym{,}  \Lmv{z}  \Lsym{:}   \Lnt{C}  \leftharpoonup  \Lnt{B}   \vdash  \Lsym{[}   \mathsf{app}_l\, \Lmv{z} \, \Lsym{[}  \Lnt{t}  \Lsym{/}  \Lmv{x}  \Lsym{]}  \Lnt{t_{{\mathrm{1}}}}   \Lsym{/}  \Lmv{y}  \Lsym{]}  \Lnt{t_{{\mathrm{2}}}}  :  \Lnt{D} }
    \end{math}
  \end{center}  

\item[Case:] secondary hypothesis: left introduction of implication
  (second case).
  The proof
  \begin{center}
    \begin{math}
      $$\mprset{flushleft}
      \inferrule* [right=cut] {
        \inferrule* [right=] {
          \inferrule* [right=,vdots=1.5em,fraction=\,] {
            \,
          }{\pi_1}          
        }{ \Gamma  \vdash  \Lnt{t}  :  \Lnt{A} }      
        \\
        $$\mprset{flushleft}
        \inferrule* [right=IRl] {
          \inferrule* [right=] {
            \inferrule* [right=,vdots=1.5em,fraction=\,] {
              \,
            }{\pi_2}          
          }{ \Gamma_{{\mathrm{1}}}  \vdash  \Lnt{t_{{\mathrm{1}}}}  :  \Lnt{B} }      
          \\
          \inferrule* [right=] {
            \inferrule* [right=,vdots=1.5em,fraction=\,] {
              \,
            }{\pi_3}          
          }{ \Gamma_{{\mathrm{2}}}  \Lsym{,}  \Lmv{x}  \Lsym{:}  \Lnt{A}  \Lsym{,}  \Gamma_{{\mathrm{3}}}  \Lsym{,}  \Lmv{y}  \Lsym{:}  \Lnt{C}  \vdash  \Lnt{t_{{\mathrm{2}}}}  :  \Lnt{D} }      
        }{ \Gamma_{{\mathrm{2}}}  \Lsym{,}  \Lmv{x}  \Lsym{:}  \Lnt{A}  \Lsym{,}  \Gamma_{{\mathrm{3}}}  \Lsym{,}  \Lmv{z}  \Lsym{:}   \Lnt{B}  \rightharpoonup  \Lnt{C}   \Lsym{,}  \Gamma_{{\mathrm{1}}}  \vdash  \Lsym{[}   \mathsf{app}_r\, \Lmv{z} \, \Lnt{t_{{\mathrm{1}}}}   \Lsym{/}  \Lmv{y}  \Lsym{]}  \Lnt{t_{{\mathrm{2}}}}  :  \Lnt{D} }
      }{ \Gamma_{{\mathrm{2}}}  \Lsym{,}  \Gamma  \Lsym{,}  \Gamma_{{\mathrm{3}}}  \Lsym{,}  \Lmv{z}  \Lsym{:}   \Lnt{B}  \rightharpoonup  \Lnt{C}   \Lsym{,}  \Gamma_{{\mathrm{1}}}  \vdash  \Lsym{[}  \Lnt{t}  \Lsym{/}  \Lmv{x}  \Lsym{]}  \Lsym{[}   \mathsf{app}_r\, \Lmv{z} \, \Lnt{t_{{\mathrm{1}}}}   \Lsym{/}  \Lmv{y}  \Lsym{]}  \Lnt{t_{{\mathrm{2}}}}  :  \Lnt{D} }
    \end{math}
  \end{center}
  transforms into the proof
  \begin{center}
    \begin{math}      
        $$\mprset{flushleft}
        \inferrule* [right=IRl] {
          \inferrule* [right=] {
            \inferrule* [right=,vdots=1.5em,fraction=\,] {
              \,
            }{\pi_2}          
          }{ \Gamma_{{\mathrm{1}}}  \vdash  \Lnt{t_{{\mathrm{1}}}}  :  \Lnt{B} }      
          \\
          $$\mprset{flushleft}
          \inferrule* [right=cut] {
            \inferrule* [right=] {
              \inferrule* [right=,vdots=1.5em,fraction=\,] {
                \,
              }{\pi_1}          
            }{ \Gamma  \vdash  \Lnt{t}  :  \Lnt{A} }      
            \\
            \inferrule* [right=] {
              \inferrule* [right=,vdots=1.5em,fraction=\,] {
                \,
              }{\pi_3}          
            }{ \Gamma_{{\mathrm{2}}}  \Lsym{,}  \Lmv{x}  \Lsym{:}  \Lnt{A}  \Lsym{,}  \Gamma_{{\mathrm{3}}}  \Lsym{,}  \Lmv{y}  \Lsym{:}  \Lnt{C}  \vdash  \Lnt{t_{{\mathrm{2}}}}  :  \Lnt{D} }      
          }{ \Gamma_{{\mathrm{2}}}  \Lsym{,}  \Gamma  \Lsym{,}  \Gamma_{{\mathrm{3}}}  \Lsym{,}  \Lmv{y}  \Lsym{:}  \Lnt{C}  \vdash  \Lsym{[}  \Lnt{t}  \Lsym{/}  \Lmv{x}  \Lsym{]}  \Lnt{t_{{\mathrm{2}}}}  :  \Lnt{D} }
        }{ \Gamma_{{\mathrm{2}}}  \Lsym{,}  \Gamma  \Lsym{,}  \Gamma_{{\mathrm{3}}}  \Lsym{,}  \Lmv{z}  \Lsym{:}   \Lnt{B}  \rightharpoonup  \Lnt{C}   \Lsym{,}  \Gamma_{{\mathrm{1}}}  \vdash  \Lsym{[}   \mathsf{app}_r\, \Lmv{z} \, \Lnt{t_{{\mathrm{1}}}}   \Lsym{/}  \Lmv{y}  \Lsym{]}  \Lsym{[}  \Lnt{t}  \Lsym{/}  \Lmv{x}  \Lsym{]}  \Lnt{t_{{\mathrm{2}}}}  :  \Lnt{D} }
    \end{math}
  \end{center}

\item[Case:] secondary hypothesis: left introduction of implication
  (second case).
  The proof
  \begin{center}
    \begin{math}
      $$\mprset{flushleft}
      \inferrule* [right=cut] {
        \inferrule* [right=] {
          \inferrule* [right=,vdots=1.5em,fraction=\,] {
            \,
          }{\pi_1}          
        }{ \Gamma  \vdash  \Lnt{t}  :  \Lnt{A} }      
        \\
        $$\mprset{flushleft}
        \inferrule* [right=ILl] {
          \inferrule* [right=] {
            \inferrule* [right=,vdots=1.5em,fraction=\,] {
              \,
            }{\pi_2}          
          }{ \Gamma_{{\mathrm{1}}}  \vdash  \Lnt{t_{{\mathrm{1}}}}  :  \Lnt{B} }      
          \\
          \inferrule* [right=] {
            \inferrule* [right=,vdots=1.5em,fraction=\,] {
              \,
            }{\pi_3}          
          }{ \Gamma_{{\mathrm{2}}}  \Lsym{,}  \Lmv{x}  \Lsym{:}  \Lnt{A}  \Lsym{,}  \Gamma_{{\mathrm{3}}}  \Lsym{,}  \Lmv{y}  \Lsym{:}  \Lnt{C}  \vdash  \Lnt{t_{{\mathrm{2}}}}  :  \Lnt{D} }      
        }{ \Gamma_{{\mathrm{2}}}  \Lsym{,}  \Lmv{x}  \Lsym{:}  \Lnt{A}  \Lsym{,}  \Gamma_{{\mathrm{3}}}  \Lsym{,}  \Gamma_{{\mathrm{1}}}  \Lsym{,}  \Lmv{z}  \Lsym{:}   \Lnt{C}  \leftharpoonup  \Lnt{B}   \vdash  \Lsym{[}   \mathsf{app}_l\, \Lmv{z} \, \Lnt{t_{{\mathrm{1}}}}   \Lsym{/}  \Lmv{y}  \Lsym{]}  \Lnt{t_{{\mathrm{2}}}}  :  \Lnt{D} }
      }{ \Gamma_{{\mathrm{2}}}  \Lsym{,}  \Gamma  \Lsym{,}  \Gamma_{{\mathrm{3}}}  \Lsym{,}  \Gamma_{{\mathrm{1}}}  \Lsym{,}  \Lmv{z}  \Lsym{:}   \Lnt{C}  \leftharpoonup  \Lnt{B}   \vdash  \Lsym{[}  \Lnt{t}  \Lsym{/}  \Lmv{x}  \Lsym{]}  \Lsym{[}   \mathsf{app}_l\, \Lmv{z} \, \Lnt{t_{{\mathrm{1}}}}   \Lsym{/}  \Lmv{y}  \Lsym{]}  \Lnt{t_{{\mathrm{2}}}}  :  \Lnt{D} }
    \end{math}
  \end{center}
  transforms into the proof
  \begin{center}
    \begin{math}      
        $$\mprset{flushleft}
        \inferrule* [right=ILl] {
          \inferrule* [right=] {
            \inferrule* [right=,vdots=1.5em,fraction=\,] {
              \,
            }{\pi_2}          
          }{ \Gamma_{{\mathrm{1}}}  \vdash  \Lnt{t_{{\mathrm{1}}}}  :  \Lnt{B} }      
          \\
          $$\mprset{flushleft}
          \inferrule* [right=cut] {
            \inferrule* [right=] {
              \inferrule* [right=,vdots=1.5em,fraction=\,] {
                \,
              }{\pi_1}          
            }{ \Gamma  \vdash  \Lnt{t}  :  \Lnt{A} }      
            \\
            \inferrule* [right=] {
              \inferrule* [right=,vdots=1.5em,fraction=\,] {
                \,
              }{\pi_3}          
            }{ \Gamma_{{\mathrm{2}}}  \Lsym{,}  \Lmv{x}  \Lsym{:}  \Lnt{A}  \Lsym{,}  \Gamma_{{\mathrm{3}}}  \Lsym{,}  \Lmv{y}  \Lsym{:}  \Lnt{C}  \vdash  \Lnt{t_{{\mathrm{2}}}}  :  \Lnt{D} }      
          }{ \Gamma_{{\mathrm{2}}}  \Lsym{,}  \Gamma  \Lsym{,}  \Gamma_{{\mathrm{3}}}  \Lsym{,}  \Lmv{y}  \Lsym{:}  \Lnt{C}  \vdash  \Lsym{[}  \Lnt{t}  \Lsym{/}  \Lmv{x}  \Lsym{]}  \Lnt{t_{{\mathrm{2}}}}  :  \Lnt{D} }
        }{ \Gamma_{{\mathrm{2}}}  \Lsym{,}  \Gamma  \Lsym{,}  \Gamma_{{\mathrm{3}}}  \Lsym{,}  \Gamma_{{\mathrm{1}}}  \Lsym{,}  \Lmv{z}  \Lsym{:}   \Lnt{C}  \leftharpoonup  \Lnt{B}   \vdash  \Lsym{[}   \mathsf{app}_l\, \Lmv{z} \, \Lnt{t_{{\mathrm{1}}}}   \Lsym{/}  \Lmv{y}  \Lsym{]}  \Lsym{[}  \Lnt{t}  \Lsym{/}  \Lmv{x}  \Lsym{]}  \Lnt{t_{{\mathrm{2}}}}  :  \Lnt{D} }
    \end{math}
  \end{center}  

\item[Case:] secondary hypothesis: left introduction of implication
  (third case).
  The proof
  \begin{center}
    \begin{math}
      $$\mprset{flushleft}
      \inferrule* [right=cut] {
        \inferrule* [right=] {
          \inferrule* [right=,vdots=1.5em,fraction=\,] {
            \,
          }{\pi_1}          
        }{ \Gamma  \vdash  \Lnt{t}  :  \Lnt{A} }      
        \\
        $$\mprset{flushleft}
        \inferrule* [right=IRl] {
          \inferrule* [right=] {
            \inferrule* [right=,vdots=1.5em,fraction=\,] {
              \,
            }{\pi_2}          
          }{ \Gamma_{{\mathrm{1}}}  \vdash  \Lnt{t_{{\mathrm{1}}}}  :  \Lnt{B} }      
          \\
          \inferrule* [right=] {
            \inferrule* [right=,vdots=1.5em,fraction=\,] {
              \,
            }{\pi_3}          
          }{ \Gamma_{{\mathrm{2}}}  \Lsym{,}  \Lmv{y}  \Lsym{:}  \Lnt{C}  \Lsym{,}  \Gamma_{{\mathrm{3}}}  \Lsym{,}  \Lmv{x}  \Lsym{:}  \Lnt{A}  \vdash  \Lnt{t_{{\mathrm{2}}}}  :  \Lnt{D} }      
        }{ \Gamma_{{\mathrm{2}}}  \Lsym{,}  \Lmv{z}  \Lsym{:}   \Lnt{B}  \rightharpoonup  \Lnt{C}   \Lsym{,}  \Gamma_{{\mathrm{1}}}  \Lsym{,}  \Gamma_{{\mathrm{3}}}  \Lsym{,}  \Lmv{x}  \Lsym{:}  \Lnt{A}  \vdash  \Lsym{[}   \mathsf{app}_r\, \Lmv{z} \, \Lnt{t_{{\mathrm{1}}}}   \Lsym{/}  \Lmv{y}  \Lsym{]}  \Lnt{t_{{\mathrm{2}}}}  :  \Lnt{D} }
      }{ \Gamma_{{\mathrm{2}}}  \Lsym{,}  \Lmv{z}  \Lsym{:}   \Lnt{B}  \rightharpoonup  \Lnt{C}   \Lsym{,}  \Gamma_{{\mathrm{1}}}  \Lsym{,}  \Gamma_{{\mathrm{3}}}  \Lsym{,}  \Gamma  \vdash  \Lsym{[}  \Lnt{t}  \Lsym{/}  \Lmv{x}  \Lsym{]}  \Lsym{[}   \mathsf{app}_r\, \Lmv{z} \, \Lnt{t_{{\mathrm{1}}}}   \Lsym{/}  \Lmv{y}  \Lsym{]}  \Lnt{t_{{\mathrm{2}}}}  :  \Lnt{D} }
    \end{math}
  \end{center}
  transforms into the proof
  \begin{center}
    \begin{math}
        $$\mprset{flushleft}
        \inferrule* [right=IRl] {
          \inferrule* [right=] {
            \inferrule* [right=,vdots=1.5em,fraction=\,] {
              \,
            }{\pi_2}          
          }{ \Gamma_{{\mathrm{1}}}  \vdash  \Lnt{t_{{\mathrm{1}}}}  :  \Lnt{B} }      
          \\
          $$\mprset{flushleft}
          \inferrule* [right=cut] {
            \inferrule* [right=] {
              \inferrule* [right=,vdots=1.5em,fraction=\,] {
                \,
              }{\pi_1}          
            }{ \Gamma  \vdash  \Lnt{t}  :  \Lnt{A} }      
            \\
            \inferrule* [right=] {
              \inferrule* [right=,vdots=1.5em,fraction=\,] {
                \,
              }{\pi_3}          
            }{ \Gamma_{{\mathrm{2}}}  \Lsym{,}  \Lmv{y}  \Lsym{:}  \Lnt{C}  \Lsym{,}  \Gamma_{{\mathrm{3}}}  \Lsym{,}  \Lmv{x}  \Lsym{:}  \Lnt{A}  \vdash  \Lnt{t_{{\mathrm{2}}}}  :  \Lnt{D} }      
          }{ \Gamma_{{\mathrm{2}}}  \Lsym{,}  \Lmv{y}  \Lsym{:}  \Lnt{C}  \Lsym{,}  \Gamma_{{\mathrm{3}}}  \Lsym{,}  \Gamma  \vdash  \Lsym{[}  \Lnt{t}  \Lsym{/}  \Lmv{x}  \Lsym{]}  \Lnt{t_{{\mathrm{2}}}}  :  \Lnt{D} }
        }{ \Gamma_{{\mathrm{2}}}  \Lsym{,}  \Lmv{z}  \Lsym{:}   \Lnt{B}  \rightharpoonup  \Lnt{C}   \Lsym{,}  \Gamma_{{\mathrm{1}}}  \Lsym{,}  \Gamma_{{\mathrm{3}}}  \Lsym{,}  \Gamma  \vdash  \Lsym{[}   \mathsf{app}_r\, \Lmv{z} \, \Lnt{t_{{\mathrm{1}}}}   \Lsym{/}  \Lmv{y}  \Lsym{]}  \Lsym{[}  \Lnt{t}  \Lsym{/}  \Lmv{x}  \Lsym{]}  \Lnt{t_{{\mathrm{2}}}}  :  \Lnt{D} }
    \end{math}
  \end{center}

\item[Case:] secondary hypothesis: left introduction of implication
  (third case).
  The proof
  \begin{center}
    \begin{math}
      $$\mprset{flushleft}
      \inferrule* [right=cut] {
        \inferrule* [right=] {
          \inferrule* [right=,vdots=1.5em,fraction=\,] {
            \,
          }{\pi_1}          
        }{ \Gamma  \vdash  \Lnt{t}  :  \Lnt{A} }      
        \\
        $$\mprset{flushleft}
        \inferrule* [right=ILl] {
          \inferrule* [right=] {
            \inferrule* [right=,vdots=1.5em,fraction=\,] {
              \,
            }{\pi_2}          
          }{ \Gamma_{{\mathrm{1}}}  \vdash  \Lnt{t_{{\mathrm{1}}}}  :  \Lnt{B} }      
          \\
          \inferrule* [right=] {
            \inferrule* [right=,vdots=1.5em,fraction=\,] {
              \,
            }{\pi_3}          
          }{ \Gamma_{{\mathrm{2}}}  \Lsym{,}  \Lmv{y}  \Lsym{:}  \Lnt{C}  \Lsym{,}  \Gamma_{{\mathrm{3}}}  \Lsym{,}  \Lmv{x}  \Lsym{:}  \Lnt{A}  \vdash  \Lnt{t_{{\mathrm{2}}}}  :  \Lnt{D} }      
        }{ \Gamma_{{\mathrm{2}}}  \Lsym{,}  \Gamma_{{\mathrm{1}}}  \Lsym{,}  \Lmv{z}  \Lsym{:}   \Lnt{C}  \leftharpoonup  \Lnt{B}   \Lsym{,}  \Gamma_{{\mathrm{3}}}  \Lsym{,}  \Lmv{x}  \Lsym{:}  \Lnt{A}  \vdash  \Lsym{[}   \mathsf{app}_l\, \Lmv{z} \, \Lnt{t_{{\mathrm{1}}}}   \Lsym{/}  \Lmv{y}  \Lsym{]}  \Lnt{t_{{\mathrm{2}}}}  :  \Lnt{D} }
      }{ \Gamma_{{\mathrm{2}}}  \Lsym{,}  \Gamma_{{\mathrm{1}}}  \Lsym{,}  \Lmv{z}  \Lsym{:}   \Lnt{C}  \leftharpoonup  \Lnt{B}   \Lsym{,}  \Gamma_{{\mathrm{3}}}  \Lsym{,}  \Gamma  \vdash  \Lsym{[}  \Lnt{t}  \Lsym{/}  \Lmv{x}  \Lsym{]}  \Lsym{[}   \mathsf{app}_l\, \Lmv{z} \, \Lnt{t_{{\mathrm{1}}}}   \Lsym{/}  \Lmv{y}  \Lsym{]}  \Lnt{t_{{\mathrm{2}}}}  :  \Lnt{D} }
    \end{math}
  \end{center}
  transforms into the proof
  \begin{center}
    \begin{math}
        $$\mprset{flushleft}
        \inferrule* [right=ILl] {
          \inferrule* [right=] {
            \inferrule* [right=,vdots=1.5em,fraction=\,] {
              \,
            }{\pi_2}          
          }{ \Gamma_{{\mathrm{1}}}  \vdash  \Lnt{t_{{\mathrm{1}}}}  :  \Lnt{B} }      
          \\
          $$\mprset{flushleft}
          \inferrule* [right=cut] {
            \inferrule* [right=] {
              \inferrule* [right=,vdots=1.5em,fraction=\,] {
                \,
              }{\pi_1}          
            }{ \Gamma  \vdash  \Lnt{t}  :  \Lnt{A} }      
            \\
            \inferrule* [right=] {
              \inferrule* [right=,vdots=1.5em,fraction=\,] {
                \,
              }{\pi_3}          
            }{ \Gamma_{{\mathrm{2}}}  \Lsym{,}  \Lmv{y}  \Lsym{:}  \Lnt{C}  \Lsym{,}  \Gamma_{{\mathrm{3}}}  \Lsym{,}  \Lmv{x}  \Lsym{:}  \Lnt{A}  \vdash  \Lnt{t_{{\mathrm{2}}}}  :  \Lnt{D} }      
          }{ \Gamma_{{\mathrm{2}}}  \Lsym{,}  \Lmv{y}  \Lsym{:}  \Lnt{C}  \Lsym{,}  \Gamma_{{\mathrm{3}}}  \Lsym{,}  \Gamma  \vdash  \Lsym{[}  \Lnt{t}  \Lsym{/}  \Lmv{x}  \Lsym{]}  \Lnt{t_{{\mathrm{2}}}}  :  \Lnt{D} }
        }{ \Gamma_{{\mathrm{2}}}  \Lsym{,}  \Gamma_{{\mathrm{1}}}  \Lsym{,}  \Lmv{z}  \Lsym{:}   \Lnt{C}  \leftharpoonup  \Lnt{B}   \Lsym{,}  \Gamma_{{\mathrm{3}}}  \Lsym{,}  \Gamma  \vdash  \Lsym{[}   \mathsf{app}_l\, \Lmv{z} \, \Lnt{t_{{\mathrm{1}}}}   \Lsym{/}  \Lmv{y}  \Lsym{]}  \Lsym{[}  \Lnt{t}  \Lsym{/}  \Lmv{x}  \Lsym{]}  \Lnt{t_{{\mathrm{2}}}}  :  \Lnt{D} }
    \end{math}
  \end{center}  

\item[Case:] secondary hypothesis: right introduction of implication.
  The proof
  \begin{center}
    \begin{math}
      $$\mprset{flushleft}
      \inferrule* [right=cut] {
        \inferrule* [right=] {
          \inferrule* [right=,vdots=1.5em,fraction=\,] {
            \,
          }{\pi_1}          
        }{ \Gamma  \vdash  \Lnt{t}  :  \Lnt{A} }      
        \\
        $$\mprset{flushleft}
        \inferrule* [right=IRr] {
          \inferrule* [right=] {
            \inferrule* [right=,vdots=1.5em,fraction=\,] {
              \,
            }{\pi_2}          
          }{ \Gamma_{{\mathrm{1}}}  \Lsym{,}  \Lmv{x}  \Lsym{:}  \Lnt{A}  \Lsym{,}  \Gamma_{{\mathrm{2}}}  \Lsym{,}  \Lmv{y}  \Lsym{:}  \Lnt{B}  \vdash  \Lnt{t_{{\mathrm{1}}}}  :  \Lnt{C} }      
        }{ \Gamma_{{\mathrm{1}}}  \Lsym{,}  \Lmv{x}  \Lsym{:}  \Lnt{A}  \Lsym{,}  \Gamma_{{\mathrm{2}}}  \vdash   \lambda_r  \Lmv{y} : \Lnt{A} . \Lnt{t_{{\mathrm{1}}}}   :   \Lnt{B}  \rightharpoonup  \Lnt{C}  }
      }{ \Gamma_{{\mathrm{1}}}  \Lsym{,}  \Gamma  \Lsym{,}  \Gamma_{{\mathrm{2}}}  \vdash  \Lsym{[}  \Lnt{t}  \Lsym{/}  \Lmv{x}  \Lsym{]}   (   \lambda_r  \Lmv{y} : \Lnt{A} . \Lnt{t_{{\mathrm{1}}}}   )   :   \Lnt{B}  \rightharpoonup  \Lnt{C}  }
    \end{math}
  \end{center}
  transforms into the proof
  \begin{center}
    \begin{math}
      $$\mprset{flushleft}
      \inferrule* [right=IRr] {
        $$\mprset{flushleft}
        \inferrule* [right=cut] {
          \inferrule* [right=] {
            \inferrule* [right=,vdots=1.5em,fraction=\,] {
              \,
            }{\pi_1}          
          }{ \Gamma  \vdash  \Lnt{t}  :  \Lnt{A} }      
          \\        
          \inferrule* [right=] {
            \inferrule* [right=,vdots=1.5em,fraction=\,] {
              \,
            }{\pi_2}          
          }{ \Gamma_{{\mathrm{1}}}  \Lsym{,}  \Lmv{x}  \Lsym{:}  \Lnt{A}  \Lsym{,}  \Gamma_{{\mathrm{2}}}  \Lsym{,}  \Lmv{y}  \Lsym{:}  \Lnt{B}  \vdash  \Lnt{t_{{\mathrm{1}}}}  :  \Lnt{C} }      
        }{ \Gamma_{{\mathrm{1}}}  \Lsym{,}  \Gamma  \Lsym{,}  \Gamma_{{\mathrm{2}}}  \Lsym{,}  \Lmv{y}  \Lsym{:}  \Lnt{B}  \vdash  \Lsym{[}  \Lnt{t}  \Lsym{/}  \Lmv{x}  \Lsym{]}  \Lnt{t_{{\mathrm{1}}}}  :  \Lnt{C} }
      }{ \Gamma_{{\mathrm{1}}}  \Lsym{,}  \Gamma  \Lsym{,}  \Gamma_{{\mathrm{2}}}  \vdash   \lambda_r  \Lmv{y} : \Lnt{A} . \Lsym{[}  \Lnt{t}  \Lsym{/}  \Lmv{x}  \Lsym{]}  \Lnt{t_{{\mathrm{1}}}}   :   \Lnt{B}  \rightharpoonup  \Lnt{C}  }
    \end{math}
  \end{center}

\item[Case:] secondary hypothesis: right introduction of implication.
  The proof
  \begin{center}
    \begin{math}
      $$\mprset{flushleft}
      \inferrule* [right=cut] {
        \inferrule* [right=] {
          \inferrule* [right=,vdots=1.5em,fraction=\,] {
            \,
          }{\pi_1}          
        }{ \Gamma  \vdash  \Lnt{t}  :  \Lnt{A} }      
        \\
        $$\mprset{flushleft}
        \inferrule* [right=ILr] {
          \inferrule* [right=] {
            \inferrule* [right=,vdots=1.5em,fraction=\,] {
              \,
            }{\pi_2}          
          }{ \Lmv{y}  \Lsym{:}  \Lnt{B}  \Lsym{,}  \Gamma_{{\mathrm{1}}}  \Lsym{,}  \Lmv{x}  \Lsym{:}  \Lnt{A}  \Lsym{,}  \Gamma_{{\mathrm{2}}}  \vdash  \Lnt{t_{{\mathrm{1}}}}  :  \Lnt{C} }      
        }{ \Gamma_{{\mathrm{1}}}  \Lsym{,}  \Lmv{x}  \Lsym{:}  \Lnt{A}  \Lsym{,}  \Gamma_{{\mathrm{2}}}  \vdash   \lambda_l  \Lmv{y} : \Lnt{A} . \Lnt{t_{{\mathrm{1}}}}   :   \Lnt{C}  \leftharpoonup  \Lnt{B}  }
      }{ \Gamma_{{\mathrm{1}}}  \Lsym{,}  \Gamma  \Lsym{,}  \Gamma_{{\mathrm{2}}}  \vdash  \Lsym{[}  \Lnt{t}  \Lsym{/}  \Lmv{x}  \Lsym{]}   (   \lambda_l  \Lmv{y} : \Lnt{A} . \Lnt{t_{{\mathrm{1}}}}   )   :   \Lnt{C}  \leftharpoonup  \Lnt{B}  }
    \end{math}
  \end{center}
  transforms into the proof
  \begin{center}
    \begin{math}
      $$\mprset{flushleft}
      \inferrule* [right=ILr] {
        $$\mprset{flushleft}
        \inferrule* [right=cut] {
          \inferrule* [right=] {
            \inferrule* [right=,vdots=1.5em,fraction=\,] {
              \,
            }{\pi_1}          
          }{ \Gamma  \vdash  \Lnt{t}  :  \Lnt{A} }      
          \\        
          \inferrule* [right=] {
            \inferrule* [right=,vdots=1.5em,fraction=\,] {
              \,
            }{\pi_2}          
          }{ \Lmv{y}  \Lsym{:}  \Lnt{B}  \Lsym{,}  \Gamma_{{\mathrm{1}}}  \Lsym{,}  \Lmv{x}  \Lsym{:}  \Lnt{A}  \Lsym{,}  \Gamma_{{\mathrm{2}}}  \vdash  \Lnt{t_{{\mathrm{1}}}}  :  \Lnt{C} }      
        }{ \Lmv{y}  \Lsym{:}  \Lnt{B}  \Lsym{,}  \Gamma_{{\mathrm{1}}}  \Lsym{,}  \Gamma  \Lsym{,}  \Gamma_{{\mathrm{2}}}  \vdash  \Lsym{[}  \Lnt{t}  \Lsym{/}  \Lmv{x}  \Lsym{]}  \Lnt{t_{{\mathrm{1}}}}  :  \Lnt{C} }
      }{ \Gamma_{{\mathrm{1}}}  \Lsym{,}  \Gamma  \Lsym{,}  \Gamma_{{\mathrm{2}}}  \vdash   \lambda_l  \Lmv{y} : \Lnt{A} . \Lsym{[}  \Lnt{t}  \Lsym{/}  \Lmv{x}  \Lsym{]}  \Lnt{t_{{\mathrm{1}}}}   :   \Lnt{C}  \leftharpoonup  \Lnt{B}  }
    \end{math}
  \end{center}  

\item[Case:] secondary hypothesis: left introduction of tensor unit.
  The proof 
  \begin{center}
    \begin{math}
      $$\mprset{flushleft}
      \inferrule* [right=cut] {
        \inferrule* [right=] {
          \inferrule* [right=,vdots=1.5em,fraction=\,] {
            \,
          }{\pi_1}          
        }{ \Gamma  \vdash  \Lnt{t}  :  \Lnt{A} }      
        \\
        $$\mprset{flushleft}
        \inferrule* [right=Ul] {
          \inferrule* [right=] {
            \inferrule* [right=,vdots=1.5em,fraction=\,] {
              \,
            }{\pi_2}          
          }{ \Gamma_{{\mathrm{1}}}  \Lsym{,}  \Lmv{x}  \Lsym{:}  \Lnt{A}  \Lsym{,}  \Gamma_{{\mathrm{2}}}  \vdash  \Lnt{t'}  :  \Lnt{B} }      
        }{ \Gamma_{{\mathrm{1}}}  \Lsym{,}  \Lmv{x}  \Lsym{:}  \Lnt{A}  \Lsym{,}  \Gamma_{{\mathrm{2}}}  \Lsym{,}  \Lmv{y}  \Lsym{:}   I   \vdash   \mathsf{let}\, \Lmv{y} \,\mathsf{be}\,  \mathsf{unit}  \,\mathsf{in}\, \Lnt{t'}   :  \Lnt{B} }
      }{ \Gamma_{{\mathrm{1}}}  \Lsym{,}  \Gamma  \Lsym{,}  \Gamma_{{\mathrm{2}}}  \Lsym{,}  \Lmv{y}  \Lsym{:}   I   \vdash  \Lsym{[}  \Lnt{t}  \Lsym{/}  \Lmv{x}  \Lsym{]}   \mathsf{let}\, \Lmv{y} \,\mathsf{be}\,  \mathsf{unit}  \,\mathsf{in}\, \Lnt{t'}   :  \Lnt{B} }
    \end{math}
  \end{center}
  transforms into the proof
  \begin{center}
    \begin{math}
      $$\mprset{flushleft}
      \inferrule* [right=Ul] {
        $$\mprset{flushleft}
        \inferrule* [right=cut] {
          \inferrule* [right=] {
            \inferrule* [right=,vdots=1.5em,fraction=\,] {
              \,
            }{\pi_1}          
          }{ \Gamma  \vdash  \Lnt{t}  :  \Lnt{A} }      
          \\
          \inferrule* [right=] {
            \inferrule* [right=,vdots=1.5em,fraction=\,] {
              \,
            }{\pi_2}          
          }{ \Gamma_{{\mathrm{1}}}  \Lsym{,}  \Lmv{x}  \Lsym{:}  \Lnt{A}  \Lsym{,}  \Gamma_{{\mathrm{2}}}  \vdash  \Lnt{t'}  :  \Lnt{B} }      
        }{ \Gamma_{{\mathrm{1}}}  \Lsym{,}  \Gamma  \Lsym{,}  \Gamma_{{\mathrm{2}}}  \vdash  \Lsym{[}  \Lnt{t}  \Lsym{/}  \Lmv{x}  \Lsym{]}  \Lnt{t'}  :  \Lnt{B} }
      }{ \Gamma_{{\mathrm{1}}}  \Lsym{,}  \Gamma  \Lsym{,}  \Gamma_{{\mathrm{2}}}  \Lsym{,}  \Lmv{y}  \Lsym{:}   I   \vdash   \mathsf{let}\, \Lmv{y} \,\mathsf{be}\,  \mathsf{unit}  \,\mathsf{in}\, \Lsym{[}  \Lnt{t}  \Lsym{/}  \Lmv{x}  \Lsym{]}  \Lnt{t'}   :  \Lnt{B} }
    \end{math}
  \end{center}  
  
\end{itemize}

%% % section cut_elimination_procedure_for_l (end)

\end{document}

http://www.di.ens.fr/~zappa/readings/wmm10.pdf
Sewell, Peter, et al. "Ott: effective tool support for the working semanticist." ACM SIGPLAN Notices. Vol. 42. No. 9. ACM, 2007.
https://www.cs.kent.ac.uk/people/staff/sao/documents/icfp07.pdf


